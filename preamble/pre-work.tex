% ------------------------------------------------------------------------
% LaTeX - Preambel ******************************************************
% ------------------------------------------------------------------------
% pre-work
% ========================================================================
% % ToDo kennzeichnen
\newcommand{\workTodo}[1]{\textcolor{red}{todo: #1}}

% % Für Datum und Zeit in Fusszeile
% % !!!Inhalt bei Fertigstellung der Arbeit löschen
\newcommand{\workMarkDateTime}{}

% % Alle Namen werden im Titel und im hyperref-Paket eingetragen
% % !!! Überall für <Wert> das Entsprechende eintragen

 % <Typ> Studienarbeit, Dipolmarbeit, Studienarbeit oder Bachlor-Abschlussarbeit
\newcommand{\workTyp}{Masterarbeit\xspace}

 % <Titel> der Arbeit
\newcommand{\workTitel}{Fahrspurerkennung in Luftaufnahmen mittels Fahrzeugtrajektorien}

 % <Studiengang> z.B. Kommunikationstechnik
\newcommand{\workStudiengang}{Informatik\xspace}

% <Semester> mit Jahr z.B. Sommersemester 2008
\newcommand{\workSemester}{Wintersemester 2018\xspace}

% <Name> des Studenten
\newcommand{\workNameStudent}{Steffen Schmid\xspace}

% <Pruefer> Name des pr�fenden (betreuenden) Professor an der Hochschule
\newcommand{\workPruefer}{Prof. Dr. Christoph Reich\xspace}


% %%% Nur bei Abschluss-Arbeiten

% <Datum> der Abgabe der Arbeit (Eidesstatliche Erklärung)
\newcommand{\workDatum}{15. Februar 2019\xspace}

% <Zweitpr�fer>
\newcommand{\workZweitPruefer}{}

% <Zeitraum>
\newcommand{\workZeitraum}{01.09.2018 - 28.02.2019\xspace}


% %%% Nur bei Industrie-Arbeiten:

% <Firma>
\newcommand{\workFirma}{IT-Designers GmbH\xspace}

% <Betreuer in der Firma>
\newcommand{\workBetreuer}{Dr. Stefan Kaufmann\xspace}