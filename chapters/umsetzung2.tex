%!TEX root = ../Thesis.tex

\chapter{Fahrspur-Bestimmung aus Trajektorie-Clustern}
\label{cha:lane_definition}
% Cluster-Bereinigung: Entfernen von Outliern (Spurwechselvorgänge)
%   Beschreibung Probleme: Performance, Zuverlässigkeit (initial Distanzbasiert, erweitert Dichtebasiert --> Ähnliche Ergebnisse und Performance)
%   Verfahren mit evtl. besseren Resultaten noch aufwendiger und basierend meist auf selben Ideen (Distanzmaße, Dichten etc.)
% Bestimmung Referenz-Trajektorie
% Bestimmung von Spur-Envelopes
% Partitionierung der initialen Spur-Schätzungen
% Alignment der Spuren

In diesem Kapitel wird beschrieben, wie aus den bereits identifizierten Trajektorie-Clustern
Geometrie-Informationen der Fahrspuren abgeleitet werden. Hierzu sind primär drei Schritte notwendig:
Bestimmung der Spur-Mittellinien, Bestimmung der Spurhüllen und anschließend die Partitionierung sich überlagernder
Spuren. Hinzu kommen weitere kleine Schritte, welche Ebenfalls nachfolgend beschrieben werden.

\section{Ausfilterung von Spurwechselvorgängen}
\label{sec:real2_filter_lane_change}

% Warum?
% Wie?
% Probleme, Andere Lösungen und deren Nachteile

\section{Bestimmung Spurmittellinien}
\label{sec:real2_define_lane_centerline}

% Warum nicht Traj. mit min avg Distanz zu anderen?
% Vorgehen
% Ergebnisse

\section{Bestimmung Spurhüllen}
\label{sec:real2_define_lane_envelope}

% Erklärung SpurDefinition (LaneEstimate) Über Nodes? Siehe Rel. Work.
% Bestimmung benachbarter Spuren
% Bestimmung paralleler Spuren
% Bestimmung des mittleren Abstands zwischen zwei parallelen Spurpaaren
% Bestimmung Envelope-Punkte

\section{Partitionierung von Fahrspuren}
\label{sec:real2_lane_partitioning}

% Arbeiten nur noch mit LaneEstimates
% Grundidee: Sich überschneidende Spuren an Schnittpunkte partitionieren
% Problem: welche Spur ganz lassen und welche partitionieren
% Aufteilung in drei Spur-Typen (isolated, primary, secundary)
% Finden von sich überlappenden Spurpaaren
% Entscheidung für zu partitionierende Spuren (nach Kriterien)
% Partitionierung