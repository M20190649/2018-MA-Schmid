%!TEX root = ../Thesis.tex

\chapter{Fahrspur-Bestimmung aus Trajektorie-Clustern}
\label{cha:lane_definition}

In diesem Kapitel wird beschrieben, wie aus den bereits identifizierten Trajektorie-Clustern
Geometrie-Informationen der Fahrspuren abgeleitet werden. Hierzu sind primär drei Schritte notwendig:
Bestimmung der Spur-Mittellinien, Bestimmung der Spurhüllen und anschließend die Partitionierung sich überlagernder
Spuren. Hinzu kommen weitere kleine Schritte, welche ebenfalls nachfolgend beschrieben werden.

\section{Ausfilterung von Spurwechselvorgängen}
\label{sec:real2_filter_lane_change}

Bevor mithilfe der im vorherigen Schritt gewonnenen Trajektorie-Cluster Mittellinien von Fahrspuren bestimmt
werden können, müssen diese nochmals vorverarbeitet werden. Die einzelnen Cluster enthalten teilweise
Bewegungsbahnen, welche Spurwechselvorgänge oder andere Abweichungen von einer Fahrspur beschreiben.
Diese Trajektorien müssen, so weit wie möglich,
entfernt werden, da sie die anschließende Geometrie-Bestimmung negativ beeinträchtigen. Die Trajektorien
eines Clusters sollten eindeutig einer realen Fahrspur zuzuordnen sein. In Abbildung
\ref{fig:real2_clusters_pre_postpro} sind beispielhaft zwei Cluster dargestellt, welche eine Vielzahl an
Spurwechselvorgängen enthalten.

\begin{figure}[H]
    \centering
    \includegraphics[width=0.8\linewidth]{../resources/img/umsetzung/U2/Clusters_Pre_Postprocessing}
    \caption{Trajektorie-Cluster mit Spurwechselvorgängen}
    \label{fig:real2_clusters_pre_postpro}
\end{figure}

Die Grundidee, welche der Ausfilterung der Abweichungen zugrunde liegt, ist, jene Trajektorien
aus einem Cluster zu entfernen, welche eine überdurchschnittlich hohe mittlere Distanz zu allen anderen
Trajektorien des Clusters besitzen. Als Distanzmaß wird erneut die LCSS Distanz verwendet.
Ein ähnlicher, Distanz-basierter Ausreißer-Detektionsansatz wird in \cite[]{Mirge2017} vorgestellt.
Der in dieser Arbeit verwendete Algorithmus ist in Listing \ref{lst:pseudo_post_processing} beschrieben.

\begin{lstlisting}[caption=Pseudocode Cluster Post-Processing, language=Pseudo, label=lst:pseudo_post_processing]
algorithm filterCluster:
  input:  unfiltered trajectories of cluster: trajsIn
  output: filtered trajectories of cluster

  meanTrajectoryDistances :=
    for each traj in trajsIn do
      yield mean LCSS distance of traj to all other trajectories of trajsIn
    end

  clusterCmpVal := select median of meanTrajectoryDistances as comparison value

  resultTrajs :=
    for each traj in trajsIn do
      if meanDist of traj < 1.5 * clusterCmpVal then
        yield trajs
      end
    end

  return resultTrajs
\end{lstlisting}

Das gewählte Verfahren ist einfach, eignet sich aber dennoch gut, um das gewünschte Ziel zu erreichen.
Die in Abbildung \ref{fig:real2_clusters_post_postpro} dargestellten Ergebnisse zeigen dies.
Aus den in Abbildung \ref{fig:real2_clusters_pre_postpro} enthaltenen Clustern wurden alle Trajektorien
mit Spurwechselvorgängen oder Abweichungen entfernt. Die Effektivität konnte auch durch die Anwendung
auf andere Datensätze bestätigt werden.
Wenn in einem Cluster viele Trajektorien Abweichungen von einer Spur besitzen, ist es möglich, dass der Algorithmus
diese nicht vollständig entfernt. Eine geringe Anzahl von Abweichungen beziehungsweise kleine Ausreißer
sind allerdings unproblematisch. 

\begin{figure}[H]
    \centering
    \includegraphics[width=0.8\linewidth]{../resources/img/umsetzung/U2/Clusters_Post_Postprocessing}
    \caption{Trajektorie-Cluster ohne Spurwechselvorgänge}
    \label{fig:real2_clusters_post_postpro}
\end{figure}

Zu beachten ist auch, dass die Zeitkomplexität des Verfahrens bei großen Trajektorie-Clustern
nicht unerheblich ist. Dies ist auf die Komplexität des LCSS Distanzmaßes zurückzuführen.
Bei der Untersuchung alternativer Vorgehensweise wurde allerdings klar, dass es grundlegend schwer ist,
dieses Problem performant zu lösen. In \cite[]{Meng2018} werden hierzu diverse Distanz- und Dichte-basierte
Arbeiten vorgestellt und verglichen. Da diese alle allerdings nicht weniger komplex sind und häufig
die verfolgten Ziele über die hier geforderten hinausgehen, wurde entschieden den oben beschriebenen Ansatz
beizubehalten. Durch die Parallelisierung der Berechnung der mittleren Abstände konnte die Performance
des Ansatzes außerdem nochmals deutlich gesteigert werden.

\section{Bestimmung der Spurmittellinien}
\label{sec:real2_define_lane_centerline}

Nachdem die Trajektorie-Cluster nun weitestgehend von Fahrspurwechseln und anderen Abweichungen bereinigt
wurden, beschreiben die verbleibenden Bewegungsbahnen eines Clusters idealerweise eine Fahrspur.
Anhand dieser Trajektorien können daher nun die Mittellinie der Spuren bestimmt werden.

Die zu bestimmende Mittellinie verläuft durch die Mitte des entsprechenden
Trajektorie-Clusters. Die einzelnen Bewegungsbahnen besitzen jeweils leichte Abweichungen von dieser
\textit{``Ideallinie''}.
Als Spurmittellinie -- wie von \cite{Hu2005} vorgeschlagen -- jene Trajektorie zu wählen, welche die
kleinste Summe der Distanzen zu allen anderen Trajektorien eines Clusters besitzt, ist nicht praktikabel,
da die so gefunden Trajektorie nicht über ihre komplette Länge in der Mitte des Clusters verlaufen muss.
Auf eine Bestimmung der Mittellinie mittels linearer oder polynomialer Regression, wie beispielsweise in
\cite[]{Chen2014} oder \cite[]{Melo2006},
wurde ebenfalls verzichtet. Der Grund hierfür ist, dass einerseits die Komplexität der Fahrbahnverläufe
nicht bekannt ist und andererseits das Erlernen der Spurrepräsentation zu aufwendig ist.

Der in dieser Arbeit gewählte Ansatz zur Bestimmung der Spurmittellinien macht sich die von Atev et al.
vorgestellte Notation einer relativen Position innerhalb einer Trajektorie zu nutze, welche in
Gleichung \ref{eq_atev_relPos} und \ref{eq_atev_findPointAtRelPow} gegeben ist.
Die Koordinaten der Spurlinien ergeben sich aus den Mittelwerten von Trajektorie-Punkten, welche sich
alle an der selben relativen Position befinden. Der verwendete Algorithmus ist nachfolgend beschrieben.
\begin{lstlisting}[caption=Pseudocode Cluster Post-Processing, language=Pseudo, label=lst:pseudo_post_processing]
algorithm calculateCenterline:
  input:  trajectories of cluster: trajsIn
  output: centerline of lane

  meanTrajLength := calculate mean point length of trajectories
  relPositions := define range with relative positions of size meanTrajLength
                  and step-size (1 / meanTrajLength)

  centerline :=
    for each relP in relPositions do
      pointsAtRelPos := get points at relP for each traj in trajsIn
      yield mean of all points in pointsAtRelPos
    end

  return centerline
\end{lstlisting}

Da die Form und der Verlauf von Trajektorien eines Clusters sich nur wenig unterscheiden, liefert dieses
Vorgehen gute Ergebnisse. In Abbildung \ref{fig:real2_results_centerline_detection} a) ist beispielhaft
dargestellt, wie eine Mittellinien innerhalb eines Trajektorie-Clusters verlaufen.
In \ref{fig:real2_results_centerline_detection} b) sind alle Spurmittellinien der Neckartor-Kreuzung,
welche auf die oben beschriebene Weise bestimmt wurden, abgebildet.

% TODO: Bild mit Spurmittellinien in Aufnahme
\begin{figure}[H]
    \centering
    \subfloat[]{{
        \includegraphics[align=c, width=0.45\linewidth]{../resources/img/umsetzung/U2/cluster_with_centerline}
    }}
    \qquad
    \subfloat[]{{
        \includegraphics[align=c, width=0.4\linewidth]{../resources/img/umsetzung/U2/laneCenters}
    }}
    \caption{Spurmittellinie in einem Trajektorie-Cluster a), Spurmittellinien Neckartor-Kreuzung b)}
    \label{fig:real2_results_centerline_detection}
\end{figure}

Die Mittellinien werden, da es sich bei ihnen ebenfalls nur um Sequenzen von 2D-Koordinaten handelt,
als Trajektorie-Objekte repräsentiert (siehe Abbildung \ref{fig:real_trajectory_classDia}).

\section{Bestimmung der Spurhüllen}
\label{sec:real2_define_lane_envelope}

Nachdem im vorherigen Schritt die Mittellinien der Fahrspuren bestimmt wurden, wird nun für jede Spurlinie
eine Hülle definiert. In folgendem Abschnitt wird beschrieben, wie diese erstellt werden.

Die Spurhülle bestimmt die Dimension einer Fahrbahn. In dieser Arbeit verlaufen sie idealerweise entlang
realer Begrenzungslinien welche zwei Spuren voneinander oder eine Fahrbahn von einem Seitenstreifen et cetera trennen.
Eine Mittellinie und eine Hülle beschreiben zusammen die Geometrie einer Fahrspur. Diese Geometrie
soll die realen Dimensionen einer Spur möglichst genau abbilden. Aus diesem Grund ist es nicht möglich,
die Hüllen lediglich auf Basis von statistischen Informationen der Trajektorie-Cluster zu bestimmen,
wie das beispielsweise in \cite[]{WeimingHu2006} oder \cite[]{Morris2011} gemacht wird.
Der in dieser Arbeit verwendete Ansatz zur Bestimmung der Spurhüllen basiert daher lediglich auf den
Spurmittellinien und nutzt ihre relative Lage zueinander. Ansätze hierzu stammen aus den Arbeiten von
\cite[]{Hsieh2006} und \cite[]{Makris2005}, welche in Abschnitt \ref{sec:rw_lane_detection} vorgestellt wurden.

Die grundlegende Idee, auf welcher die Bestimmung der Spurhüllen basiert, ist konzeptionell in Abbildung
\ref{fig:real2_envelope_definition_concept} dargestellt.
Für zwei parallel zueinander verlaufende Spuren $l_1$ und $l_2$ werden die Hüllen, über den Abstand zwischen
ihren Mittellinien, bestimmt. Besitzen die Linien den Abstand $d$ zueinander, so beträgt die Breite
der Spuren ebenfalls $d$. Der Abstand $e_d$ zwischen einer Mittellinie und einer Hüll-Linie beträgt folglich
$1/2\ d$.

\begin{figure}[H]
    \centering
    \includegraphics[width=0.75\linewidth]{../resources/img/umsetzung/U2/concept_lane_envelope}
    \caption{Bestimmung Spürhüllen}
    \label{fig:real2_envelope_definition_concept}
\end{figure}

Die oben beschriebene Definition einer Bahn-Geometrie wird im Spurerkennungs-Modul über eine Klasse
\textit{LaneGeometry} repräsentiert. Diese ist in Abbildung \ref{fig:real2_laneGeometry_ClassDia} dargestellt.
Das Feld \textit{variance} entspricht dem Abstand $e_d$ zwischen der \textit{centerline} und den Hüll-Linien.

\begin{figure}[H]
    \centering
    \includegraphics[width=0.38\linewidth]{../resources/img/umsetzung/U2/LaneGeometry_ClassDia}
    \caption{Aufbau LaneGeometry Klasse}
    \label{fig:real2_laneGeometry_ClassDia}
\end{figure}

Einen ähnlichen Ansatz zur Bestimmung von Spurbegrenzungslinien verwenden auch Hsieh et al. in ihrer Arbeit.
Sie gehen jedoch, da die von ihnen verwendeten
Aufnahmen von einer statischen Kamera über einer Autobahn stammen, davon aus, dass alle Fahrspuren parallel
zueinander verlaufen. Da das in dieser Arbeit entwickelte Spurerkennungs-Modul mit unterschiedlichen
Aufnahmen und Straßentopologien umgehen können muss, kann diese Annahme nicht getroffen werden.
Nachfolgend werden die Schritte vorgestellt, welche angewandt werden, um in einer Menge von Spuren
jene zu finden, welche parallel zueinander verlaufen, auf Basis dieser die Spurbreiten zu bestimmen und
anschließend die Hüll-Linien zu definieren.

\subsection*{Identifikation paralleler Fahrspuren}
% beschreibung finden benachbarter Fahrspuren (Vorauswahl)
% Beschreibung bestimmung Paralleler Fahrspuren
%   mit PseudoCode?!

Um in einer Menge von Fahrspuren, welche über Mittellinien repräsentiert werden, jene zu finden, die
parallel zueinander verlaufen, werden in einem ersten Schritt benachbarte Spur-Paare gesucht.
Zwei Fahrspuren $l1$ und $l2$ gelten als benachbart, wenn sich der Start oder das Ende
von Spur $l2$ in einem Bereich mit Radius $\sigma$ um den Start von $l1$ befindet.
Der Wert für $\sigma$ wurde auf Basis der in Deutschland geltenden
\textit{``Richtlinien für die Anlage von Autobahnen''} (\acrshort*{raa}) \cite[]{RAA2008}
und der \textit{``Richtlinien für die Anlage von Landstraßen''} (\acrshort*{ral}) \cite[]{RAL2012} bestimmt. Diese Regelwerke spezifizieren
die in der Bundesrepublik derzeit zulässigen Straßenquerschnitte. Die in ihnen festgelegten
Spurbreiten variieren zwischen 3.25 und 3.75 Metern.
Um auch besonders breite benachbarte Spur-Paare identifizieren zu können, wurde für $\sigma$ der Wert 4m gewählt.

Die auf diese Weise gefundenen Trajektorie-Paare starten oder enden als benachbarte Spuren.
Der Algorithmus welcher prüft, ob zwei Spuren tatsächlich parallel zueinander verlaufen, ist in Listing
\ref{lst:pseudo_checkParallel} beschrieben. Die zu vergleichenden Mittellinien werden jeweils in eine
Reihe von Richtungsvektoren umgewandelt. Anschließend werden die paarweisen Differenzen zwischen den Vektoren
berechnet und gemittelt.
Liegt die sich so ergebende Abweichung unter einem Grenzwert $\delta$, so handelt es sich um parallele Spuren.
\begin{lstlisting}[caption=Pseudocode Überprüfung der Parallelität zweier Mittellinien, language=Pseudo, label=lst:pseudo_checkParallel]
algorithm lanesAreParallel:
  input:  lane-centerline: l1, lane-centerline: l2, delta
  output: True or False

  inOppositeDirections := check if l1 and l2 run in opposite directions
  if inOppositeDirections then
    reverse points of centerline l2
  end

  l1DirectionVec := calculate direction vectors for l1
  l2DirectionVec := calculate direction vectors for l2
  dirDifferences := calculate pairwise difference between direction vectors

  meanDiff := calculate mean divergence for X- and Y-axis components
  lengthDiff := calculate difference between length of l1 and l2

  return meanDiff < delta && lengthDiff < 10.0
\end{lstlisting}

Für $\delta$ wurde experimentell der Wert 0.1 bestimmt. In den verschiedenen Test-Datensätzen konnten
so zuverlässig parallele Spur-Paare identifiziert werden. Anhand dieser werden im nächsten Schritt
die Spurhüllen bestimmt.

\subsection*{Berechnung der Hüllen}

% Bestimmung der mittleren Distanzen zwischen Spuren, Berechnung der Spurvarianz
% Spuren ohne Nachbar erhalten standard-Varianz
% Berechnung der Hüllen, Formeln, Referenz auf Abb. oben

Bevor für eine Mittellinie die sie umgebende Spurhülle bestimmt werden kann, muss die Breite der Spur
ermittelt werden. Diese ergibt sich, für die im vorherigen Schritt bestimmten parallelen Spurpaare, aus
deren mittleren Abstand zueinander. Verlaufen zu einer Spurlinie zwei Bahnen parallel, werden
die Abstände zu beiden berechnet und der kleinere wird als Spurbreite gewählt.
Allen Bahnen, welche keine parallele Nachbarspur besitzen, wird die minimale Spurbreite zugeordnet,
welche im vorherigen Schritt bestimmt wurde.
Falls sich in einer Aufnahme keine parallelen Spuren befinden, wird für die Spurbreite ein Standartwert
von 3.5 Metern verwendet, welcher sich ebenfalls aus den RAA und RAL Richtlinien ableitet.

Nachdem auf die oben beschriebene Weise Breiten für alle Fahrspuren bestimmt wurden, können die Hüll-Linien
berechnet werden. Hierzu werden für jeden Punkt der Mittellinie zwei zugehörige Hüll-Punkte berechnet.
Das hierzu verwendete Vorgehen ist in Abbildung \ref{fig:real2_envelope_definition_concept} dargestellt.

\begin{figure}[H]
    \centering
    \includegraphics[width=0.5\linewidth]{../resources/img/umsetzung/U2/calc_env_point}
    \caption{Bestimmung Spürhüllen}
    \label{fig:real2_envelope_definition_concept}
\end{figure}

Um die Hüll-Punkte für einen Punkt $B$ einer Mittellinie zu bestimmen, wird eine Gerade $g$ durch die
Punkte $A$ und $C$ gelegt, welche sich vor und hinter $B$ befinden.

\begin{ceqn}
\begin{align}
    g: \vec{x} = \overrightarrow{OA} + t \cdot \overrightarrow{AC}
\end{align}
\end{ceqn}

Anschließend wird durch $B$ eine Gerade $h$ gelegt, welche orthogonal zu $g$ verläuft.

\begin{ceqn}
\begin{align}
    h: \vec{x} = \overrightarrow{OB} + t \cdot \hat{X}
\end{align}
\end{ceqn}

Für ihren Richtungsvektor $\hat{X}$, welcher sich aus $\overrightarrow{AC}$ ergibt, gilt
$\overrightarrow{AC} \cdot \hat{X} = 0$. Da $\hat{X}$ ein Einheitsvektor
ist, können die Hüll-Punkte, welche auf $h$ liegen und den Abstand $e_d$ von $B$ besitzten, einfach
bestimmt werden, indem $e_d$ beziehungsweise $-e_d$ für $t$ in die Geradengleichung von $h$ eingesetzt wird.

Auf diese Weise werden die Hüll-Linien für alle Fahrspuren bestimmt. In Abbildung \ref{fig:real2_results_geometry_definition} sind die
Ergebnisse der Spur-Geometrie-Bestimmung dargestellt. Teil a) zeigt die Spurmittellinien mit ihren
sie umgebenden Hüllen. Teil b) zeigt die in die TrackerApplication eingefügten Spuren. 

\begin{figure}[H]
    \centering
    \subfloat[]{{
        \includegraphics[align=c, width=0.4\linewidth]{../resources/img/umsetzung/U2/laneEstimates}
    }}
    \qquad
    \subfloat[]{{
        \includegraphics[align=c, width=0.45\linewidth]{../resources/img/umsetzung/U2/result_laneEst_Screenshot}
    }}
    \caption{Plot Spurmittellinien und Hüllen a), Ergebnis Spur-Geometrien in TrackerApplication b)}
    \label{fig:real2_results_geometry_definition}
\end{figure}

Die obige Abbildung zeigt, dass die berechneten Spur-Geometrien in den meisten Fällen gut mit den realen
Spur-Dimensionen übereinstimmen. Problematisch sind allerding die Überlagerungen der Spuren in manchen
Regionen. Diese werden im nächsten Schritt entfernt.

\section{Partitionierung von Fahrspuren}
\label{sec:real2_lane_partitioning}

% Arbeiten nur noch mit LaneEstimates
% Grundidee: Sich überschneidende Spuren an Schnittpunkte partitionieren
% Problem: welche Spur ganz lassen und welche partitionieren
% Aufteilung in drei Spur-Typen (isolated, primary, secundary)
% Finden von sich überlappenden Spurpaaren
% Entscheidung für zu partitionierende Spuren (nach Kriterien)
% Partitionierung

Fahrspuren, welche mithilfe dieser Arbeit erkannt werden, kommen bei der Verkehrsanalyse zum Einsatz.
Unter anderem können mit ihrer Hilfe Spurwechselvorgänge von Fahrzeugen untersucht werden. Damit eine
solche Analyse sinnvoll ist, dürfen sich Fahrspuren nicht über einen größeren Bereich hinweg überlagern.
Aus diesem Grund werden Spuren, welche in Teilen identisch mit anderen verlaufen, partitioniert. Die
identischen Teile werden verworfen und nur die separaten beibehalten.
Unproblematisch ist es, wenn sich Spuren lediglich kreuzen, ihre Überlagerung also geringfügig ist.
In diesem Fall werden die Spuren nicht partitioniert. Es wird daher zwischen sich überlagernden und sich
kreuzenden Spuren unterschieden. Nachfolgend wird das Verfahren
zur Identifikation von sich überlagernden Fahrspuren und der Partitionierungs-Vorgang beschrieben.

Um Fahrspuren zuverlässig und sinnvoll partitionieren zu können, müssen grundlegend zwei Probleme bewältigt
werden. Es müssen zuerst die sich überlagernden Spur-Paare und deren Schnittpunkte gefunden werden und
anschließend muss für jedes Paar entschieden werden, welche Spur partitioniert wird und welche erhalten bleibt.
Hierzu wird zuerst die Menge der Spur-Geometrien in drei Kategorien unterteilt:

\begin{itemize}
    \item isolierte Fahrspuren
    \item primäre Fahrspuren
    \item sekundäre Fahrspuren
\end{itemize}

Isolierte Fahrspuren sind hierbei jene, welche keine Überschneidung mit anderen Spuren besitzen. Primäre
Fahrspuren besitzen keine Überschneidungen untereinander, können sich aber mit anderen Spuren kreuzen oder
von ihnen überlagert werden. In einer Menge von Spur-Geometrien bildet das größte Subset von parallel
zueinander verlaufenden Spuren die primären Spuren. Sekundäre Fahrspuren sind all jene, welche
weder isoliert noch primär sind. In Abbildung \ref{fig:real2_prim_and_sec_lanes} a) sind die primären und
sekundären Spur-Geometrien der Neckartor-Kreuzung dargestellt.

\begin{figure}[H]
    \centering
    \subfloat[]{{
        \includegraphics[align=c, width=0.35\linewidth]{../resources/img/umsetzung/U2/prims_and_secs}
    }}
    \qquad
    \subfloat[]{{
        \includegraphics[align=c, width=0.35\linewidth]{../resources/img/umsetzung/U2/overlapping_lanes}
    }}
    \caption{a) Primäre (blau) und sekundäre (grün) Spuren, b) sich überlagernde und kreuzende Spur-Paare}
    \label{fig:real2_prim_and_sec_lanes}
\end{figure}

Nachdem die Spuren, anhand der oberen Definition, in die drei Kategorien unterteilt wurden, werden
anschließend die primären und sekundären Spuren nach sich überlagernden Paaren durchsucht.
Zwei Spuren überschneiden sich, wenn, wie in
Abbildung \ref{fig:real2_lane_crossing} zu sehen, die Mittellinie einer Spur innerhalb der Hülle einer
anderen Spur liegt. Die Punkte $b_1$ und $b_2$ entsprechen hierbei den äußeren und inneren Grenzpunkten
der Überschneidung der Mittellinie von $l_2$ mit der Hülle von $l_1$. Zur Bestimmung der Schnittmenge
wird wieder die JTS Topology Suite eingesetzt.

\begin{figure}[H]
    \centering
    \includegraphics[width=0.5\linewidth]{../resources/img/umsetzung/U2/lane_crossing}
    \caption{Überlagerung zweier Spur-Abschnitte}
    \label{fig:real2_lane_crossing}
\end{figure}

Zwei Spuren $l_1$ und $l_2$ kreuzen sich nicht nur, sondern überlagern sich, wenn der Abschnitt zwischen
den Grenzpunkten $b_1$ und $b_2$ mindestens 10\% der Spurlänge ausmacht. Auf diese Weise werden die
sich überlagernden Spur-Paare und die zugehörigen Schnittpunkte bestimmt.
Aus jeder Überschneidung ergeben sich zwei Spur-Paare.
Im Fall von Abbildung \ref{fig:real2_lane_crossing} wird einerseits $l_1$ von $l_2$ überlagert und andererseits
$l_2$ von $l_1$.  
Daher enthält Abbildung \ref{fig:real2_prim_and_sec_lanes} b) beispielsweise vier sich überlagernde Spur-Paare
(blau-grau) und zwei sich kreuzende Paare (grau-grau).

Nachdem auf die oben beschriebene Weise die sich überlagernden Spur-Paare bestimmt wurden, wird anschließend
entschieden, welche Spur der Paare partitioniert wird und welche erhalten bleibt. Zuerst wird hierzu
überprüft, ob es sich bei einer der Geometrien um eine primäre Fahrspur handelt. Ist dies der Fall, so
bleibt diese vollständig. Existiert in einem Paar keine primäre Spur, so wird das Krümmungsverhalten der Spuren
um ihren Schnittpunkt herum untersucht. Das Ziel ist es, jene Spur, welche eine stärker Krümmung besitzt,
zu partitionieren. Es handelt sich bei ihr mit höherer Wahrscheinlichkeit um eine Abbiegespur et cetera,
welche aus einer geraden Spur hervorgeht oder in eine solche übergeht. Der Algorithmus zur Auswahl
der gekrümmteren Spur ist in Listing \ref{lst:pseudo_selectMoreCurvy} beschrieben.
\begin{lstlisting}[caption=Pseudocode Auswahl gekrümmtr Fahrspur, language=Pseudo, label=lst:pseudo_selectMoreCurvy]
algorithm selectMoreCurvyLane:
  input:  lane-geo: l1, lane-geo: l2, boundPoints: bounds
  output: l1 or l2 based on curviness around bounds

  innerBoundPoint := select inner bound point based on distance
                     from b1 and b2 to the edges of l1

  l1Subset := get subset of l1 around innerBoundPoint (length 30 points)
  l2Subset := get subset of l2 around innerBoundPoint (length 30 points)

  l1SubCurvMea := estimate curvature of l1Subset 
  l2SubCurvMea := estimate curvature of l2Subset 

  if l1SubCurvMea >= l2SubCurvMea then
    return l1
  else
    return l2
  end
\end{lstlisting}

Im Fall der sich überlagernden Spurpaare in Abbildung \ref{fig:real2_lane_crossing} ist $b_2$ der \textit{innerBoundPoint}.
Zur Bestimmung der Krümmung einer Fahrspur in einem Bereich, siehe Listing \ref{lst:pseudo_selectMoreCurvy}
Zeile 8 + 9, wird der Winkel $\varphi$ zwischen den Richtungsvektoren des Anfang und Endes der Teilspur berechnet.
Er ergibt sich anhand Gleichung \ref{eq_angle_skalar}. Auf Basis dieses einfachen Krümmungsmaßes wird entschieden,
welche der zwei sekundär-Spuren partitioniert wird.

\begin{ceqn}
\begin{align}
\label{eq_angle_skalar}
    \varphi=\arccos \frac{\vec a \cdot \vec b}{|\vec a| |\vec b|}
\end{align}
\end{ceqn}

Nachdem, auf die oben beschriebene Weise, alle zu teilenden Spuren und die zugehörigen Grenzpunkte
bestimmt wurden, folgt die eigentliche Partitionierung. Aus den Spur-Geometrien werden alle Bereiche
entfernt, welche zwischen den zwei Grenzpunkten einer Überlagerung liegen. In Abbildung \ref{fig:real2_lane_crossing}
wird so beispielsweise der rot gekennzeichnete Bereich der Spur $l_2$ zwischen $b_1$ und $b_2$ entfernt.

Abbildung \ref{fig:real2_results_partitioning} zeigt das Ergebnis der Spur-Partitionierung im Fall des
Neckartor Datensatzes. Es wurden alle Spur-Überlagerungen entfernt.

\begin{figure}[H]
    \centering
    \subfloat[]{{
        \includegraphics[align=c, width=0.4\linewidth]{../resources/img/umsetzung/U2/partitionedLanes}
    }}
    \qquad
    \subfloat[]{{
        \includegraphics[align=c, width=0.45\linewidth]{../resources/img/umsetzung/U2/result_lanePartitioned_Screenshot}
    }}
    \caption{Plot partitionierte Spur-Geometrien a), Ergebnis in der TrackerApplication b)}
    \label{fig:real2_results_partitioning}
\end{figure}