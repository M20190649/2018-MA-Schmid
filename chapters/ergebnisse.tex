%!TEX root = ../Thesis.tex

\chapter{Auswertung und Ergebnisse}
\label{cha:results}

Die Stärken, Schwächen und Ergebnisse des entwickelten Algorithmus werden im nachfolgenden Kapitel
zusammengefasst, diskutiert und ausgewertet. Es wird zuerst abschnittsweise auf die drei primären Schritte
des entwickelten Spurerkennungsverfahrens eingegangen. Endergebnisse werden anschließend
in Form von Screenshots vorgestellt, in welchen die erkannten Fahrspuren zu sehen sind.

\section{Evaluierung der Datenvorverarbeitung}
\label{sec:results_eval_dataprocessing}

Die Datenvorverarbeitung ist ein wichtiger Teilschritt bei der Erkennung von Fahrspuren. Nur wenn aus
den Roh-Trajektorien die meisten Defekte entfernt wurden, können die nachfolgenden
Schritte zuverlässig funktionieren. Die angewandten Schritte zur
Entfernung der Defekte sind in Abschnitt \ref{sec:realisation_preprocessing} beschrieben.

Anhand von Abbildung \ref{fig:real_result_2nd_Prepro} im Umsetzungskapitel wurde bereits gezeigt,
dass die Datenvorverarbeitung grundsätzlich funktioniert.
In Abbildung \ref{fig:results_prePro_heilbronner} sind nun die Roh-Trajektorien und bereinigten Bewegungsbahnen
eines weiteren Datensatzes dargestellt, welcher von der Heilbronner-Straße in Stuttgart stammt.

\begin{figure}[H]
    \centering
    \subfloat[]{{
        \includegraphics[align=c, width=0.33\linewidth]{resources/img/results/Heilbronner/rawTrajectories}
    }}
    \qquad \qquad
    \subfloat[]{{
        \includegraphics[align=c, width=0.33\linewidth]{resources/img/results/Heilbronner/preProTrajs}
    }}
    \caption[Ergebnis Trajektorie-Vorverarbeitung der Heilbronner-Straße]
            {Fahrzeugtrajektorien im Rohformat a), vorverarbeitete Trajektorien b) - Datensatz \textit{Heilbronner-Straße}}
    \label{fig:results_prePro_heilbronner}
\end{figure}

Die Plots zeigen gut, dass die vielen in a) vorkommenden Defekte entfernt wurden. Von den
circa 1050 Roh-Trajektorien im ursprünglichen Datensatz bleiben nach der Vorverarbeitung etwa 450 intakte
Bewegungsbahnen übrig. Die Mehrzahl der Defekte in diesem Fall stammt von \textit{False-Positive}-Detektionen
und unterbrochener Fahrzeug-Detektionen (siehe Abschnitt \ref{sec:grund_challenges_reconstruction}).

Ein problematisches Verhalten der Datenvorverarbeitung wurde beim Testen der Spurerkennung anhand eines
Datensatzes aus Steinheim deutlich. In Abbildung \ref{fig:results_horizon_problem} a) ist der
untersuchte Straßenabschnitt dargestellt.

\begin{figure}[H]
    \centering
    \includegraphics[width=0.5\linewidth]{resources/img/results/Steinheim/steinheim}
    \caption[Straßenabschnitt Datensatz Steinheim]
            {Straßenabschnitt Datensatz Steinheim mit am Horizont verschwindenden Fahrspuren}
    \label{fig:results_horizon_problem}
\end{figure}

Schwierig an dieser Aufnahme ist, dass die Fahrzeuge, welche sich auf der oben links startenden Fahrbahn bewegen,
am Anfang beziehungsweise Ende ihrer Fahrt sehr klein sind. Da Fahrzeuge in einer Videoaufnahme ab einer gewissen Größe
nurnoch sehr unzuverlässig detektiert werden, brechen die Trajektorien in diesem Fall auf sehr
unterschiedlichen Höhen ab.
Aufgrund der stark variierenden Start- und End-Positionen, entfernt der in Abschnitt
\ref{sec:real1_remove_broken_trajectories} beschriebene Algorithmus zur Identifikation unterbrochener
Trajektorien, auch die meisten Bewegungsbahnen von Fahrzeugen, welche sich auf den Horizont zu oder von im weg bewegen.
Somit können für diese Fahrspuren auch keine Spur-Geometrien bestimmt werden.
Dieses Verhalten kann immer dann auftreten, wenn Fahrbahnen auf einen von der Kamera weit entfernten Horizont zulaufen.

Da Fahrzeuge im Bereich eines solchen Horizonts grundsätzlich unzuverlässig erkannt werden, ist auch eine Fahrverhaltensanalyse
mithilfe von Fahrspuren hier nicht sinnvoll. Es wurde daher entschieden, dem Anwender die Möglichkeit zu geben, eine
Horizont-Linie zu definieren. Diese sollte vor der Höhe liegen, ab welcher die Fahrzeugerkennung unzuverlässig wird.
In einem ersten Vorverarbeitungsschritt werden alle Trajektorie-Punkte oberhalb dieser
Linie entfernt. Dank der Beschneidung der Trajektorien bleiben diese in den nachfolgenden Verarbeitungsschritten
erhalten und es können Spuren im gewünschten Ausschnitt erkannt werden.

Mit Ausnahme des unerwünschten Verhaltens im Fall von Trajektorien, welche am Horizont sehr unterschiedliche Start-
beziehungsweise End-Positionen besitzen, funktioniert die Datenvorverarbeitung zuverlässig und wie gewünscht.
Voraussetzung dafür, dass nach der Vorverarbeitung noch genug Trajektorien vorliegen, welche weiterverarbeitet werden können,
ist, dass in den Rohdaten ausreichend intakte Bewegungsbahnen vorhanden sind.

\section{Evaluierung der Clusteranalyse}
\label{sec:results_eval_clustering}

Die Clusteranalyse der Trajektorien bildet die Grundlage für die anschließende Bestimmung der Spur-Geometrien.
Sie ist daher ein kritischer Bestandteil der Spurerkennung. Der in dieser Arbeit eingesetzte Ansatz zur
Gruppierung der Trajektorien wurde in Abschnitt \ref{sec:real_ansatz_dbscan_lcss} beschrieben.
Hier wurden auch bereits Ergebnisse der Clusteranalyse für die Datensätze \textit{Esslingen} und
\textit{Neckartor} vorgestellt.

Da der angewandte Clustering-Algorithmus die vollständigen Verläufe der
Trajektorien vergleicht, ist die wichtigste Vorraussetzung zur Identifikation eines Clusters, dass ausreichend
Trajektorien vorliegen, welche eine identische Bewegung durch einen Straßenabschnitt beschreiben.
In Abschnitt \ref{sec:results_clustering_dbscan_lcss} wurde erwähnt, dass ein Cluster mindestens aus
fünf Trajektorien bestehen muss. Diese Untergrenze wurde gewählt, um nicht zu viele Cluster zu identifizieren, welche
eigentlich keine Fahrspur beschreiben. Würde die Grenze niedriger angesetzt werden, so würden beispielsweise
vermehrt Cluster aus Trajektorien gebildet, welche Überholvorgänge beschreiben.

In den meisten Datensätzen, anhand derer der in dieser Arbeit entwickelte Algorithmus getestet wurde,
existierten ausreichend Trajektorien pro Fahrspur, um diese zuverlässig zu identifizieren. Eine Ausnahme
stellt der Datensatz von der Heilbronner-Straße in Stuttgart dar, dessen Trajektorien bereits
in Abschnitt \ref{sec:results_eval_dataprocessing} dargestellt sind. In Teil b) dieser Abbildung
wird deutlich, dass die Fahrspuren im unteren Bereich des Straßenabschnitts nur wenig befahren sind.
Dies wirkt sich auch auf das Ergebnis der Clusteranalyse aus, welches in Abbildung \ref{fig:results_clusters_heilbronner} dargestellt ist.
Zwar existieren ausreichend Trajektorien, welche sich von oben nach unten links bewegen, allerdings können
keine Gruppen aus den Trajektorien gebildet werden, welche sich unten rechts befinden.
Die wenigen in diesem Bereich existierenden Trajektorien haben sehr unterschiedliche Bewegungsbahnen,
weshalb hier kein Cluster identifiziert wird.

\begin{figure}[H]
    \centering
    \includegraphics[width=0.4\linewidth]{resources/img/results/Heilbronner/filteredClusters_Heilbronner}
    \caption{Ergebnis Trajektorie-Cluster der Heilbronner-Straße}
    \label{fig:results_clusters_heilbronner}
\end{figure}

Die Plots in Abbildung \ref{fig:results_clusters_neckartor} a) und b) zeigen das Ergebnis der Clusteranalyse für
den \textit{Neckartor}-Datensatz, aus welchem vor Anwendung des Algorithmus zufällig 50\% der Trajektorien entfernt wurden.

\begin{figure}
    \centering
    \subfloat[]{{
        \includegraphics[align=c, width=0.4\linewidth]{resources/img/results/Neckartor/filteredClusters1}
    }}
    \qquad
    \subfloat[]{{
        \includegraphics[align=c, width=0.4\linewidth]{resources/img/results/Neckartor/filteredClusters2}
    }}
    \caption[Ergebnisse Test Clusteranalyse auf Neckartor Datensatz]
            {Trajektorie-Cluster des Neckartor-Datensatzes nach zufälliger Entfernung von 50\% der Trajektorien}
    \label{fig:results_clusters_neckartor}
\end{figure}

Auch hier wird deutlich, dass das Ergebnis maßgeblich von der Anzahl der Trajektorien pro Spur abhängt.
In Plot a) wurden vorwiegend vertikal verlaufende Trajektorien entfernt, weshalb hier nur wenige Spurcluster identifiziert werden.
In Plot b) wurden hingegen, trotz des Fehlens von 50\% der Trajektorien, fast alle Spurcluster erkannt
(vergleiche Abbildung \ref{fig:real_turning_lane} c)).

Außer von der Anzahl der Trajektorien, hängt das Ergebnis der Clusteranalyse primär von den gewählten
Parametern des DBSCAN Clusteralgorithmus ab. Die Standardparameter für das Clustering-Verfahren wurden
in Abschnitt \ref{sec:results_clustering_dbscan_lcss} definiert. In Abschnitt \ref{sec:real1_adjustment_clustering_parameter}
wurde zudem ein Verfahren vorgestellt, mit dessen Hilfe der Parameter $\epsilon_{DBSCAN}$
automatisch angepasst wird, wenn die Qualität der ermittelten Spur-Cluster nicht den Erwartungen entspricht.
Alle bislang in der Arbeit gezeigten Ergebnisse der Clusteranalysen wurden ohne manuelle Parameteranpassungen bestimmt.
Die Spur-Cluster beschreiben die in einem Straßenabschnitt existierenden Fahrspuren meist sehr gut,
weshalb auch die daraus abgeleiteten Spur-Geometrien die Fahrbahnen gut abbilden.

In speziellen Fällen kann es allerdings vorkommen, dass trotz schlechter Clustering-Ergebnisse
keine automatische Neuparametrisierung und Durchführung der Clusteranalyse vorgenommen wird.
In dieser Situation können nur über eine manuelle Parametrisierung die Spur-Cluster zuverlässig identifiziert werden.
Dies trifft beispielsweise auf den Datensatz \textit{Steinheim} zu. Die unter Einsatz der Standardparameter
ermittelten Spur-Cluster sind in Abbildung \ref{fig:results_clusters_steinheim} a) dargestellt.

\begin{figure}
    \centering
    \subfloat[]{{
        \includegraphics[align=c, width=0.35\linewidth]{resources/img/results/Steinheim/filteredClusters_03}
    }}
    \qquad \qquad
    \subfloat[]{{
        \includegraphics[align=c, width=0.35\linewidth]{resources/img/results/Steinheim/filteredClusters_015}
    }}
    \caption[Ergebnisse Clusteranalyse Steinheim]
            {a) Ergebnis Standard-Clusteranalyse ($\epsilon_{DBSCAN} = 0.3$),
            b) Ergebnis Clusteranalyse nach manuelle Anpassung von $\epsilon_{DBSCAN}$ auf 0.1}
    \label{fig:results_clusters_steinheim}
\end{figure}

Da die Spur-Geometrien, welche anhand der in Abbildung \ref{fig:results_clusters_steinheim} a) zu sehenden
Cluster gebildet werden, jeweils nur eine Fahrspur des Straßenabschnittes \textit{Steinheim} beschreiben,
existieren in den abgeleiteten Weg-Zeit-Trajektorien keine Überschneidungen. Die Spurerkennung wird daher
nicht automatisch unter Verwendung eines angepassten Parameters wiederholt (siehe Abschnitt \ref{sec:real1_adjustment_clustering_parameter}).
Um die zweite, nach hinten verlaufende Fahrspur zu identifizieren, muss $\epsilon_{DBSCAN}$ manuell angepasst werden.
Das Ergebnis ist in Abbildung \ref{fig:results_clusters_steinheim} b) dargestellt.
Da die Trajektorien auf den nach hinten verlaufenden Fahrspuren aufgrund der drei Brücken
(siehe Abbildung \ref{fig:results_horizon_problem}) sehr viele Unterbrechungen
aufweisen, kann der Clusteralgorithmus für die linke Spur zudem nur ein Cluster identifizieren, welches bis
knapp vor die Brücken verläuft.

Für die Clusteranalyse kann zusammenfassend festgehalten werden, dass sie gut funktioniert, wenn ausreichend
intakte Trajektorien zur Beschreibung einer Spur vorliegen. Dank sorgfältig gewählter Standardparameter
und einer automatischen Parametrisierung können in den meisten Fällen die korrekten Spur-Cluster voll automatisch
ermittelt werden. Nur in Ausnahmefällen muss die Clusteranalyse manuell konfiguriert werden.

\section{Evaluierung der Spur-Geometrie-Bestimmung}

Nach der Clusteranalyse werden anhand der identifizierten Spur-Cluster die Geometrien der Fahrspuren
bestimmt. Die hierzu angewandten Verfahren sind in Abschnitt \ref{cha:lane_definition} beschrieben.

Wurden im vorherigen Schritt Trajektorie-Cluster für die in einer Aufnahme enthaltenen Fahrspuren identifiziert,
so können auf Basis dieser meist zuverlässige Spur-Geometrien abgeleitet werden.
In Abbildung \ref{fig:results_laneGeometries} sind beispielhaft Spur-Geometrien aus drei unterschiedlichen Datensätzen dargestellt.

\begin{figure}
    \centering
    \subfloat[Heilbronner-Straße]{{
        \includegraphics[align=c, width=0.45\linewidth]{resources/img/results/Heilbronner/improvedLanes_Heilbronner}
    }}
    \qquad
    \subfloat[Düsseldorf]{{
        \includegraphics[align=c, width=0.45\linewidth]{resources/img/results/Duesseldorf/improvedLanes_Duesseldorf}
    }}
    \hfill
    \subfloat[Steinheim]{{
        \includegraphics[align=c, width=0.45\linewidth]{resources/img/results/Steinheim/improvedLanes2}
    }}
    \caption[Ergebnisse der Spur-Geometrie-Bestimmung]
            {Ergebnisse der Spur-Geometrie-Bestimmung von verschiedenen Straßenabschnitten}
    \label{fig:results_laneGeometries}
\end{figure}

Anhand der abgebildeten Spur-Geometrien ist zu erkennen, dass das in Abschnitt \ref{cha:lane_definition}
beschriebene Verfahren gute Ergebnisse in unterschiedlichen Situationen liefert.
Plot a) zeigt die Spur-Geometrien, welche im Fall des Datensatzes der Heilbronner-Straße bestimmt werden.
Für jedes Spurcluster, welches in Abbildung \ref{fig:results_clusters_heilbronner} a) zu sehen ist, wurde eine Geometrie erstellt.
Es ist zudem zu sehen, dass die Fahrspuren partitioniert wurden, um Überlagerungen zu vermeiden.
Plot b) zeigt, dass Spur-Geometrien auch korrekt bestimmt werden, wenn in einer Aufnahme
Fahrbahnerweiterung existieren, wie dies im \textit{Düsseldorf}-Datensatz der Fall ist.
Anhand von Plot c), welcher die für den Datensatz \textit{Steinheim} ermittelten Spur-Geometrien zeigt, wird
außerdem deutlich, dass auch die Geometrien kreisförmiger Fahrspuren und Kreisverkehre bestimmt werden können.

Ein Problem, welches bei der Bestimmung der Spur-Geometrien auftreten kann, ist in Abbildung
\ref{fig:results_defekts_laneGeos} dargestellt.

\begin{figure}[H]
    \centering
    \includegraphics[width=0.5\linewidth]{resources/img/results/Defekte/Defekt_Part2}
    \caption{Partitionierungsproblem in den Spur-Geometrien}
    \label{fig:results_defekts_laneGeos}
\end{figure}

Das Beispiel zeigt einen Ausschnitt aus den Spur-Geometrien des Datensatzes \textit{Heilbronner-Straße}.
Auf dem entsprechenden Straßenabschnitt (vergleiche Abb. \ref{fig:results_lanes2} b)),
auf welchem sich die Spuren befinden, verbreitert sich am rechten Rand der Aufnahme die Fahrbahn nach oben hin.
Es entstehen zwei neue Fahrspuren. In den partitionierten Spur-Geometrien sollte daher eigentlich die
oberer Spur (rot) aus der mittlere Spur (schwarz) hervorgehen, welche wiederum selbst aus der unteren
Spur (blau) hervorgeht.
Wie in der Abbildung zu sehen ist, wird allerdings die mittlere Spur partitioniert
und die obere bleibt erhalten. Hier trifft der in Abschnitt \ref{sec:real2_lane_partitioning} beschrieben
Algorithmus eine falsche Partitionierungsentscheidung.

Partitionierungsprobleme können in ähnlicher Art gelegentlich auftreten. Der Grund hierfür
ist, dass es sehr schwierig ist, ein Verfahren zu definieren, welches in allen möglichen
Spurüberlagerungs-Konstellationen die richtige Entscheidung trifft.
Ein menschlicher Betrachter kann bei der Überlagerung zweier Spuren zwar meist intuitiv feststellen, welche
partitioniert werden muss, einem Algorithmus dies beizubringen, ist allerdings schwierig,
da die Entscheidung von vielen verschiedenen, situationsbezogenen Kriterien abhängt.

Der verwendete Partitionierungsansatz liefert, wie angestrebt, in der Mehrzahl der Fälle eine gute Lösung.
Es musste bei der Entwicklung darauf geachtet werden, das keine Verfahren erstellt wird, welches zu sehr
auf wenige Szenarien angepasst ist. 
Häufig gibt es bei der Teilung der Spuren auch kein eindeutiges ``Richtig'' oder ``Falsch''.
Welche Spur erhalten bleiben soll, ist dann Interpretationssache. Daher, und da die Fehler
leicht über die Oberfläche der \textit{Vehicle-Tracker} Anwendung korrigiert werden können, wurde der verwendete
Algorithmus hier nicht weiter angepasst.

\section{Ergebnisse der Spurerkennung}

Nachdem in den vorherigen Abschnitten auf die Fähigkeiten und Probleme der drei Hauptschritte des
Spurerkennungsverfahrens eingegangen wurde, werden nun Ergebnisse in Form von Screenshots vorgestellt.
In ihnen sind die automatisch erkannten und erstellten Spuren zu sehen, wie sie in der Anwendung
\textit{Vehicle-Tracker} dargestellt sind.

Die ersten zwei Screenshots, welche in Abbildung \ref{fig:results_lanes1} dargestellt sind, zeigen die ermittelten
Spuren für die Datensätze \textit{Esslingen} und \textit{Düsseldorf}.
In den Aufnahmen ist zu erkennen, dass die erstellten Fahrspuren gut mit den realen Verläufen der
Spuren auf der Fahrbahn übereinstimmen.
Dies gilt auch im Fall des Datensatzes \textit{Düsseldorf}, obwohl die hier detektierten Fahrzeugpositionen
aufgrund des niedrigen Aufnahmewinkels nicht immer in der Nähe der Spurmitte liegen
(siehe Abschnitt \ref{sec:grund_challenges_reconstruction}).

\begin{figure}[H]
    \centering
    \subfloat[Fahrspuren Esslingen]{{
        \includegraphics[align=c, width=0.5\linewidth]{resources/img/results/Lanes/Entennest_Lanes}
    }}
    \subfloat[Fahrspuren Düsseldorf]{{
        \includegraphics[align=c, width=0.5\linewidth]{resources/img/results/Lanes/Duesseldorf_Lanes}
    }}
    \caption{Erkannte Fahrspuren auf geraden Straßenabschnitten}
    \label{fig:results_lanes1}
\end{figure}

In Abbildung \ref{fig:results_lanes2} sind die erkannten Spuren der Datensätze \textit{Heilbronner-Straße} und \textit{Neckartor}
dargestellt. Auch hier stimmen die Spur-Geometrien gut mit den realen Spur-Ausmaßen überein. Zudem wird deutlich,
dass sowohl die Partitionierung der Spuren als auch das Angleichen benachbarter Spurenden die gewünschten Resultate liefern.
Im Fall der Heilbronner-Straße ist nun auch zu sehen, dass im unteren rechten Bereich keine Spur erkannt wird.
Der Grund hierfür wurde in Abschnitt \ref{sec:results_eval_clustering} beschrieben.

\begin{figure}[H]
    \centering
    \subfloat[Fahrspuren Neckartor]{{
        \includegraphics[align=c, width=0.5\linewidth]{resources/img/results/Lanes/Neckartor_Lanes}
    }}
    \subfloat[Fahrspuren Heilbronner-Straße]{{
        \includegraphics[align=c, width=0.5\linewidth]{resources/img/results/Lanes/Heilbronner_Lanes}
    }}
    \caption{Erkannte Fahrspuren auf Kreuzungen}
    \label{fig:results_lanes2}
\end{figure}

Der Screenshot in Abbildung \ref{fig:results_lanes3} zeigt die erkannten Fahrspuren im Datensatz \textit{Steinheim}.
Dieses Beispiel zeigt, dass Fahrspuren auch in Straßenabschnitten mit Kreisverkehren oder kreisförmigen Spuren
erkannt werden. Auch hier stimmen, trotz des niedrigen Aufnahmewinkels, die Fahrspuren weitestgehend
mit den realen Fahrbahnverläufen überein. Der frühe Abbruch der linken, nach oben verlaufenden Fahrspur
lässt sich auf die Unterbrechungen der Trajektorien in diesem Bereich zurückführen.

\begin{figure}[H]
    \centering
    \subfloat[Fahrspuren Steinheim]{{
        \includegraphics[align=c, width=0.6\linewidth]{resources/img/results/Lanes/Steinheim_Lanes}
    }}
    \caption{Erkannte Fahrspuren auf kreisförmigen Fahrbahnen}
    \label{fig:results_lanes3}
\end{figure}

Die oben abgebildeten und beschriebenen Ergebnisse zeigen, dass der entwickelte Spurerkennungsalgorithmus die in
Abschnitt \ref{sec:requirements} aufgestellten Anforderungen erfüllt.
Es können Fahrspuren in Straßen mit unterschiedlichen Topologien erkannt werden. Diese stimmen
mit den realen Spur-Geometrien meist gut überein. Auch bei niedrigen Aufnahmewinkeln können
Fahrspuren aus den Trajektorien abgeleitet werden, welche den Verlauf der Fahrbahnen ausreichend genau
beschreiben. Mit den meisten Defekten und Ausreißern in den Trajektorien kann umgegangen werden.
Schwierigkeiten hat das Verfahren, wenn die Mehrzahl der Trajektorien einer Fahrspur
Unterbrechungen aufweisen.