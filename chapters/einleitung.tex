%!TEX root = ../Thesis.tex

\chapter{Einleitung}
\label{cha:introduction}

Staus und zäh fließender Verkehr sind sowohl auf Schnell- und Autobahnen, als auch in Städten ein großes
Problem und Ärgerniss für Autofahrer. Sie kosten diese nicht nur wertvolle Zeit, sondern auch viel Geld.
Laut einer Studie von \cite[]{Cookson} kosten Staus jeden deutschen Autofahren pro Jahr durchschnittlich 1770 €.
In Summe ergeben sich hieraus beinahe 80 Milliarden Euro an Kosten.
Stau ist allerdings nicht nur finanziell für Privatpersonen oder auch Unternehmen ein großer Faktor,
sondern er erhöht auch das Unfallrisiko und trägt maßgeblich zur schlechten Luftqualität in Innenstädten bei.
Aufgrund längerer Fahrzeiten und der häufigen Be- und Entschleunigung, steigt der Kraftstoffverbrauch der
Fahrzeuge und dadurch auch die Schadstoffbelastung in der Luft \cite[]{Hemmerle2016}.

Die wichtigste Voraussetzung um Staus präventiv entgegenwirken zu können, ist, den Verkehr so gut wie
möglich zu verstehen.  Nötig ist ein Verständnis des Straßenverkehrs als Ganzes, sowie der Auswirkungen,
welche einzelne Verkehrsteilnehmer und deren Verhalten, auf diesen haben. Hierzu ist das Erstellen von
Simulationen sowie die Auswertung realer Verkehrsaufkommen unerlässlich.
Die auf diese Weise gesammelten Erkenntnisse bilden die Grundlage, um Straßenabschnitte, insbesondere
auch in Innenstädten, intelligent zu gestalten.
Des Weiteren können sie eingesetzt werden, um beispielsweise Ampelschaltungen in Städten zu optimieren,
wovon auch bestehende Infrastrukturen profitieren können.

Dank der Tatsache, dass unbemannte Luftfahrzeuge (\acrshort*{uav}) wie Drohnen immer leichter und günstiger
verfügbar sind, und die von ihnen erstellten Aufnahmen teils eine sehr gute Qualität besitzen, werden
diese immer häufiger zur Analyse des Straßenverkehrs eingesetzt. Über Methoden aus dem Umfeld des
maschinellen Sehens und maschinellen Lernens können aus Luftaufnahmen eine Vielzahl an interessanter
Informationen extrahiert werden.

Diese Arbeit beschäftigt sich mit der Realisierung einer automatischen Fahrspurerkennung in Luftaufnahmen.
Hierzu werden die Trajektoriedaten von Fahrzeugen ausgewertet.
Die Analyse von Verkehrssituation wird, nicht zuletzt auch aufgrund der zunehmenden Relevanz des autonomen
Fahrens, immer wichtiger.
Eine automatisierte Spurerkennung ist hierbei ein wichtiger Teil des Prozesses, da mit Hilfe der
erkannten Spuren unter anderem Spurwechsel- und Überholvorgänge sowie das Verhalten der Fahrzeuge
auf einer Spur untereinander untersucht werden können.

\section{Rahmen der Arbeit}
\label{sec:rahmen_arbeit}

\subsection{Das Projekt MEC-View}
\label{sec:mec_view}

\subsection{Das Teilprojekt MEC-View Luftbeobachtung}
\label{sec:mecview_sim}

\section{Motivation und Ziele}
\label{sec:motivation_goals}

% u.a.:
% Beschreibung der aus den Luftaufnahmen ermittelten (ermittelbaren) Werte
% Vorteile der Verwendung von Luftaufnahmen zur Erstellung von Verkehrssimulationen

\section{Aufbau dieser Arbeit}
\label{sec:aufbau}