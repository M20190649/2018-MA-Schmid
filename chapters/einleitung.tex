%!TEX root = ../Thesis.tex

\chapter{Einleitung}
\label{cha:introduction}

Staus und zäh fließender Verkehr sind sowohl auf Schnell- und Autobahnen, als auch in Städten ein großes
Problem und Ärgerniss für Autofahrer. Sie kosten diese nicht nur wertvolle Zeit, sondern auch viel Geld.
Laut einer Studie von \cite[]{Cookson} kostet Stau jeden deutschen Autofahren pro Jahr durchschnittlich 1770 €.
In Summe ergeben sich hieraus beinahe 80 Milliarden Euro an Kosten.
Stau ist allerdings nicht nur finanziell für Privatpersonen oder auch Unternehmen ein großer Faktor,
sondern er erhöht auch das Unfallrisiko und trägt maßgeblich zur schlechten Luftqualität in Innenstädten bei.
Aufgrund längerer Fahrzeiten und der häufigen Be- und Entschleunigung, steigt der Kraftstoffverbrauch der
Fahrzeuge und dadurch auch die Schadstoffbelastung in der Luft \cite[]{Hemmerle2016}.

Um Stau so gut wie möglich vermeiden zu können, muss man den Verkehr verstehen. Nötig
ist ein Verständnis des Straßenverkehrs als Ganzes, sowie der Auswirkungen, welche einzelne Verkehrsteilnehmer
und deren Verhalten, auf diesen haben. Hierzu ist das Erstellen von Simulationen sowie die Auswertung
realer Verkehrsaufkommen unerlässlich.
Die auf diese Weise gesammelten Erkenntnisse bilden die Grundlage, um Straßenabschnitte, insbesondere
auch in Innenstädten, intelligent zu gestalten.
Des Weiteren können sie eingesetzt werden, um beispielsweise Ampelschaltungen in Städten zu optimieren,
wovon auch bestehende Infrastrukturen profitieren können.

Diese Arbeit beschäftigt sich mit der Realisierung einer automatischen Fahrspurerkennung aus Luftaufnahmen,
welche bei der Analyse von Spurwechselvorgängen zum Einsatz kommt. Hierzu werden die Trajektoriedaten
von Fahrzeugen ausgewertet.


\section{Rahmen der Arbeit}
\label{sec:rahmen_arbeit}

\subsection{Das Projekt MEC-View}
\label{sec:mec_view}

\subsection{MEC-View Luftbeobachtung}
\label{sec:mecview_sim}

\section{Motivation und Ziele}
\label{sec:motivation_goals}

\section{Aufbau dieser Arbeit}
\label{sec:aufbau}