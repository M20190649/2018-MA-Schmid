%!TEX root = ../Thesis.tex

% Umsetzung

\chapter{Clustering von Fahrzeugtrajektorien}
\label{cha:realisation_clustering}

% Feature Selection (Positionen Relevant)
% Beispiel für Rohdaten (Kreuzung)

\section{Vorverarbeitung der Roh-Trajektorien}
\label{sec:realisation_preprocessing}

% Resampling
% Aussortierung zu kurzer Trajektorien
% Aussortierung stehender Trajektorien
% Trimmen Truck-Trajektories

\section{Gruppierung der Trajektorien}
\label{sec:realisation_clustering}

% genaue Beschreibung des Vorgehens, bis finale Clustering Lösung erreicht wurde
% Ansätze: Gründe, Stärken, tatsächliche Problem
% Ansatz A:
%   Mod. Hausdorff Distanz und Spectral Clustering (bas. auf Avet et al.) (Erklärung Grundfunktionsweise Spectral-Clustering)
%   Weil: SC performant, deterministisch, oft verwendet
%   Probleme: Clusteranzahl Bestimmung, Umgehen mit Ausreißern, Tatsächliche Ergebnisse nicht gut
% Ansatz B:
%   LCSS Distanz (in anderen Papern gute Ergebnisse) (impl. mittels bottom up dyna. programmierung, Verwendung Eucl. Dist.)
%   Wieso D2 aus Vlachos et al. verwendet? (Verschiebung unerwünscht)
%   DBSCAN Clustering
%   --> DM kann besser mit Ausreißern umgehen und DBSCAN berücksichtigt diese auch
%   bessere Ergebnisse


\chapter{Fahrbahn-Bestimmung aus Trajektorie-Clustern}
\label{cha:lane_definition}

% Cluster-Bereinigung: Entfernen von Outliern (Spurwechselvorgänge)
% Bestimmung Referenz-Trajektorie
% Bestimmung von Spur-Envelopes
% Partitionierung der initialen Spur-Schätzungen

\chapter{Fahrbahn Klassifizierung}
\label{cha:realisation_lane_classification}