%!TEX root = ../Thesis.tex

\chapter{Untersuchung möglicher Straßentopologien}
\label{cha:street_topologies}

% Beschreibung möglicher (Auswahl) Straßentopologien und ihrer Herausforderungen für die Erkennung von Fahrbahnen
% Idealerweise immer Bild einer entsprechenden Topologie und der entsprechenden Roh-Trajektorien

% Erläuterung: Was ist eine "Fahrspur" in dieser Arbeit. (d.h. nicht notwendigerweise eine Spur, wie sie auf Straße markiert ist. Beschreibt übliche Fahrbahn eines Fahrzeugs?!)

% Landstraßen (ein / zweispurig),
\section{Landstraßen}

% einfachste Straßentopologie, daher einfach darin Fahrspuren zu identifizieren
% entweder klar separierte Fahrspuren, welche parallel zueinander laufen oder eine (breite) Spur, welche in beide Richtungen genutzt wird
% üblicherweise wenig Abzweigungen, Einfahren etc.

% Bild einer leicht gekrümmten / gewundenen Landstraße


% Autobahnen (inkl. Auffahrten, Abfahrten)
\section{Autobahnen}

% ähneln Landstraße, Spuren hier immer separiert. Fahrzeuge fahren in auf einer Spur in eine Richtung
% Bahnen üblicherweise breiter.
% Auffahrten und Abfahren existieren

% Bild Entennest


% Kreuzungen (inkl. Abbiegespuren)
\section{Kreuzungen}

% Fahrbahnen kreuzen sich, geregelt über Ampelanlagen oder Rechts-vor-Links
% Abbiegespuren
% Fahrspuren überlagern sich. Sinnvolle Aufteilung

% Kreisverkehre
\section{Kreisverkehre}

% 12 Bewegungsbahnen durch Kreisverkehr (ausgenommen 360Grad Wendungen etc.)
% schwierig zu definieren, wie fahrspuren durch Kreisverkehr verlaufen
% Welche partitionieren
% Werden alle Bahnen richtig erkannt? Viele Überdeckungen. Richtig geclusterd?