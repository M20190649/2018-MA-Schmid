%!TEX root = ../Thesis.tex

\chapter{Zusammenfassung und Ausblick}
\label{cha:end}

Im Rahmen dieser Arbeit wurde ein Verfahren zur automatischen Erkennung von Fahrspuren in Videoaufnahmen
anhand von Trajektoriedaten entwickelt.
Nachfolgend werden die Ergebnisse der Thesis zusammengefasst und ein kurzer Ausblick gegeben.

\section{Ergebnisse dieser Arbeit}

Eine Untersuchung der Literatur zum Thema Fahrspurerkennung zu Beginn der Arbeit
(siehe Kapitel \ref{cha:related_work}) machte deutlich, dass die existierenden Ansätze 
nur für eingeschränkte Anwendungsfälle entwickelt wurden. 
Jene Lösungen, welche anhand von Fahrzeugtrajektorien die Fahrspuren ermitteln, können meist nur
in sehr speziellen Straßentopologien eingesetzt werden.
Des Weiteren stimmen die in den verwandten Arbeiten ermittelten Spur-Geometrien nur selten mit den
realen Dimensionen der Fahrspuren auf der Straße überein.
Aus diesen Gründen eignen sich die existierenden Verfahren nicht für den Einsatz in einer
Anwendung, mit deren Hilfe detaillierte Verkehrsanalysen aus Luftbeobachtungen erstellt werden.

In dieser Arbeit werden Fahrspuren anhand von Fahrzeugtrajektorien erkannt. Die verwendeten Trajektorien
wurden aus den zu analysierenden Luftaufnahmen rekonstruiert.
Die Spurerkennung basiert auf einer Clusteranalyse der Bewegungsbahnen
und einem neu entwickelten, mehrstufigen Verfahren zur Erstellung von Spur-Geometrien.
Mithilfe der Clusteranalyse werden Spur-Cluster identifiziert, welche Bewegungen von Fahrzeugen auf einer Fahrspur beschreiben.
Jene Cluster können auch in Trajektoriedatensätzen mit komplexen Fahrbahnverläufen identifiziert werden.
Damit der Clusteralgorithmus in unterschiedlichen Situationen jeweils gute Ergebnisse liefert, wurde ein Verfahren
entwickelt, um diesen automatisch zu parametrisieren.
Die Ergebnisse der Clusteranalyse dienen als Grundlage für die Ermittlung der Spur-Geometrien.
Zur Erstellung der Geometrien wurde ein robustes Verfahren für die Bestimmung der Spurmittellinien entwickelt.
Anhand der Mittellinien und deren relativer Lage zueinander, werden Hülllinien definiert,
welche die Dimensionen der Fahrspuren beschreiben. Diese werden partitioniert,
um Überlagerungen zu entfernen, bevor sie in den Verkehrsanalysen verwendet werden können.

In einer Evaluation konnte bestätigt werden, dass das entwickelte Spurerkennungsverfahren die in Abschnitt
\ref{sec:requirements} definierten Anforderungen erfüllt und die oben genannten Probleme der verwandten Arbeiten löst.
Mithilfe des Spurerkennungsverfahrens können Fahrspuren in Straßenabschnitten mit unterschiedlichsten
Fahrbahnverläufen identifiziert werden. Die ermittelten Spur-Geometrien stimmen mit den realen Fahrbahnverläufen
zudem meist gut überein und besitzen keine Überlagerungen. Dank der automatischen Parametrisierung des Verfahrens,
können die Fahrspuren zudem meist vollautomatisch ermittelt werden, was dem Anwender eine manuelle Konfiguration
unter Anwendung von Expertenwissen abnimmt.

Der entwickelte Algorithmus wurde in die Anwendung \textit{Vehicle-Tracker} integriert,
welche im Teilprojekt ``Luftbeobachtung'' des MEC-View Forschungsprojektes zur Analyse des Fahrverhaltens
von Verkehrsteilnehmern eingesetzt wird.

\section{Ausblick}

Das im Rahmen dieser Thesis entwickelte Spurerkennungsverfahren ermöglicht in Zukunft eine schnellere Auswertung
des Fahrverhaltens von Fahrzeugen in Luftaufnahmen. Die Spurerkennung ist ein wichtiger Schritt hin zu einer
vollständig automatisierten Auswertung von langen Verkehrsmesskampagnen.
Mithilfe der ermittelten Spurinformationen können
unter anderem Überhol- und Spurwechselvorgänge sowie das Verhalten der Fahrzeuge auf einer Spur
untereinander untersucht werden. Die Durchführung dieser Untersuchungen ist wichtig für die Erstellung
detaillierter Verkehrsanalysen. Erkenntnisse, welche aus den Analysen gewonnen werden, können, im
Rahmen des MEC-View Projektes, in Zukunft auch zur Optimierung des Fahrverhaltens von autonomen
Fahrzeugen eingesetzt werden.

Die Spur-Geometrien können zudem auch zur Evaluierung der Qualität der Fahrzeugerkennung und Verfolgung,
und insbesondere zur Identifikation von Ausreißern, eingesetzt werden.
Da sich Fahrzeuge, deren Fahrverhalten untersucht werden soll, auf den Fahrbahnen eines Straßenabschnittes befinden,
kann anhand der Spurinformationen ermittelt werden, welche Trajektorien außerhalb dieser liegen und
daher vermutlich Ausreißer sind.

Die Zuverlässigkeit der Fahrspurerkennung könnte in Zukunft durch die Identifizierung und Implementierung
eines alternativen Clustering-Verfahrens weiter gesteigert werden.
Der derzeit eingesetzte Ansatz, welcher komplette Trajektorien gruppiert, liefert zwar in der Mehrzahl der Fälle
gute Ergebnisse, führt jedoch auch dazu, dass in manchen Situationen Fahrspuren nicht korrekt erkannt werden.
Ein alternatives Verfahren könnte sich beispielsweise an der Arbeit von \cite[]{Xu2015} orientieren, in welcher
anhand von Trajektoriedaten eine \textit{``Heat-Map''} erstellt wird, aus welcher anschließend ``Mittellinien''
anhand eines \textit{``Adaptive Multi-Kernel-Based Shrinkage''}-Algorithmus extrahiert werden.
Hierbei fallen Unterbrechungen von Trajektorien nicht ins Gewicht. Ein Nachteil des von Xu et al. vorgestellten
Verfahrens ist, dass die ermittelten Spurmittellinien teilweise nur schlecht mit den realen Spurverläufen
übereinstimmen oder für eine Spur mehrere Zentrallinien bestimmt werden. 

Da die einzelnen Schritte der Spurerkennung unabhängig voneinander arbeiten, könnte ein
alternatives Clusterverfahren mit geringen Auswirkungen auf den Rest des Verfahrens integriert werden.