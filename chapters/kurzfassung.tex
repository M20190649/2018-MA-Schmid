%!TEX root = ../Thesis.tex

\chapter{Kurzfassung}

In dieser Arbeit wurde ein Verfahren zur automatischen Erkennung von Fahrspuren in Luftaufnahmen
auf Basis von Trajektoriedaten entwickelt. Umgesetzt wurde die Thesis im Rahmen des vom \acrshort*{bmwi}
geforderten Forschungsprojekt MEC-View, in welchem unter anderem das Fahrverhalten von Fahrzeugen
anhand von Luftbeobachtungen untersucht wird. Fahrspurinformationen sind für solche Analysen
wichtig, da nur mit ihnen beispielsweise Überhol- oder Spurwechselvorgänge untersucht werden können.

% Das im Rahmen dieser Arbeit entwickelte Verfahren kann Fahrspuren in unterschiedlichen Straßentopologien
Das im Rahmen dieser Arbeit entwickelte Verfahren kann Fahrspuren auf Straßen mit unterschiedlich komplexen Spur-Topologien erkennen. 
Die Verläufe und Breiten der algorithmisch definierten Spuren stimmen zudem gut mit den realen
Fahrspurverläufen überein. In diesen Punkten geht das entwickelte Verfahren über die bislang von den
verwandten Arbeiten vorgestellten Ansätze zur Fahrspuridentifikation hinaus.
Der Spurerkennungsalgorithmus nutzt eine Clusteranalyse und verschiedene, selbst entwickelte Verfahren zur
Ableitung von Fahrspur-Geometrien aus Trajektoriedaten.

In einer Auswertung und Evaluation wurden die Stärken und Schwächen des entwickelten Algorithmus ermittelt.
Es zeigt sich, dass dieser in den meisten untersuchten Luftaufnahmen zuverlässig Fahrspuren identifizieren kann.
Primäre Anforderung hierfür ist, dass jede Fahrspur eines Straßenabschnitts von ausreichend vielen ununterbrochenen
Trajektorien beschrieben wird.

Der entwickelte Algorithmus wurde in die Anwendung \textit{Vehicle-Tracker} integriert,
welche im MEC-View Teilprojekt ``Luftbeobachtung'' zur Analyse des Verkehrs eingesetzt wird.
Dank der automatischen Spurerkennung können in Zukunft Luftaufnahmen des Straßenverkehrs schneller
ausgewertet werden.


% Ziel der automatischen Spurerkennung war es, Fahrspuren in möglichst vielen verschiedenen Straßenabschnitten
% identifizieren zu können. Diese algorithmisch bestimmten Fahrspuren sollten außerdem mit den realen Abmaßen
% der Spuren auf der Straßen bestmöglich übereinstimmen.