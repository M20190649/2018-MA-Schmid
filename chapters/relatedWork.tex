%!TEX root = ../Thesis.tex

\chapter{Verwandte Arbeiten}
\label{cha:related_work}

Das folgende Kapitel gibt eine Überblick über wissenschaftliche
Arbeiten, welche sich bereits mit der Analyse von Trajektoriedaten und der Erkennung von Fahrspuren beschäftigen.
Zu Beginn werden Arbeiten vorgestellt, welche sich mit der Clusteranalyse von Trajektorien befassen.
Anschließend wird betrachtet, wie in der Literatur die Erkennung von Fahrspuren auf Basis
von Trajektorien umgesetzt wird.
Am Ende des Kapitels werden Defizite der existierenden Lösungen aufgezeigt und analysiert, welche
spezifischen Neuerungen für die Umsetzung dieser Arbeit nötig sind.

\section{Clusteranalyse von Trajektorien}
\label{sec:rw_clustering}

Aufgrund der vielen Erkenntnisse welche aus Trajektoriedaten gewonnen werden können, ist ihre
Auswertung schon seit geraumer Zeit Gegenstand wissenschaftlicher Untersuchungen.
Nachfolgende Arbeiten beschäftigen sich mit der Clusteranalyse von Trajektorien.
Die Auswahl zeigt prototypisch, wie unterschiedliche die Anwendungsszenarien und Ziele bei solchen Analyse sind.

% Fu et al., 2005
\subsubsection*{Similarity based vehicle trajectory clustering and anomaly detection}
Eine Arbeit, welche ein sehr typisches Anwendungszenario behandelt, stammt von \cite[]{Hu2005}. Die Autoren
beschreiben in dieser Veröffentlichung ein Verfahren zur Clusteranalyse von Fahrzeugtrajektorien. Ziel dieser
ist es, auf Basis der entdeckten Spur-Cluster, anormale Verkehrsmanöver in Live-Aufnahmen von Straßenabschnitten
detektieren zu können. Solche Manöver sind beispielsweise ``Fahren abseits der üblichen Bahnen'' oder
``zu schnelles/langsames Fahren''.
Die Fahrzeugtrajektorien sind in dieser Arbeit als Sequenzen zwei-dimensionaler Punkte definiert.
Um sie zu gruppieren, setzen Hu et al. auf klassische Clusterverfahren und die Verwendung eines
einfachen, metrischen Distanzmaßes. Dieses Maß, bekannt als HU-Distanz (siehe Abschnitt \ref{sec:hu_distance}),
vergleicht Trajektorien über den mittleren Abstand zwischen Punktpaaren.
Da dies nur zuverlässig möglich ist, wenn die Trajektorien einige Bedingungen erfüllen, müssen die
Autoren diese vorverarbeiten. Sie vereinheitlichen daher die Abstände der Punkte einer Trajektorie und erweitern
sie in Richtung der Szenen-Grenzen.

\begin{figure}[H]
    \centering
    \includegraphics[width=0.5\linewidth]{resources/img/RelatedWork/Fu_HierarchicalClustering}
    \caption[Zweistufiger Clustering-Vorgang von Hu et al.]
            {Zweistufiger Clustering-Vorgang von \cite[]{Hu2005}; Identifikation von dominanten Pfaden und Fahrspuren}
    \label{fig:relw_hu_two_step_cluster}
\end{figure}

Unter Verwendung des definierten Distanzmaßes werden die Trajektorien in einem zweistufigen Verfahren verarbeitet.
In den zwei Phasen werden, wie in Abbildung \ref{fig:relw_hu_two_step_cluster} dargestellt, zuerst dominante
Fahrpfade extrahiert, welche anschließend weiter in einzelne Fahrspuren untergliedert werden.
Als Clusteralgorithmen vergleichen die Autoren den \textit{Spectral-Clustering} Ansatz \cite[]{Ng2002}
mit einem \textit{Fuzzy-k-Means} Verfahren \cite[]{xie1991validity}.
Die Untersuchungen zeigen, dass der Spectral Clustering Ansatz nicht nur bessere Ergebnisse liefert, sonderen dieser
über mehrere Durchläufe hinweg auch stabil sind, wohingegen die Resultate des Fuzzy-Ansatzes variieren.


% Junejo et al., 2004
\subsubsection*{Multi Feature Path Modeling for Video Surveillance}
Eine weitere Arbeit welche das Ziel hat, anormale Bewegungsmuster auf Basis von Trajektorien zu entdecken,
stammt von \cite[]{Junejo2004}. In diesem Fall geht es den Autoren allerdings nicht um das Finden von Fahrzeug-Fahrspuren,
sondern um die Extraktion von Laufpfaden von Fußgängern.
Die Bewegungsbahnen der Passanten werden aus Aufnahmen stationärer Überwachungskameras gewonnen und
als zwei-dimensionale Punktreihen repräsentiert.
Zum Vergleich der Trajektorien, verwenden die Autoren die Hausdorff-Distanz als Distanzmaß.
Die üblicherweise negativen Eigenschaften dieses
Vergleichkriteriums (siehe Abschnitt \ref{sec:hausdorff_distance}), konkret die Missachtung der
Trajektorie-Orientierung, sind bei diesem Anwendungsfall kein Nachteil, sondern gewünscht.
Da Fußgänger auf einem Weg in entgegengesetzte Richtungen gehen können, muss die Orientierung
ihrer Trajektorien ignoriert werden.
Auf Basis der Hausdorff-Distanz erstellen Junejo et al. einen vollständigen Graphen, in welchem die Knoten Trajektorien
und die gewichteten Kanten den Distanzen zwischen Trajektorien entsprechen.
Sie zerlegen diesen Graphen mit Hilfe eines rekursiven \textit{min-cut}-Graphen-Algorithmus, welcher sich
an der Arbeit von \cite[]{boykov2004experimental} orientiert, und erhalten so die Cluster für die
extrahierten Fußgänger-Trajektorien.

% Atev et al., 2010
\subsubsection*{Clustering of Vehicle Trajectories}
\label{sec:atev_et_al}
In \cite[]{Atev2010} ist das Ziel der Autoren, ein Verfahren zu finden, mit welchem Fahrzeugtrajektorien
bestmöglich gruppiert werden können, ohne diese im Voraus anpassen zu müssen.
Sie vergleichen hierzu die Performance von drei unterschiedlichen Distanzmaßen unter Verwendung von zwei Clusteralgorithmen.
Primäres Augenmerk legen die Autoren auf ein von ihnen bereits in \cite[]{Atev2006} entwickeltes Distanzmaß,
welches auf der Hausdorff-Distanz basiert und sowohl die Orientierung von Trajektorien berücksichtigt
als auch robust gegenüber Ausreißern ist. Dieses neue Maß ist für zwei Trajektorien $P$ und $Q$
wie folgt definiert:

\begin{ceqn}
\begin{align}
\label{eq_modHausdorff}
    h_{\alpha, N, C}(P, Q) = \overset{\alpha}{\underset{p \in P}{ord}}\ \Big\{ \underset{q \in N_Q(C_{P,Q}(p))}{min} d(p, q) \Big\}
\end{align}
\end{ceqn}

Hierbei entspricht $C_{P,Q}$ einem Mapping $P \rightarrow Q$, welches einem Punkt $p \in P$ einen entsprechenden
Punkt $q \in Q$ zuweist, welcher die selbe relative Position in $Q$ besitzt wie $p$ in $P$.
$N_Q$ definiert ein Subset von $Q$ als Nachbarschaft des Punktes $q$. Zusammen definieren $N_Q$ und $C_{P,Q}$ eine
Struktur, in welcher der Vergleich der Trajektorien stattfindet. Dieses Vorgehen wird in Abbildung
\ref{fig:relw_atev_modh} visualisiert. Der Operator $ord_{p \in P}^{\alpha} f(p)$ selektiert jenen Wert aus $f(p)$, welcher
größer ist als $\alpha$-Prozent der Werte.

\begin{figure}[H]
    \centering
    \includegraphics[width=0.6\linewidth]{resources/img/RelatedWork/Atev_modHausdorff}
    \caption[Funktionsweise der modifizierten Hausdorff-Distanz]{Funktionsweise der modifizierten Hausdorff-Distanz \cite[]{Atev2010}}
    \label{fig:relw_atev_modh}
\end{figure}

Dank dieser Modifizierungen eignet sich das Distanzmaß für den Vergleich von Trajektorien:
$C_{P,Q}$ sorgt für den Einbezug der Orientierung und über die Nachbarschaft $N_Q$ und $ord_{p \in P}^{\alpha} f(p)$
kann der Einfluss von Ausreißern minimiert werden.

Das Distanzmaß vergleichen Atev et al. unter Verwendung eines Spectral und eines Agglomerativen
Clusteralgorithmus mit der \textit{Longest-Common-Subsequence} (LCSS) und der \textit{Dynamic-Time-Warping}-Distanz (\acrshort*{dtw}).
Die Ergebnisse der Untersuchungen für vier verschiedene Datensätze zeigen, dass die beste Cluster-Performance
mit Hilfe der modifizierten Hausdorff-Distanz und des Spectral-Clustering erreicht wird.

Dass das von Atev et al. vorgeschlagene Distanzmaß gute Clusterergebnisse produziert, wurde auch von \cite[]{Morris2009}
bestätigt. In ihrer Untersuchung waren alledings die Ergebnisse, welche mithilfe des LCSS Maßes erreicht wurden,
ebenso gut oder teilweise besser.

% Chen et al., 2011
\subsubsection*{Clustering of trajectories based on Hausdorff Distance}
Eine weitere interessante Arbeit zur Clusteranalyse von Trajektorien stammt von \cite[]{Chen2011}.
Die Autoren haben das Ziel, Muster in den Bewegungsbahnen von Hurrikans, welche im Zeitraum von 1850 bis 2010
über den Atlantik zogen, zu erkennen.
Sie verwenden hierzu einen angepassten DBSCAN Clusteralgorithmus und das Hausdorff-Distanzmaß.
Um die Missachtung der Orientierung kompensieren zu können, und zudem auch Ähnlichkeiten
in Sub-Trajektorien zu erkennen, wählen die Autoren eine etwas andere Darstellung der Trajektorien.
Sie definieren eine Bewegungsbahn als eine Folge sogenannter \textit{``Flow-Vektoren''}, welche neben
Positions- auch Richtungsinformationen enthalten. Ein solcher Vektor ist definiert über:

\begin{ceqn}
\begin{align}
    f_i = (x_i, y_i, dx_i, dy_i)
\end{align}
\end{ceqn}

wobei gilt:

\begin{ceqn}
\begin{align}
    dx_i = (x_{i+1} - x_i)/\sqrt{(x_{i+1} - x_i)^2 + (y_{i+1} - y_i)^2} \\
    dy_i = (y_{i+1} - y_i)/\sqrt{(x_{i+1} - x_i)^2 + (y_{i+1} - y_i)^2}
\end{align}
\end{ceqn}

Die Distanz zwischen zwei \textit{Flow-Vektoren} ist ihr euklidscher Abstand. Auf diese Weise wird bei
der Berechnung der Hausdorff-Distanz (siehe Abschnitt \ref{sec:hausdorff_distance}) auch die Richtung
der Trajektorien berücksichtigt.
Um ähnliche Sub-Trajektorien entdecken zu können, teilen Chen et al. die Trajektorien an den Positionen
``charakteristischer'' Vektoren. Diese beschreiben Richtungsänderungen in einer
Bewegungsbahn und werden identifiziert über die Abweichungen in den Richtungskomponenten zweier
aufeinanderfolgender Flow-Vektoren.
Veranschaulicht ist dies in Abbildung \ref{fig:relw_chen_flow_vector}.

\begin{figure}[H]
    \centering
    \includegraphics[width=0.45\linewidth]{resources/img/RelatedWork/Chen_trajectory_splitting}
    \caption[Zerlegung einer Trajektorie in Sub-Trajektorien (Chen et al.)]
            {Zerlegung einer Trajektorie in Sub-Trajektorien anhand von ``charakteristischen Flow-Vektoren'' \cite[]{Chen2011}}
    \label{fig:relw_chen_flow_vector}
\end{figure}

Die auf diese Weise ermittelten Sub-Trajektorien, werden von den Autoren mittels eines DBSCAN Algorithmus gruppiert.
Sie können so die üblichen Bewegungsbahnen von Hurrikans über dem Atlantik bestimmen.


% Vlachos et al., 2002
\subsubsection*{Discovering Similar Multidimensional Trajectories}
\label{sec:relw_vlachos}
Die Arbeit \cite[]{Vlachos2002} thematisiert nicht direkt die Clusteranalyse von Trajektorien sondern
beschäftigt sich mit dem Vergleich von Bewegungsbahnen im drei-dimensionalen Raum. Konkret ist ihr Ziel,
Trajektorien vergleichen zu können, welche etwa die Handbewegungen beim Ausführen von Zeichensprache beschreiben.
Hierzu definieren die Autoren erstmals die Grundversion des LCSS Distanzmaßes, welches in vielen Arbeiten,
unter anderem in \cite[]{Atev2006}, \cite[]{Buzan2004} und \cite[]{Chen2005} zum Einsatz kommt.
Auf dessen Basis erstellen sie ein Distanzmaß, welche es ermöglicht formgleiche aber im Raum verschobene Trajektorien zu finden.
Die Grundversion der LCSS-Distanz und ein darauf basierendes, einfaches Distanzmaß ist, nach Vlachos et al.,
bereits in Abschnitt \ref{sec:lcss_distance} vorgestellt worden.
Dieses Maß erweitern die Autoren zudem wie folgt:

\begin{ceqn}
\begin{align}
    D2_{LCSS}(\delta, \epsilon, A, B) = 1 - \underset{f_{c,d} \in F}{max}\ D_{LCSS}(\delta, \epsilon, A, f_{c,d}(B))
\end{align}
\end{ceqn}

Hierbei ist $F$ eine Menge von Translations-Funktionen, welche die Trajektorien entlang der Achsen verschieben.
Sie besitzen die Form

\begin{ceqn}
\begin{align}
    f_{c, d}(A) = ((a_{x, 1} + c, a_{y, 1} + d), ..., (a_{x, n} + c, a_{y, n} + d))
\end{align}
\end{ceqn}

Abbildung \ref{fig:relw_vlachos_translation} veranschaulicht die Funktionsweise des Distanzmaßes. Es eignet sich immer dann, wenn Trajektorien
mit ähnlicher Form gefunden werden sollen, welche zudem eine gewisse räumliche Verschiebung aufweisen können.
Diese kann über die Größe von $F$ gesteuert werden.

\begin{figure}[H]
    \centering
    \includegraphics[width=0.55\linewidth]{resources/img/RelatedWork/vlachos_translation}
    \caption[Verschiebung einer Trajektorie im Raum]{Verschiebung einer Trajektorie im Raum \cite[]{Vlachos2002}}
    \label{fig:relw_vlachos_translation}
\end{figure}

% Ren et al., 2014
\subsubsection*{Lane Detection in Video-Based Intelligent Transportation Monitoring via Fast Extracting and Clustering of Vehicle Motion Trajectories}

\cite[]{Ren2014} stellen in ihrer Arbeit ein Vorgehen zur Clusteranalyse von Trajektorien vor,
welches auf der \textit{Rough-Set}-Theorie beruht. Sie extrahieren Fahrzeugpositionen aus Aufnahmen stationärer
Überwachungskameras und stellen diese, wie die meisten Autoren, als Sequenzen zwei-dimensionaler
Punkte dar. Ihr Ziel ist anschließend, anhand einer Gruppierung der Fahrzeugtrajektorien,
die Spurmittelpunkte der Fahrbahnen zu bestimmen. Da ein nicht unerheblicher Anteil der Trajektorien Spurwechselvorgänge
enthält, welche eine Extraktion der Mittellinien erschweren, verwenden Ren et al. einen iterativen \textit{Rough-k-Means}
Algorithmus zur Gruppierung der Trajektorien. Hierbei wird jedes Cluster über eine obere und untere Approximation beschrieben.
Die untere Approximation enthält dabei die Trajektorien, welche eindeutig der Spur zugeordnet werden können.
Die obere Näherung hingegen auch jene, welche Spurwechsel et cetera beschreiben. Bei der Berechnung der Spurmitten, werden
die Trajektorien der unteren Approximation höher gewichtet, als die der oberen. Ein sehr ähnlicher Cluster-Ansatz wurde
bereits in \cite[]{Lingras2004} vorgestellt.
Die initialen Mittellinien bestimmen die Autoren anhand einer \textit{Aktivitäts}- oder \textit{Heat}-Map,
welche sie während der Extraktion der Fahrzeugpositionen erstellen. Die Clusteranzahl $k$ muss händisch definiert werden.

Als Maß für die Distanz zwischen einer Trajektorie $A_x$ und einer Spurmitte $c_i$ verwenden Ren et al. die
Hausdorff-Distanz $h(A_x, c_i)$. Für eine Trajektorie wird somit die nächste Mittellinie wie folgt gefunden:

\begin{ceqn}
\begin{align}
    h(A_x, c_m) = \underset{i = 1 ... k}{min}\ h(A_x, c_i)
\end{align}
\end{ceqn}

Hieraus ergibt sich die nachfolgende Definition für die Zuordnung der Bewegungsbahnen zu den Cluster-Näherungen:

\begin{ceqn}
\begin{align}
    \begin{cases}
        A_x \in \overline{C_m} \land A_x \in \overline{C_j} & \text{if } j \neq m \land \frac{h(A_x, c_j)}{h(A_x, c_m)} \leq \lambda \\
        A_x \in \underline{C_m} & \text{otherwise}
    \end{cases}
\end{align}
\end{ceqn}

Für den Grenzwert $\lambda$ gilt $1 \leq \lambda \leq 1.5$. $\overline{C_m}$ und $\underline{C_m}$ entsprechen der oberen und unteren
Näherung des $m$-ten Clusters und es gilt $\underline{C_m} \subseteq \overline{C_m}$.
Nachdem in jeder Iteration des Cluster-Vorgangs die Näherungen auf diese Weise bestimmt wurden, werden die neue Mittellinien
anhand Gleichung \ref{eq_ren_rough} errechnet.

\begin{ceqn}
\begin{align}
    \label{eq_ren_rough}
    c_i =
    \begin{cases}
        \frac{w_l \sum_{A_x \in \underline{C_i}} A_x}{|\underline{C_i}|} + \frac{(1 - w_l) \sum_{A_x \in (\overline{C_i} - \underline{C_i})} A_x}{|\overline{C_i} - \underline{C_i}|} & \text{if } \overline{C_i} \neq \underline{C_i} \\
        \frac{\sum_{A_x \in \underline{C_i}} A_x}{|\underline{C_i|}} & \text{otherwise}
    \end{cases}
\end{align}
\end{ceqn}

Als Gewichtungen $w_l$ verwenden Ren et al. Werte im Bereich $[0.5, 1]$. $| \cdot |$ entspricht hier der Kardinalität einer Menge.

Unter Verwendung dieser Clustering-Methode ist es den Autoren von \cite[]{Ren2014} möglich, auch bei einer hohen Anzahl von
Ausreißern und Spurwechselvorgängen, stabile Spurmittellinien zu bestimmen. Ergebnisse, welche dies zeigen, sind in Abbildung
\ref{fig:relw_ren_example_detection} dargestellt.

\begin{figure}[H]
    \centering
    \includegraphics[width=0.95\linewidth]{resources/img/RelatedWork/ren_examples_detection}
    \caption[Ergebnisse Spurmittellinien-Erkennung (Ren et al.)]
            {Ergebnisse Spurmittellinien-Erkennung auf unterschiedlichen Straßenabschnitten \cite[]{Ren2014}}
    \label{fig:relw_ren_example_detection}
\end{figure}


\section{Erkennung und Definition von Fahrspuren}
\label{sec:rw_lane_detection}

Ziel dieser Arbeit ist es, Fahrspuren zuverlässig in Videoaufnahmen erkennen zu können. Dieser Abschnitt
geht daher auf Veröffentlichungen mit ähnlichen Zielen ein.

Die meisten Veröffentlichungen in diesem Bereich lösen das Problem, indem sie nach visuellen Merkmalen,
primär Spurtrennlinien, in Videoaufnahmen suchen. Die Aufnahmen werden dabei entweder von stationären Kameras,
von bemannten oder unbemannten Luftfahrzeugen oder einem Fahrzeug selbst erstellt. Zur Extraktion der Merkmale werden
üblicherweise Methoden aus den Gebieten des maschinellen Sehen (\acrshort*{cv}) oder maschinellen Lernens (\acrshort*{ml}) verwendet.
Arbeiten, welche CV-basierte Ansätze verfolgen, stammen beispielsweise von \cite[]{Lai2000}, \cite[]{McCall2006} oder \cite[]{Aly2008}.
ML-gestützte Arbeiten wurden dahingegend unter anderem von \cite[]{Kim2008}, \cite[]{Gopalan2012} oder
\cite[]{Neven2018} veröffentlicht.
Ein häufiges Einsatzgebiet für die visuelle Spurerkennung über eine im Fahrzeug verbaute Kamera, sind Spurhalte- oder
Spurwechsel-Assistenten.
% Ein mögliches Anwendungsgebiet für solche Ansätze ist Beispiel die Entwicklung eines Spurhalte-Assistenten.

Da visuelle Ansätze, wie bereits zu Beginn der Arbeit erläutert, aufgrund von Verdeckungen oder Änderungen
in der Belichtung problematisch sind, werden nachfolgend ausschließlich Arbeiten vorgestellt, welche Fahrspuren
aus Trajektoriedaten extrahieren. Hierbei setzten die meisten als ersten Schritt auf eine Clusteranalyse von Trajektorien.
In der Repräsentation und Extraktion der Fahrspuren variieren die Ansätze hingegen.

% Fu et al. & Junejo et al.
%   Clustering und dann Bestimmung Envelopes basierend auf Varianz der Trajektorien in Cluster
\subsubsection*{A System for Learning Statistical Motion Patterns}
Die Arbeit \cite[]{WeimingHu2006} basiert auf dem bereits früher veröffentlichten Artikel \cite[]{Hu2005} der selben Autoren,
welcher oben beschrieben wurde. In dieser Publikation gehen Hu et al. genauer darauf ein, wie sie auf Basis
der Ergebnisse der Clusteranalyse, statistische Informationen über die Fahrbahnen berechnen.
Hierzu wird für jedes Cluster zuerst eine Referenz-Trajektorie $T_r$ bestimmt. Dies ist jene Bewegungsbahn,
bei welcher die Summe der Distanzen zu allen anderen Trajektorien des Clusters minimal ist. Das verwendete
Distanzmaß ist hierbei des selbe, welches auch bei der Clusteranalyse zum Einsatz kam.
Die Autoren berechnen anschließend für jede Trajektorie-Gruppe eine Kette Gausscher-Wahrscheinlichkeits-Verteilungen
$\{ \varphi_1, \varphi_2, ..., \varphi_l \}$, wobei $l$ die Anzahl der Punkte der Referenz-Trajektorie ist.
Für jedes $\varphi_i$ wird der Mittelwert und die Kovarianz, basierend auf den Trajektorie-Punkten, welche dem
$i$-ten Punkt von $T_r$ am nächsten liegen, berechnet.
Anhand der Kovarianz-Werte erstellen die Autoren Hüllen für die Fahrbahnen. Ergebnisse dieses Vorgehens sind in
Abbildung \ref{fig:relw_hu_example_envelope} dargestellt.

\begin{figure}[H]
    \centering
    \subfloat{{
        \includegraphics[align=c, width=0.49\linewidth]{resources/img/RelatedWork/hu_example_envelope}
    }}
    \subfloat{{
        \includegraphics[align=c, width=0.49\linewidth]{resources/img/RelatedWork/hu_example_envelope2}
    }}
    \caption[Spurhüllen in Hu et al.]{Trajektorien und ermittelte Spurhüllen in \cite[]{WeimingHu2006}}
    \label{fig:relw_hu_example_envelope}
\end{figure}

Anzumerken ist, dass die extrahierten Spurhüllen nicht die tatsächliche Form, insbesondere die Breite, einer Fahrspur
wiederspiegeln. Sie sind deutlich schmäler.


% Makris et al., 2005
\subsubsection*{Learning Semantic Scene Models From Observing Activity in Visual Surveillance}
In \cite[]{Makris2005} extrahieren die Autoren übliche Bewegungsbahnen von Fußgängern aus stationären Videoaufnahmen.
Sie verwenden hierzu keine klassische Clusteranalyse, sondern ein iteratives und adaptives Online Verfahren,
welches neue Bewegungstrajektorien automatisch bestehenden Routen zuordnet oder neue initialisiert,
falls zu hohe Abweichungen zwischen der Trajektorie und den existierenden Bahnen bestehen.
Routen definieren Makris et al. hierbei als eine Sequenz von Knoten, welche folgende Merkmale besitzen:

\begin{itemize}
    \item 2D-Mittelpunkt
    \item Gewichtung (Anzahl der Trajektorien im Bereich des Knoten)
    \item Zwei Hüllpunkte (Maximale- beziehungsweise Standardabweichung der Trajektorien der Route im Bereich des Knoten)
\end{itemize}

Abbildung \ref{fig:relw_hu_example_envelope} a) veranschaulicht die Definition einer Route.
Um aus einfachen Bewegungsbahnen Routen zu erstellen, verwenden die Autoren das nachfolgend beschriebene Verfahren,
welches dem agglomerativen Cluster-Ansatz ähnelt.

\begin{enumerate}
    \item Die erste Trajektorie initialisiert die erste Route.
    \item Neue Trajektorien werden mit bestehenden Bahnen abgeglichen und
    \begin{enumerate}
        \item bei Übereinstimmung wird Route aktualisiert, oder
        \item bei keiner Übereinstimmung wird neue Route erzeugt.
    \end{enumerate}
    \item Aktualisierte Routen werden auf definierten Knotenabstand $r$ resampled.
    \item Alle Routen werden miteinander verglichen und
    \begin{enumerate}
        \item bei einer Überlagerung der Routen werden diese fusioniert.
    \end{enumerate}
\end{enumerate}

Als Vergleichsmaß für die Trajektorien und Routen, verwenden Makris et al. die maximale Distanz zwischen einer Trajektorie und
einer Routen-Hülle. Diese Distanz muss sich unterhalb eines bestimmten Grenzwertes befinden, damit eine Trajektorie
einer Route zugeordnet wird. Geschieht dies, dann werden alle Hüll- und Mittelpunkte neue berechnet.
Ein Ergebnis des Verfahrens ist in Abbildung \ref{fig:relw_results_makris} b) dargestellt.

\begin{figure}[H]
    \centering
    \subfloat[]{{
        \includegraphics[align=c, width=0.4\linewidth]{resources/img/RelatedWork/makris_route}
    }}
    \qquad
    \subfloat[]{{
        \includegraphics[align=c, width=0.4\linewidth]{resources/img/RelatedWork/makris_result}
    }}
    \caption[Routen-Definition und Ergebnisse Routen-Erkennung (Makris et al.)]{a) Routen Definition, b) Ergebnisse Routen-Erkennung \cite[]{Makris2005}}
    \label{fig:relw_results_makris}
\end{figure}


% Morris et al., 2011
\subsubsection*{Trajectory Learning for Activity Understanding: Unsupervised, Multilevel, and Long-Term Adaptive Approach}

In \cite[]{Morris2011} stellen die Autoren ein dreistufiges Framework zur Auswertung von Fahrzeugtrajektorien vor.
Ziel des Frameworks ist es, die Bewegungsmuster von Objekten mittels eines erlernten Vokabulars beschreiben zu können
sowie Aktivitäten vorhersagen und Anomalien erkennen zu können.
In einem ersten Schritt werden daher sogenannte \textit{Points of Interests} (POI) identifiziert. Die von Morris
et al. untersuchten POI's sind die Eintritts-, Austritts- und Stop-Zonen innerhalb einer Szene.
Nach deren Bestimmung identifiziert das vorgestellte Framework übliche Bewegungsmuster in den Trajektorien
und definiert auf deren Basis anschließend Pfade.

Sich häufig wiederholende Bewegungsmuster werden von den Autoren über eine Clusteranalyse ermittelt.
Zum Vergleich der Trajektorien kommt das in Abschnitt \ref{sec:lcss_distance} vorgestellte LCSS Distanzmaß zum Einsatz
und als Clusteralgorithmus das Spectral-Clustering. Morris et al. verwenden statt des im Spectral-Clustering
üblicherweise eingesetzten k-Mean-Algorithmus, einen Fuzzy-C-Mean Ansatz, welcher für jede Trajektorie
einen Zugehörigkeitswert $u_{ik} \in [0, 1]$ zum Cluster $k$ bestimmt.
Anhand der Zugehörigkeiten der Bewegungsbahnen zu den Clustern werden Referenz-Trajektorien bestimmt.
Diese ergeben sich für jedes Cluster als gewichteter Durchschnitt aller Trajektorien. Als Gewichte
werden die Werte $u_{ik}$ verwendet. Die so erstellten Referenz-Linien sind in Abbildung
\ref{fig:relw_morris_results} a) zu sehen.

\begin{figure}[H]
    \centering
    \subfloat[]{{
        \includegraphics[align=c, width=0.35\linewidth]{resources/img/RelatedWork/morris_routes_paths_1}
    }}
    \qquad
    \subfloat[]{{
        \includegraphics[align=c, width=0.35\linewidth]{resources/img/RelatedWork/morris_routes_paths_2}
    }}
    \caption[Referenz-Trajektorien und Pfade (Morris et al.)]{a) Refernz-Trajektorien, b) Pfade basierend auf HMMs \cite[]{Morris2011}}
    \label{fig:relw_morris_results}
\end{figure}

Basierend auf den Clustern definieren Morris et al. außerdem Pfade, welche die räumliche und zeitliche Dynamik
der Fahrzeuge an einer bestimmten Stelle abbilden. Sie verwenden hierzu Hidden-Markov-Modelle (\acrshort*{hmm}),
welche sie mittels der Baum-Welch-Methode trainieren.
Die Ergebnisse dieses Ansatzes sind in Abbildung \ref{fig:relw_morris_results} b) dargestellt. Die
Ellipsen repräsentieren die HMM's, welche angeben, wie die erwartete Bewegung eines Fahrzeugs in einem
bestimmten Bereich aussieht.
Die in \cite[]{Morris2011} bestimmten Pfade ermöglichen es diverse Aussagen über das Verhalten der Fahrzeuge
zu treffen. Sie repräsentieren allerdings nicht die realen Spur-Geometrien eines Straßenabschnittes.

% Hsieh et al., 2006
\subsubsection*{Automatic Traffic Surveillance System for Vehicle Tracking and Classification}
% Heat Map --> Spurmittellinien --> Begrenzer aus Mittellinien bestimmen

\cite[]{Hsieh2006} stellen in ihrer Arbeit ein System zur Extraktion von Fahrzeugpositionen aus Aufnahmen
stationärer Verkehrskameras vor. Auf diesen aufbauend definieren sie einen Algorithmus, mit dessen Hilfe es
möglich ist, Spurmittel- und Begrenzungs-Linien zu entdecken.
Hierzu erzeugen Hsieh et al. ein Histogramm $H_{vehicle}(x,y)$, welches die Häufigkeit abbildet, das sich Fahrzeuge über eine
bestimmte Pixel-Position $(x, y)$ bewegen. Ein solches ist in Abbildung \ref{fig:relw_hsieh_results} a) dargestellt.
Die Autoren gehen davon aus, dass die Fahrzeugpositionen sich mehrheitlich in der Mitte einer Spur befinden und
ein Histogramm, welches für eine Fahrbahn mit $L_n$ Spuren erstellt wurde, $L_n$ Maxima in jeder Zeile besitzt,
welche die Spurmitten darstellen.
Sie isolieren daher die Maxima in $H_{vehicle}$ als Mittellinien. Es sei anschließend $C_{L_k}^{j}$ die $k$-te Spurmittellinie
in Zeile $j$ des Histogramms. Daraus berechnen Hsieh et al. die Spurbegrenzungslinien. Alle innen liegenden
Begrenzungen ergeben sich aus Gleichung \ref{eq_hsieh1}. $DL_{k}^{j}$ entspricht hierbei einem Punkt
in der $j$-ten Reihe der $k$-ten Begrenzung.

\begin{ceqn}
\begin{align}
\label{eq_hsieh1}
    DL_{k}^{j} = \frac{1}{2} (C_{L_{k-1}}^{j} + C_{L_{k}}^{j})
\end{align}
\end{ceqn}

Die Breite $w_{L_k}^{j}$ der $k$-ten Fahrspur ergibt sich zudem wie folgt:

\begin{ceqn}
\begin{align}
\label{eq_hsieh2}
    w_{L_k}^{j} = | C_{L_{k}}^{j} - C_{L_{k-1}}^{j} |
\end{align}
\end{ceqn}

Die Positionen der äußeren Spurbegrenzungen einer Fahrbahn ergeben sich für die $j$-te Zeile des Histogramms
aus den Gleichungen \ref{eq_hsieh3} und \ref{eq_hsieh4}.

\begin{ceqn}
\begin{align}
\label{eq_hsieh3}
    (x_{DL_0^j}, j) &= (x_{DL_1^j - w_{L_0}^j}, j) \\
\label{eq_hsieh4}
    (x_{DL_{N_L}^j}, j) &= (x_{DL_{N_{L-1}}^j + w_{L_0}^j}, j)
\end{align}
\end{ceqn}

Ein Beispiel für die Ergebnisse der Spurerkennung von Hsieh et al. ist in Abbildung \ref{fig:relw_hsieh_results} b) dargestellt.
Das Verfahren funktioniert gut, wenn Fahrspuren nebeneinader und parallel zueinander liegen. Die Dimensionen
sich kreuzender Fahrspuren können mit dem Verfahren beispielsweise allerdings nicht bestimmt werden.

\begin{figure}[H]
    \centering
    \subfloat[]{{
        \includegraphics[align=c, width=0.33\linewidth]{resources/img/RelatedWork/hsieh_histogram}
    }}
    \subfloat[]{{
        \includegraphics[align=c, width=0.68\linewidth]{resources/img/RelatedWork/hsieh_centers_and_borders}
    }}
    \caption[Ergebnisse Histogramm Erstellung und Spurextraktion (Hsieh et al.)]{a) Fahrzeugpositions Histogramm, b) Spurmittellinien und Spurbegrenzungslinien \cite[]{Hsieh2006}}
    \label{fig:relw_hsieh_results}
\end{figure}


\section{Defizite vorhandener Lösungen und benötigte Neuerungen}
\label{sec:rw_deficites}

Es existiert eine große Anzahl von Arbeiten, welche sich mit der Clusteranalyse von Trajektoriedaten
befasst. Die in Abschnitt \ref{sec:rw_clustering} vorgestellten Veröffentlichungen stellen nur eine
kleine Auswahl dar.
Die Arbeiten entwickeln Lösungen in verschiedensten Anwendungsgebieten wie der Verkehrs-, Wetter und Verhaltensanalyse
und suchen daher in den Trajektoriedaten nach unterschiedlichsten Mustern. Die geforderte Genauigkeit der
Clustering-Ergebnisse unterscheidet sich je nach Anwendungsfall ebenfalls stark.
Im Rahmen dieser Masterarbeit muss ein Verfahren identifiziert und entwickelt werden, welches Fahrspur-Cluster
in Trajektoriedaten zuverlässig identifizieren kann. Die Fahrspuren können hierbei unterschiedlichste
Geometrien aufweisen.

Ein Defizit vieler vorhandener Arbeiten ist, dass diese meist mit wenigen unterschiedlichen
Trajektoriedatensätzen arbeiten und die verwendeten Daten selten Defekte wie Unterbrechungen,
Ausreißer oder inkorrekte Positionsinformationen aufweisen, welche allerdings durch Fehler in der Fahrzeugverfolgung entstehen können.
In einigen Arbeit wird teilweise nur mit generierten Trajektoriedaten gearbeitet.
Da in der vorliegenden Thesis davon ausgegangen werden muss, dass all diese Probleme auftreten können,
muss ein zuverlässiges Verfahren zur Bereinigung von Trajektoriedaten entwickelt werden. Ohne einen solchen Vorverarbeitungschritt,
wäre ein akkurates Clustering nicht möglich.

Deutlich weniger Veröffentlichungen gibt es zum Thema der Identifikation von Fahrspuren. Zwar existieren
einige Arbeiten, welche Fahrspuren auf Basis von Trajektorien ermitteln, diese entsprechen aber
in den allermeisten Fällen nicht den realen Verläufen der Fahrbahnen. Häufig werden Fahrspuren
anhand statistischer Verfahren, wie in \cite[]{WeimingHu2006} oder in \cite[]{Teng2015} beschrieben, ermittelt.
\cite[]{Hsieh2006} ermitteln in ihrer Arbeit Spur-Begrenzungslinien, allerdings nur für kurze, parallele
Fahrspuren eines Autobahnabschnittes. Auch die von \cite[]{Liu2010}, \cite[]{Sochor2014} und \cite[]{Chen2013}
veröffentlichten Verfahren eignen sich nur zur Identifikation von parallelen Fahrspur-Geometrien,
welche von statischen Kameras aufgenommen werden.

Ziel dieser Arbeit ist es, Fahrspuren in unterschiedlichen Straßentopologien identifizieren zu können.
Die erkannten Spuren sollen außerdem den realen Abmaßen der Spuren auf der Straße bestmöglich entsprechen.
Da dies von keiner existierenden Arbeit geboten wird, muss das Verfahren hierzu selbst entwickelt werden.
Es muss zudem eine Lösung zur Partitionierung von Fahrspuren entwickelt werden, da sich Spuren
in dieser Arbeit nicht über längere Bereiche hinweg überlagern dürfen.

% Es muss zudem für jede
% automatisch erkannte Spur überprüft werden, ob es sich bei ihr tatsächlich um eine real Fahrspur handeln kann,
% oder ob ihr Verlauf im Verhältnis zu dem der anderen Bahnen nicht plausibel ist.

% In dieser Arbeit sollen reale Spur-Geometrien für unterschiedlichste Straßenabschnitte bestimmt werden,
% weshalb das hierzu eingesetzte Verfahren selbst entwickelt werden muss.