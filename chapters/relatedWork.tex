%!TEX root = ../Thesis.tex

\chapter{Verwandte Arbeiten}
\label{cha:related_work}

Das folgende Kapitel gibt eine Überblick über einige wichtige und interessante wissenschaftliche
Arbeiten, welche sich mit der Analyse von Trajektoriedaten und insbesondere Fahrzeugtrajektorien beschäftigen.
Zu Beginn werden diverse Arbeiten vorgestellt, welche sich mit der Clusteranalyse von Trajektorien befassen.
Anschließend wird betrachtet, wie in der Literatur die Erkennung von Fahrspuren, vorzugsweise auf Basis
von Trajektorien, umgesetzt wird.
Zudem werden Arbeiten untersucht, welche sich bereits mit der Klassifizierung von Fahrspuren befassen. % TODO: Evtl. entfernen
Am Ende des Kapitels werden Defizite der existierenden Lösungen festgehalten und analysiert, welche
spezifischen Neuerungen für die Umsetzung dieser Arbeit nötig sind.

\section{Clusteranalyse von Trajektorien}
\label{sec:rw_clustering}
% Hier Arbeiten und Verfahren vorstellen, welche Clustering von Trajektoriedaten vornehmen
% Verschiedene Ansätze:
%   verschiedene Distanzmaße
%   verschiedene Repräsentationen von Trajektorien
%   verschiedene Domänen und Anwendungsszenarien

% Einleitung: Wichtigkeit und Informationsreichtum Trajektorien; Daher Analyse seit geraumer Zeit;
% In verschiedenen Anwendungsgebieten und mit unterschiedlichsten Zielen; Hier vorstellung verfahren, anhand dessen ersichtlich
% werden soll, wie probleme auf unterschiedliche Art gelöst werden.

% Fu et al., 2005
%   "gewöhnliches" Clustern von Trajektorien,
%   verwendung einfaches metrischen DMs
%   Starke Vorverarbeitung der Trajektorien 
%   hierarchischer Clusteransatz Ansatz (Spectral. Nach Vergleich mit Fuzzy K-Means --> Determinismus)

% Junejo et al., 2004
%   Arbeit aus etwas anderen Domäne. Erkennung und Zuordnung von Bewegungsbahnen von Fußgängern zu Pfaden
%   Erkennung von anormalem Verhalten
%   Hier Richtung nicht wichtig, daher Verwendung von Hausdorff Distanz und Min-Cut Algorithmus
%   Repräsentation von Pfad als Cluster mit Clustermittelpunkt und Envelope, welcher Varianz beschreibt

% Atev et al., 2010
%   Hier wiederum Erkennung von Fahrzeugtrajektorie-Clustern
%   Vorstellung eines neuen DM's (mod. Hausdorff) (--> Berücksichtigung Orientierung)
%   Vergleich von verschiedenen DM's und Cluster Algos (mod HD, LCSS(siehe Grundlagen), DTW)
%   Spectral-Clustering mit eig. mod. HD liefert nach Atev. beste Ergebnisse (ebenso Erweiterung zur Bestimmung von k)

% Chen et al., 2011
%   Clustering von Hurrikan Daten
%   Hier Orientierung ebenfalls relevant aber trotzdem Verwendung einfacher HD
%   Nicht relevant: Position der Trajektorien in Datensatz
%   Daher: Darstellung Trajektorien als Flow-Vektoren (Normiert!)

% Vlachos et al., 2002
%   Nicht metrisches Distanzmaß (Verschiebungen möglich) (LCSS basiert)
%   Für Erkennung von Zeichensprach Symbolen
%   Eig. sehr aufwendig. Beschreibung eines performanten Algorithmus
%   Kompliziert im Vergleich zu Chen et al.

% Melo et al., 2014
%   Weiterer etwas anderer Ansatz
%   Extraktion von Fahrzeugen und deren Position als Graph
%   Beschreibung von Trajektorien als Polynome niedrigen Grades (Meth. kl. Quadrate) (übl. 3te Grad)
%   Verwendung von rough-k-means Clustering um bereits Ausreißer (Spurwechsel) ausfiltern zu können
%   Einfache HD Distanz
%   Initiale Spur-Mitte über Aktivitäts Map bestimmt --> Vergleich der Spuren hierzu



\section{Erkennung von Fahrspuren}
\label{sec:rw_lane_detection}
% Hier Arbeiten vorstellen, welche (prim. auf Basis von Traj.Clustern) versuchen Fahrspuren / Fahrbahnen
% zu erkennen
% Auch übliche Ansätze vorstellen, welche nicht mit Traj. arbeiten (erwähnen häufig optisch / aus Fahrzeugen)

% Zuerst Ansätze, welche auf anderen (visuellen) Methoden basieren

% Melo et al.
%   repräsentiert Fahrspuren nur über Spurmittelpunkte (Clustering-Ergebnis)
%   Bestimmt Zugehörigkeit über Nähe zu Mittellinie 

% Chen et al.
%   Spur ebenfalls "nur" über Fahrbahnmittelpunkt repräsentiert.
%   Ermittelt über "polynom curve fitting" auf Clusterpunkte

% Fu et al. & Junejo et al.
%   Clustering und dann Bestimmung Envelopes basierend auf Varianz der Trajektorien in Cluster

% Hsieh et al., 2006
% Automatic Traffic Surveillance System for Vehicle Tracking and Classification

\section{Klassifizierung von Fahrspuren}
\label{sec:rw_lane_classification}

\section{Defizite vorhandener Lösungen und benötigte Neuerungen}
\label{sec:rw_deficites}