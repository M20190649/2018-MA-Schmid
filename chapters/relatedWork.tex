%!TEX root = ../Thesis.tex

\chapter{Verwandte Arbeiten}
\label{cha:related_work}

Das folgende Kapitel gibt eine Überblick über einige wichtige und interessante wissenschaftliche
Arbeiten, welche sich mit der Analyse von Trajektoriedaten und insbesondere Fahrzeugtrajektorien beschäftigen.
Zu Beginn werden diverse Arbeiten vorgestellt, welche sich mit der Clusteranalyse von Trajektorien befassen.
Anschließend wird betrachtet, wie in der Literatur die Erkennung von Fahrspuren, vorzugsweise auf Basis
von Trajektorien, umgesetzt wird.
Zudem werden Arbeiten untersucht, welche sich bereits mit der Klassifizierung von Fahrspuren befassen. % TODO: Evtl. entfernen
Am Ende des Kapitels werden Defizite der existierenden Lösungen festgehalten und analysiert, welche
spezifischen Neuerungen für die Umsetzung dieser Arbeit nötig sind.

\section{Clusteranalyse von Trajektorien}
\label{sec:rw_clustering}
% Einleitung: Wichtigkeit und Informationsreichtum Trajektorien; Daher Analyse seit geraumer Zeit;
% In verschiedenen Anwendungsgebieten und mit unterschiedlichsten Zielen; Hier vorstellung verfahren, anhand dessen ersichtlich
% werden soll, wie probleme auf unterschiedliche Art gelöst werden.

Aufgrund der großen Menge an Informationen, welche sich auf Basis von Trajektoriedaten ermitteln lassen, ist ihre
Analyse schon seit geraumer Zeit Gegenstand wissenschaftlicher Untersuchungen.
Nachfolgend werden einige Arbeiten vorgestellt, welche sich mit der Clusteranalyse von Trajektorien beschäftigen.
Die Auswahl zeigt prototypisch, wie unterschiedliche die Anwendungsszenarien und Ziele bei solchen Analyse sind.

% Fu et al., 2005
\subsubsection*{Similarity based vehicle trajectory clustering and anomaly detection}
Eine Arbeit, welche ein sehr typisches Anwendungszenario behandelt, stammt von \cite[]{Hu2005}. Die Autoren
beschreiben in dieser Veröffentlichung ein Verfahren zur Clusteranalyse von Fahrzeugtrajektorien. Ziel dieser
ist es, auf Basis der entdeckten Spur-Cluster, anormale Verkehrsmanöver in Live-Aufnahmen von Straßenabschnitten
detektieren zu können. Solche Manöver sind beispielsweise ``Fahren abseits der üblichen Bahnen'' oder
``zu schnelles/langsames Fahren''.
Die Fahrzeugtrajektorien sind in dieser Arbeit als Sequenzen zwei-dimensionaler Punkte definiert.
Um diese zu gruppieren, setzen Hu et al. auf klassische Clusterverfahren und die Verwendung eines
einfachen, metrischen Distanzmaßes. Dieses Maß, bekannt als HU Distanz (siehe Abschnitt \ref{sec:hu_distance}),
vergleicht Trajektorien über den mittleren Abstand zwischen zusammengehörigen Punktpaaren. 
Da dies nur zuverlässig möglich ist, wenn die Trajektorien einige Bedingungen erfüllen, müssen die
Autoren diese vorverarbeiten. Sie vereinheitlichen daher die Abstände der Punkte einer Trajektorie und erweitern
sie zudem in Richtung der Szenen-Grenzen.

% TODO: Figure anpassen
\begin{figure}[H]
    \centering
    \includegraphics[width=0.5\linewidth]{../resources/img/RelatedWork/Fu_HierarchicalClustering}
    \caption[Zweistufiger Clustering-Vorgang von Hu et al.]{Zweistufiger Clustering-Vorgang von Hu et al. \cite[]{Hu2005}}
    \label{fig:relw_hu_two_step_cluster}
\end{figure}

Unter Verwendung des definierten Distanzmaßes werden die Trajektorien in einem zweistufigen Verfahren verarbeitet.
In den zwei Phasen werden, wie in Abbildung \ref{fig:relw_hu_two_step_cluster} dargestellt, zuerst dominante
Fahrpfade extrahiert, welche anschließend weiter in einzelne Fahrspuren untergliedert werden.
Als eigentliche Cluster-Algorithmen vergleichen die Autoren den \textit{Spectral-Clustering} Ansatz \cite[]{Ng2002}
mit einem \textit{Fuzzy-k-Means} Verfahren \cite[]{xie1991validity}.
Die Untersuchungen zeigen, dass der Spectral Clustering Ansatz nicht nur bessere Ergebnisse liefert, sonderen diese
über mehrere Durchläufe hinweg auch stabil sind, wohingegen die Resultate des Fuzzy-Ansatzes variieren.


% Junejo et al., 2004
\subsubsection*{Multi Feature Path Modeling for Video Surveillance}
Eine weitere Arbeit welche das Ziel hat, anormale Bewegungsmuster auf Basis von Trajektorien zu entdecken,
stammt von \cite[]{Junejo2004}. In diesem Fall geht es den Autoren allerdings nicht um das Finden von Fahrzeug-Fahrspuren,
sondern um die Extraktion von Laufpfaden von Fußgängern.
Die Bewegungsbahnen der Passanten werden aus Aufnahmen stationärer Überwachungskameras gewonnen und
als zwei-dimensionale Punktreihen repräsentiert.
Um die Trajektorien zu vergleichen, verwenden die Autoren die Hausdorff Distanz als Ähnlichkeitsmaß. 
Die üblicherweise negativen Eigenschaften dieses
Vergleichkriteriums (siehe Abschnitt \ref{sec:hausdorff_distance}), konkret die Missachtung der
Trajektorie-Orientierung, sind bei diesem Anwendungsfall kein Nachteil sondern gewünscht.
Da Fußgänger auf einem Pfad oder Weg in entgegengesetzte Richtungen gehen können, muss die Orientierung
ihrer Trajektorien ignoriert werden.
Auf Basis der Hausdorff Distanz erstellen Junejo et al. einen vollständigen Graphen, in welchem die Knoten Trajektorien
und die gewichteten Kanten den Distanzen zwischen Trajektorien entsprechen.
Sie zerlegen diesen Graphen mit Hilfe eines rekursiven \textit{min-cut}-Graphen-Algorithmus, welcher sich
an der Arbeit von \cite[]{boykov2004experimental} orientiert, und erhalten so die Cluster für die
extrahierten Fußgänger-Trajektorien.

% Atev et al., 2010
\subsubsection*{Clustering of Vehicle Trajectories}
In der Arbeit \cite[]{Atev2010} ist das Ziel der Autoren, ein Verfahren zu finden, mit welchem Fahrzeugtrajektorien
bestmöglich gruppiert werden können, ohne diese im Voraus anpassen zu müssen.
Sie vergleichen hierzu die Performance von drei unterschiedlicher Distanzmaße unter Verwendung von zwei Cluster-Algorithmen.
Primäres Augenmerk legen die Autoren auf ein von ihnen bereits in \cite[]{Atev2006} entwickeltes Distanzmaß, 
welches auf der Hausdorff Distanz basiert und sowohl die Orientierung von Trajektorien berücksichtigt
als auch robust gegenüber Ausreißern ist. Dieses neue Ähnlichkeitsmaß ist für zwei Trajektorien $P$ und $Q$
wie folgt definiert:

\begin{ceqn}
\begin{align}
    h_{\alpha, N, C}(P, Q) = \overset{\alpha}{\underset{p \in P}{ord}}\ \Big\{ \underset{q \in N_Q(C_{P,Q}(P))}{min} d(p, q) \Big\}
\end{align}
\end{ceqn}

Hierbei entspricht $C_{P,Q}$ einem Mapping $P \rightarrow Q$, welches einem Punkt $p \in P$ einen entsprechenden
Punkt $q \in Q$ zuweist, welcher die selbe relative Position in $Q$ besitzt wie $p$ in $P$.
$N_Q$ definiert ein Subset von $Q$ als Nachbarschaft des Punktes $q$. Zusammen definieren $N_Q$ und $C_{P,Q}$ eine
Struktur, in welcher der Abgleich der Trajektorien stattfindet. Dies ist visuell auch nochmals in Abbildung
\ref{fig:relw_atev_modh} dargestellt. Der Operator $ord_{p \in P}^{\alpha} f(p)$ selektiert jenen Wert aus $f(p)$, welcher
größer ist als $\alpha$-Prozent der Werte. 

\begin{figure}[H]
    \centering
    \includegraphics[width=0.6\linewidth]{../resources/img/RelatedWork/Atev_modHausdorff}
    \caption[Funktionsweise der modifizierten Hausdorff Distanz]{Funktionsweise der modifizierten Hausdorff Distanz \cite[]{Atev2010}}
    \label{fig:relw_atev_modh}
\end{figure}

Dank dieser Modifizierungen eignet sich das neue Ähnlichkeitsmaß gut für den Vergleich von Trajektorien:
$C_{P,Q}$ sorgt für den Einbezug der Orientierungen und über die Nachbarschaft $N_Q$ und $ord_{p \in P}^{\alpha} f(p)$
kann mit Ausreißern umgegangen werden.

Dieses Distanzmaß vergleichen Atev et al. unter Verwendung eines Spectral und eines Agglomerativen
Cluster-Algorithmus mit der \textit{Longest-Common Subsequence} (LCSS) und \textit{Dynamic Time Warping} (DTW) Distanz.
Die Ergebnisse der Untersuchungen für vier verschiedene Datensätze zeigen, dass die beste Cluster-Performance
mit Hilfe der modifizierten Hausdorff Distanz und des Spectral-Clustering erreicht wird.
Unter Verwendung der LCSS und DTW Distanzmaße, könnten die Autoren nicht die selben Resultate erzielen.

Dass das von Atev et al. vorgeschlagene Distanzmaß sehr gute Clusterergebnisse produziert, wurde auch von \cite{Morris2009}
bewiesen. In ihrer Untersuchung waren alledings die Ergebnisse, welche mithilfe des LCSS Maßes erreicht wurden,
ebenso gut, beziehungsweise teilweise sogar besser.

% Chen et al., 2011
\subsubsection*{Clustering of trajectories based on Hausdorff Distance}
Eine weitere interessante Arbeit zur Clusteranalyse von Trajektorien stammt von \cite[]{Chen2011}.
Die Autoren haben das Ziel, Muster in den Bewegungsbahnen von Hurrikans, welche im Zeitraum von 1850 bis 2010
über den Atlantik zogen, zu erkennen.
Sie verwenden hierzu einen angepassten DBSCAN Cluster-Algorithmus und das Hausdorff Distanzmaß.
Um die Missachtung der Orientierung kompensieren zu können, und zudem auch Ähnlichkeiten
in Sub-Trajektorien zu erkennen, wählen die Autoren eine etwas andere Darstellung der Trajektorien.
Sie definieren eine Bewegungsbahn als eine Folge sogenannter \textit{``Flow-Vektoren''}, welche neben
Positions- auch Richtungsinformationen enthalten. Ein solcher Vektor ist definiert über:

\begin{ceqn}
\begin{align}
    f_i = (x_i, y_i, dx_i, dy_i)
\end{align}
\end{ceqn}

wobei gilt:

\begin{ceqn}
\begin{align}
    dx_i = (x_{i+1} - x_i)/\sqrt{(x_{i+1} - x_i)^2 + (y_{i+1} - y_i)^2} \\
    dy_i = (y_{i+1} - y_i)/\sqrt{(x_{i+1} - x_i)^2 + (y_{i+1} - y_i)^2}
\end{align}
\end{ceqn}

Die Distanz zwischen zwei \textit{Flow-Vektoren} ist ihr euklidscher Abstand. Auf diese Weise wird bei
der Berechnung der Hausdorff Distanz (siehe Abschnitt \ref{sec:hausdorff_distance}) auch die Richtung
der Trajektorien berücksichtigt.
Um ähnliche Sub-Trajektorien entdecken zu können, teilen Chen et al. die Trajektorien an den Positionen
``charakteristischer'' Vektoren. Diese beschreiben Richtungsänderungen in einer
Bewegungsbahn und werden identifiziert über die Abweichungen in den Richtungskomponenten zweier
aufeinanderfolgender Flow-Vektoren. Dies ist anschaulich in Abbildung \ref{fig:relw_chen_flow_vector} dargestellt.

% TODO: Figure anpassen
\begin{figure}[H]
    \centering
    \includegraphics[width=0.45\linewidth]{../resources/img/RelatedWork/Chen_trajectory_splitting}
    \caption[Zerlegung einer Trajektorie in Sub-Trajektorien]{Zerlegung einer Trajektorie in Sub-Trajektorien \cite[]{Chen2011}}
    \label{fig:relw_chen_flow_vector}
\end{figure}

Die auf diese Weise erhaltenen Sub-Trajektorien werden von den Autoren mittels eines DBSCAN Algorithmus gebündelt.
Sie können so die üblichen Bewegungsbahnen von Hurrikans über dem Atlantik bestimmen.


% Vlachos et al., 2002
%   Nicht metrisches Distanzmaß (Verschiebungen möglich) (LCSS basiert)
%   Für Erkennung von Zeichensprach Symbolen
%   Eig. sehr aufwendig. Beschreibung eines performanten Algorithmus
%   Kompliziert im Vergleich zu Chen et al.

% Melo et al., 2014
%   Weiterer etwas anderer Ansatz
%   Extraktion von Fahrzeugen und deren Position als Graph
%   Beschreibung von Trajektorien als Polynome niedrigen Grades (Meth. kl. Quadrate) (übl. 3te Grad)
%   Verwendung von rough-k-means Clustering um bereits Ausreißer (Spurwechsel) ausfiltern zu können
%   Einfache HD Distanz
%   Initiale Spur-Mitte über Aktivitäts Map bestimmt --> Vergleich der Spuren hierzu



\section{Erkennung von Fahrspuren}
\label{sec:rw_lane_detection}
% Hier Arbeiten vorstellen, welche (prim. auf Basis von Traj.Clustern) versuchen Fahrspuren / Fahrbahnen
% zu erkennen
% Auch übliche Ansätze vorstellen, welche nicht mit Traj. arbeiten (erwähnen häufig optisch / aus Fahrzeugen)

% Zuerst Ansätze, welche auf anderen (visuellen) Methoden basieren

% Melo et al.
%   repräsentiert Fahrspuren nur über Spurmittelpunkte (Clustering-Ergebnis)
%   Bestimmt Zugehörigkeit über Nähe zu Mittellinie 

% Chen et al.
%   Spur ebenfalls "nur" über Fahrbahnmittelpunkt repräsentiert.
%   Ermittelt über "polynom curve fitting" auf Clusterpunkte

% Fu et al. & Junejo et al.
%   Clustering und dann Bestimmung Envelopes basierend auf Varianz der Trajektorien in Cluster

% Hsieh et al., 2006
% Automatic Traffic Surveillance System for Vehicle Tracking and Classification

\section{Klassifizierung von Fahrspuren}
\label{sec:rw_lane_classification}

\section{Defizite vorhandener Lösungen und benötigte Neuerungen}
\label{sec:rw_deficites}

% Eigentlich nichts zur Spurerkennung auf bspw. Kreuzungen etc.
%   Wenn Spur dann immer nur Spur für gesamtes Cluster (kein Partitioning)

% Meiste Aufnahmen sind Frontal-Aufnahmen auf Straße --> keine Verzerrungen