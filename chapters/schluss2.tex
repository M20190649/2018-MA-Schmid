%!TEX root = ../Thesis.tex

\chapter{Zusammenfassung und Ausblick}
\label{cha:end}

Im Rahmen dieser Arbeit wurde ein Verfahren zur automatischen Erkennung von Fahrspuren in Videoaufnahmen
anhand von Trajektoriedaten entwickelt.
Nachfolgend werden die Ergebnisse der Thesis zusammengefasst und ein kurzer Ausblick gegeben.

\section{Ergebnisse dieser Arbeit}

Eine Untersuchung der Literatur zum Thema Fahrspurerkennung zu Beginn dieser Arbeit
(siehe Kapitel \ref{cha:related_work}) machte deutlich, dass die existierenden Ansätze diverse
Defizite besitzen.
Jene Vorgehen, welche mit Fahrzeugtrajektorien arbeiten, können meist nur zur Identifikation von Fahrspuren
in sehr speziellen Straßentopologien eingesetzt werden.
Des Weiteren stimmen die in den verwandten Arbeiten ermittelten Spur-Geometrien nur selten mit den realen Dimensionen
der Fahrspuren auf der Straße überein.
Aus diesen Gründen eignen sich die existierenden Verfahren nicht für den Einsatz in einer
Anwendung, mit deren Hilfe detaillierte Verkehrsanalysen aus Luftbeobachtungen erstellt werden.

Erkannt werden Fahrspuren in dieser Arbeit anhand von Fahrzeugtrajektorien, welche aus den zu analysierenden
Luftaufnahmen rekonstruiert wurden. Das Verfahren setzt hierzu auf eine Clusteranalyse der Bewegungsbahnen
und ein neu entwickeltes, mehrstufiges Verfahren zur Erstellung von Spur-Geometrien.
Die Clusteranalyse dient der Identifikation von Spur-Clustern, welche Bewegungen von Fahrzeugen auf einer Fahrspur beschreiben.
Jene Cluster können auch in Trajektorie-Datensätzen mit komplexen Fahrbahnverläufen identifiziert werden.
Die Ergebnisse der Clusteranalyse dienen als Grundlage für die Ermittlung der Spur-Geometrien.
Zur Erstellung dieser wurde ein robustes Verfahren für die Bestimmung der Spurmittellinien entwickelt.
Anhand der Mittellinien und deren relativer Lage zueinander, werden Hülllinien definiert,
welche die Dimensionen der Spur-Geometrien beschreiben. Diese werden partitioniert,
um Spurüberlagerungen zu entfernen, und in einem finalen Schritt nochmals optimiert,
bevor sie in den Verkehrsanalysen verwendet werden können.

In einer Evaluation konnte bestätigt werden, dass das entwickelte Spurerkennungsverfahren die in Abschnitt
\ref{sec:requirements} definierten Anforderungen erfüllt und die oben genannten Probleme der verwandten Arbeiten löst.
Mit seiner Hilfe können Fahrspuren in Straßenabschnitten mit unterschiedlichsten Fahrbahnverläufen identifiziert werden.
Die ermittelten Spur-Geometrien stimmen zudem meist sehr gut mit den realen Fahrbahnverläufen überein
und besitzen keine Überlagerungen.

Der entwickelte Algorithmus wurde in die Anwendung \textit{Vehicle-Tracker} integriert,
welche im MEC-View Teilprojekt ``Luftbeobachtung'' zur Analyse des Fahrverhaltens von Verkehrsteilnehmern
eingesetzt wird. Dank der automatischen Spurerkennung können in Zukunft lange Verkehrsmesskampagnen deutlich
schneller und nahezu vollständig automatisch ausgewertet werden.

% Während der Umsetzung des Verfahrens und der Auswertung wurde deutlich, dass insbesondere die Qualität der Rohdaten und dahergehend
% auch die Performance der Datenvorverarbeitung und Bereinigung ausschlaggebend für die Qualität der Fahrspurerkennung ist.
% Damit das entwickelte Verfahren zuverlässig funktioniert, muss eine ausreichende Anzahl von ununterbrochenen Trajektorien vorliegen,
% welche die Fahrbewegungen auf den Fahrspuren beschreiben.
% Außerdem wurde deutlich, dass bei der Bestimmung der Spur-Geometrien kein Verfahren entwickelt werden kann, welches in
% alle Fällen optimale Ergebnisse liefert. Eine zu starke Anpassung der Algorithmen auf eine bestimmte Spur-Topologie
% resultiert meist in schlechteten Ergebnissen in anderen Szenarien. Es wurde daher ein Verfahren entwickelt, welches in
% möglichst allen Szenarien gute Ergebnisse liefert und nicht in einem Optimale.

% Ermittelt werden die Fahrspuren anhand von Fahrzeugtrajektorien unter Verwendung einer Clusteranalyse und
% eines neu entwickelten, mehrstufigen Verfahrens zur Erstellung von Spur-Geometrien.
% Die Clusteranalyse dient der Identifikation von Spur-Clustern. Das eingesetzte Verfahren kann
% solche Cluster auch in Trajektorie-Datensätzen mit komplexen Fahrbahnverläufen identifizieren.
% Die Trajektorien eines Clusters beschreiben die Fahrbewegungen auf einer Fahrbahn.


% Deutlich wurde allerdings auch, dass eine Schwäche des entwickelten Spurerkennungsalgorithmus
% dessen Abhängigkeit von einer ausreichenden Anzahl ununterbrochener Trajektorien ist. Wenn für eine
% Fahrspur eines Straßenabschnitts nicht ausreichend Trajektorien vorliegen, oder die existierenden an sehr
% unterschiedlichen Positionen unterbrochen sind, gelingt es dem Algorithmus nicht für die Spuren
% Spur-Geometrien zu bestimmen.

\section{Ausblick}

Der im Rahmen dieser Thesis entwickelte Spurerkennungsalgorithmus ermöglicht in Zukunft eine schnellere Auswertung
des Fahrverhaltens von Fahrzeugen in Luftaufnahmen. Mithilfe der ermittelten Spurinformationen können
unter anderem Überhol- und Spurwechselvorgänge sowie das Verhalten der Fahrzeuge auf einer Spur
untereinander untersucht werden. Die Durchführung dieser Analysen ist ein wichtiger Teil bei der Erstellung
detaillierter Verkehrsanalysen. Erkenntnisse, welche aus diesen Analysen gewonnen werden, können, im
Rahmen des MEC-View Projektes, in Zukunft auch zur Optimierung des Fahrverhaltens von autonomen
Fahrzeugen eingesetzt werden.

Die Spur-Geometrien könnten zudem zur Evaluierung der Qualität der Fahrzeugerkennung und Verfolgung,
und insbesondere zur Identifikation von Ausreißern, eingesetzt werden.
Da Fahrzeuge, deren Fahrverhalten untersucht werden soll, sich auf den Fahrbahnen eines Straßenabschnittes befinden,
kann anhand der Spurinformationen ermittelt werden, welche Trajektorien außerhalb dieser liegen und
daher vermutlich Ausreißer sind.

Die Zuverlässigkeit der Fahrspurerkennung könnte in Zukunft durch die Identifizierung und Implementierung
eines alternativen Clustering-Verfahrens weiter gesteigert werden.
Der derzeit eingesetzte Ansatz, welcher komplette Trajektorien gruppiert, liefert zwar in der Mehrzahl der Fälle
gute Ergebnisse, ist jedoch auch dafür verantwortlich, dass in manchen Situationen Fahrspuren nicht erkannt werden.
Ein alternatives Verfahren könnte sich beispielsweise an der Arbeit von \cite[]{Xu2015} orientieren, in welcher
anhand von Trajektoriedaten eine \textit{``Heat-Map''} erstellt wird, aus welcher anschließend ``Mittellinien''
anhand eines \textit{``Adaptive Multi-Kernel-Based Shrinkage''}-Algorithmus extrahiert werden.
Hierbei fallen Unterbrechungen von Trajektorien nicht ins Gewicht.
Da die einzelnen Schritte der Spurerkennung unabhängig voneinander arbeiten, könnte ein
solches alternatives Clusterverfahren mit geringen Auswirkungen auf den Rest des Algorithmus integriert werden.