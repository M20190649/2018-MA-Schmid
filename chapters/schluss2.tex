%!TEX root = ../Thesis.tex

\chapter{Zusammenfassung und Ausblick}
\label{cha:end}

Im Rahmen dieser Arbeit wurde ein Verfahren zur automatischen Erkennung von Fahrspuren in Videoaufnahmen
anhand von Trajektoriedaten entwickelt.
Nachfolgend werden die Ergebnisse der Thesis zusammengefasst und ein kurzer Ausblick gegeben.

\section{Ergebnisse dieser Arbeit}

% Verweis Literatur und Defizite --> Auswirkungen (kein zuverlässiger Einsatz in einer weitestgehend automatisierten Auswertung von Verkehrsmessungen)
% Der in dieser Arbeit entwickelte Spurerkennungsalgorithmus löst diese Problem.
% Ermittlung von Spuren mit unterschiedlichen Geometrien (unterschiedliche Situationen) und mit guter Übereinstimmung der Geometrien zu realen Spuren
% Erreicht werden die Ergebnisse durch die Verwendung einer Clusteranalyse, mit deren Hilfe Spur-Trajektorie-Cluster identifiziert werden,
% und mehreren selbstentwickelten Verarbeitungsschritten, welche aus diesen Clustern Spur-Geometrien ableiten.
% Um diese zu ermitteln, wurde ein robustes Verfahren konzipiert, um Spurmittellinien zu bestimmen.
% Es stellte sich heraus, dass sich die Breite von Fahrspuren sehr gut anhand des Abstandes paralleler Mittellinien bestimmen lässt.
% Plausibilitätschecks, Partitionierung, Optimierung


Eine Untersuchung der Literatur im Bereich der Fahrspurerkennung zu Beginn dieser Arbeit
(siehe Kapitel \ref{cha:related_work}) machte deutlich, dass die existierenden Ansätze unterschiedliche
Defizite besitzen. Jene Vorgehen, welche Fahrspuren aus Fahrzeugtrajektorien ableiten, können meist nur für
spezielle Straßentopologien eingesetzt werden.
Da des Weiteren die ermittelten Spur-Geometrien in den verwandten Arbeiten nur selten mit den realen Dimensionen
der Fahrspuren auf der Straße übereinstimmen, eignen sich diese Verfahren nicht für den Einsatz in einer
Anwendung, mit deren Hilfe detaillierte Verkehrsanalysen von unterschiedlichsten Luftaufnahmen erstellt werden.

Der in dieser Arbeit entwickelte Spurerkennungsalgorithmus löst die genannten Probleme der verwandten Arbeiten.
Mit seiner Hilfe können Fahrspuren in Straßenabschnitten mit unterschiedlichsten Fahrbahnverläufen identifiziert werden.
Die ermittelten Spur-Geometrien stimmen außerdem meist sehr gut mit den realen Fahrbahnverläufen überein.

Ermittelt werden die Fahrspuren anhand von Fahrzeugtrajektorien unter Verwendung einer Clusteranalyse und
eines neu entwickelten, mehrstufigen Verfahrens zur Erstellung von Spur-Geometrien.
Die Clusteranalyse dient der Identifikation von Spur-Clustern. Das eingesetzte Verfahren kann
solche Cluster auch in Trajektorie-Datensätzen mit komplexen Fahrbahnverläufen identifizieren.
Die Trajektorien eines Clusters beschreiben die Fahrbewegungen auf einer Fahrbahn.
Sie dienen als Grundlage für die Ermittlung der Spur-Geometrien. Um diese zu erstellen, wurde
ein robustes Verfahren zur Bestimmung der Spurmittellinien entwickelt. Anhand der Mittellinien werden
anschließend die Spur-Geometrien ermittelt. 
% TODO: Erweitern & Verbessern

In einer Evaluation konnte bestätigt werden, dass der entwickelte Spurerkennungsalgorithmus die gestellten
Anforderungen (siehe Abschnitt \ref{sec:requirements}) erfüllt. 
Deutlich wurde allerdings auch, dass eine Schwäche des entwickelten Spurerkennungsalgorithmus
dessen Abhängigkeit von einer ausreichenden Anzahl ununterbrochener Trajektorien ist. Wenn für eine
Fahrspur eines Straßenabschnitts nicht ausreichend Trajektorien vorliegen, oder die existierenden an sehr
unterschiedlichen Positionen unterbrochen sind, gelingt es dem Algorithmus nicht für die Spuren
Spur-Geometrien zu bestimmen.
% TODO: Updaten

Der entwickelte Algorithmus wurde in die Anwendung \textit{Vehicle-Tracker} integriert,
welche im MEC-View Teilprojekt ``Luftbeobachtung'' zur Analyse des Fahrverhaltens von Verkehrsteilnehmern
eingesetzt wird. Dank der automatischen Spurerkennung können in Zukunft Luftaufnahmen des Straßenverkehrs
schneller ausgewertet werden.

\section{Ausblick}

Der im Rahmen dieser Thesis entwickelte Spurerkennungsalgorithmus ermöglicht in Zukunft eine schnellere Auswertung
des Fahrverhaltens von Fahrzeugen in Luftaufnahmen. Mithilfe der ermittelten Spurinformationen können
unter anderem Überhol- und Spurwechselvorgänge sowie das Verhalten der Fahrzeuge auf einer Spur
untereinander untersucht werden.

Die Spur-Geometrien könnten in Zukunft zudem zur Evaluierung der Qualität der Fahrzeugerkennung und Verfolgung,
und insbesondere zur Identifikation von Ausreißern, eingesetzt werden.
Da Fahrzeuge, deren Fahrverhalten untersucht werden soll, sich auf den Fahrbahnen eines Straßenabschnittes befinden,
kann anhand der Spurinformationen ermittelt werden, welche Trajektorien außerhalb dieser liegen und
daher vermutlich Ausreißer sind.

Die Zuverlässigkeit der Fahrspurerkennung könnte in Zukunft durch die Identifizierung und Implementierung
eines alternativen Verfahrens zur Clusteranalyse weiter gesteigert werden.
Der derzeit eingesetzte Ansatz, welcher komplette Trajektorien gruppiert, liefert zwar in der Mehrzahl der Fälle
gute Ergebnisse, ist jedoch auch dafür verantwortlich, dass in manchen Situationen Fahrspuren nicht erkannt werden.
Ein alternatives Verfahren könnte sich beispielsweise an der Arbeit von \cite[]{Xu2015} orientieren, in welcher
anhand von Trajektoriedaten eine \textit{``Heat-Map''} erstellt wird, aus welcher anschließend ``Mittellinien''
anhand eines \textit{``Adaptive Multi-Kernel-Based Shrinkage''}-Algorithmus extrahiert werden.
Hierbei fallen Unterbrechungen von Trajektorien nicht ins Gewicht.
Da die einzelnen Schritte der Spurerkennung unabhängig voneinander arbeiten, könnte ein
solches alternatives Clusterverfahren mit geringen Auswirkungen auf den Rest des Algorithmus integriert werden.