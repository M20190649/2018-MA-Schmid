% % Preamble BEGINN %%%%%%%%%%%%%%%%%%%%%%%%%%%%%%%%%%%%%%%%%%%%%%%%%%%%%%%%%

%%% Preamble (Dokumentenklasse)
% ------------------------------------------------------------------------
% LaTeX - Preambel ******************************************************
% ------------------------------------------------------------------------
% Dokumentklasse (Koma Script)
% ------------------------------------------------------------------------
% basiernd auf www.matthiaspospiech.de/latex/vorlagen Diplomarbeit kompakt
% ========================================================================
\documentclass[%
   paper=A4,
   final,             % fertiges Dokument
   12pt,              % Schriftgroesse der Grundschrift
   bigheadings,
   ngerman,           % wird an andere Pakete weitergereicht
   a4paper,           % Papierformat
   pagesize,         % Schreibt die Papiergroesse in die Datei.
   % oneside,          % Einseitiges Layout
   twoside,          % Zweiseitiges Layout
   openright,        % Kapitel beginnen immer auf der rechten Seite
   titlepage,        % Titel als einzelne Seite ('titlepage' Umgebung)
   plainheadsepline, % Linie unter Kolumnentitel () plain Seitenstil
   nochapterprefix,  % keine Ausgabe von 'Kapitel:'
   bibtotoc,         % Bibliographie ins TOC
   tocindent,        % eingereuckte Gliederung
   listsindent,      % eingereuckte LOT, LOF
   pointlessnumbers, % Überschriftnummerierung ohne Punkt, siehe DUDEN !
   cleardoubleempty, % Leere linke Seite bei Zweiseitenlayout vor Kapitel
   halfparskip,      % Absatz halbe Zeile Abstand
]{scrbook}%     Klassen: scrartcl, scrreprt, scrbook

%%% Alle Namen usw. im Titel und im hyperref-Paket
% ------------------------------------------------------------------------
% LaTeX - Preambel ******************************************************
% ------------------------------------------------------------------------
% pre-work
% ========================================================================
% % ToDo kennzeichnen
\newcommand{\workTodo}[1]{\textcolor{red}{todo: #1}}

% % Für Datum und Zeit in Fusszeile
% % !!!Inhalt bei Fertigstellung der Arbeit löschen
\newcommand{\workMarkDateTime}{}

% % Alle Namen werden im Titel und im hyperref-Paket eingetragen
% % !!! Überall für <Wert> das Entsprechende eintragen

 % <Typ> Studienarbeit, Dipolmarbeit, Studienarbeit oder Bachlor-Abschlussarbeit
\newcommand{\workTyp}{Masterarbeit\xspace}

 % <Titel> der Arbeit
\newcommand{\workTitel}{Fahrspurerkennung in Luftaufnahmen mittels Fahrzeugtrajektorien}

 % <Studiengang> z.B. Kommunikationstechnik
\newcommand{\workStudiengang}{Informatik\xspace}

% <Semester> mit Jahr z.B. Sommersemester 2008
\newcommand{\workSemester}{Wintersemester 2018\xspace}

% <Name> des Studenten
\newcommand{\workNameStudent}{Steffen Schmid\xspace}

% <Pruefer> Name des pr�fenden (betreuenden) Professor an der Hochschule
\newcommand{\workPruefer}{Prof. Dr. Christoph Reich\xspace}


% %%% Nur bei Abschluss-Arbeiten

% <Datum> der Abgabe der Arbeit (Eidesstatliche Erklärung)
\newcommand{\workDatum}{\today\xspace}

% <Zweitpr�fer>
\newcommand{\workZweitPruefer}{}

% <Zeitraum>
\newcommand{\workZeitraum}{01.09.2018 - 28.02.2019\xspace}


% %%% Nur bei Industrie-Arbeiten:

% <Firma>
\newcommand{\workFirma}{IT-Designers GmbH\xspace}

% <Betreuer in der Firma>
\newcommand{\workBetreuer}{Dr. Stefan Kaufmann\xspace}

%%% Preamble (Pakete)
\input{preamble/pre-packages}

\loadglsentries{glossaries}
\glossarystyle{super}
\makeglossaries

%%% Neue Befehle
\input{preamble/pre-newcommands}
\input{preamble/pre-tablecommands} % Fuer Tabellen

%%% Silbentrennung
% ------------------------------------------------------------------------
% LaTeX - Preambel ******************************************************
% ------------------------------------------------------------------------
% pre-hyphenation
% ========================================================================
\hyphenation{Koordi-naten}
\hyphenation{Un-super-vised-Tracking}
\hyphenation{Frame-für-Frame}
\hyphenation{Welt-koord-inaten}
\hyphenation{Kalibrierungs-verfahren}
\hyphenation{Welt-koordinaten-systems}
\hyphenation{Ana-lyst}
\hyphenation{Wahr-schein-lich-keits-verteil-ungen}
\hyphenation{Cluster-algo-rithmen}
\hyphenation{Gaus-sian-Mix-ture-Models}
\hyphenation{Cluster-ing-Ergeb-nisse}
\hyphenation{Fahrzeug-trajekt-orien}
\hyphenation{Kali-brierungs-ver-fahren}
\hyphenation{EM-Algo-rithmus}
\hyphenation{Ver-teilungs-Cluster-algo-rithmen}
\hyphenation{Hausdorff-Distanz}
\hyphenation{Heil-bronner-Straße}
\hyphenation{Tra-jek-torie-Clus-ter}

% % Preamble ENDE %%%%%%%%%%%%%%%%%%%%%%%%%%%%%%%%%%%%%%%%%%%%%%%%%%%%%%%%%%

% % Inhalt BEGINN %%%%%%%%%%%%%%%%%%%%%%%%%%%%%%%%%%%%%%%%%%%%%%%%%%%%%%%%%
\begin{document}
% Tabellen-Einstellungen
\input{preamble/pre-tablesettings}
% % %%%%%% Vorspiel
\begin{spacing}{1} % Vorspiel immer mit Standard-Zeilenabstand setzen
	\frontmatter
	% % Titelblatt
	%!TEX root = ../Thesis.tex

% % Neue Befehle
\newcommand{\HRule}[2]{\noindent\rule[#1]{\linewidth}{#2}} % Horiz. Linie
\newcommand{\vlinespace}[1]{\vspace*{#1\baselineskip}} % Abstand
\newcommand{\titleemph}[1]{\textbf{#1}} % Hervorheben

\begin{titlepage}
 \sffamily
      \includegraphics[scale=0.7]{resources/img/logo_itdesigners}
      \hfill
      \includegraphics[width=5cm]{resources/img/logo_hfu}
      \HRule{13pt}{1pt}
      \centering
      \Large
      \vlinespace{1}\\
      \workTyp\\[5mm]
      im Studiengang\\[5mm]
      \workStudiengang\\
      \vspace{2em}
      \huge
      \workTitel\\

\vfill
\normalsize

\begin{center}
\begin{tabular}{lcl}
   Prüfer:      &&  \workPruefer \\
   Zweitprüfer:   && \workBetreuer \\
   Firma:         && IT-Designers GmbH \\
   Vorgelegt am:  && \workDatum \\
   Vorgelegt von: &&  Steffen Schmid \\
                  && 257721 \\
                  && Grünenbergstraße 32 \\
                  && 73066 Uhingen \\
                  && \href{mailto:steffen.schmid@hs-furtwangen.de}{steffen.schmid@hs-furtwangen.de} \\
\end{tabular}
\end{center}

\end{titlepage}

	%!TEX root = ../Thesis.tex

\chapter*{Danksagung} % oder Danksagung; Optional

An dieser Stelle möchte ich mich bei allen Personen bedanken, die mit ihrer tatkräftigen Unterstützung zum
Gelingen dieser Masterarbeit beigetragen haben.

Ganz besonders gilt dieser Dank meinen beiden Betreuern Herrn Prof. Dr. Christoph Reich von
der Hochschule Furtwangen und Herrn Dr. Stefan Kaufmann von der Firma IT-Designers GmbH, welche
mich sehr unterstützt haben und immer für Fragen zur Verfügung standen.
Vielen Dank für die aufgebrachte Zeit und die guten Ratschläge.

Des Weiteren gilt mein Dank der Firma IT-Designers GmbH, welche mir diese Arbeit ermöglicht hat und durch eine
tolle Arbeitsatmosphäre zu ihrem Gelingen beigetragen hat.

Schlussendlich möchte ich mich noch bei meiner Familie bedanken, welche mich während meines
kompletten Studiums unterstützt und motiviert hat.
	%!TEX root = ../Thesis.tex

\chapter{Kurzfassung}

In dieser Arbeit wurde ein Verfahren zur automatischen Erkennung von Fahrspuren in Luftaufnahmen
auf Basis von Trajektoriedaten entwickelt. Umgesetzt wurde die Thesis im Rahmen des vom \acrshort*{bmwi}
geforderten Forschungsprojektes MEC-View, in welchem unter anderem auch das Fahrverhalten von Fahrzeugen
anhand von Luftbeobachtungen untersucht wird. Fahrspurinformationen sind für solche Analysen
wichtig, da nur mit ihnen beispielsweise Überhol- oder Spurwechselvorgänge untersucht werden können.

Das im Rahmen dieser Arbeit entwickelte Verfahren kann Fahrspuren in unterschiedlichen Straßentopologien
erkennen. Die Verläufe und Breiten der algorithmisch definierten Spuren stimmen zudem meist gut mit den realen
Fahrspurverläufen überein. In diesen Punkten geht das entwickelte Verfahren über die bislang in den
verwandten Arbeiten existierenden Ansätze zur Fahrspuridentifikation hinaus.
Der Spurerkennungsalgorithmus nutzt eine Clusteranalyse und verschiedene, selbst entwickelte Verfahren zur
Ableitung von Fahrspur-Geometrien aus Trajektoriedaten.

In einer Auswertung und Evaluierung wurden die Stärken und Schwächen des entwickelten Algorithmus ermittelt.
Es zeigt sich, dass dieser in den meisten untersuchten Luftaufnahmen zuverlässig Fahrspuren identifizieren kann.
Primäre Anforderung hierfür ist, dass jede Fahrspur eines Straßenabschnitts von ausreichend vielen ununterbrochenen
Trajektorien beschrieben wird.

Der entwickelte Algorithmus wurde in die Anwendung \textit{Vehicle-Tracker} integriert,
welche im MEC-View Teilprojekt ``Luftbeobachtung'' zur Analyse des Verkehrs eingesetzt wird.
Dank der automatischen Spurerkennung können in Zukunft Luftaufnahmen des Straßenverkehrs schneller
ausgewertet werden.


% Ziel der automatischen Spurerkennung war es, Fahrspuren in möglichst vielen verschiedenen Straßenabschnitten
% identifizieren zu können. Diese algorithmisch bestimmten Fahrspuren sollten außerdem mit den realen Abmaßen
% der Spuren auf der Straßen bestmöglich übereinstimmen.

	% % Verzeichnisse
	\tableofcontents
	\listoffigures
	% \listoftables
	\lstlistoflistings
	\printglossary[type=\acronymtype, title=Abkürzungsverzeichnis]
\end{spacing}

% % %%%%%% Textteil (Eigentliche Arbeit)
\mainmatter
%
%!TEX root = ../Thesis.tex

\chapter{Einleitung}
\label{cha:introduction}

Staus und zäh fließender Verkehr sind sowohl auf Schnell- und Autobahnen, als auch in Städten ein großes
Problem und Ärgernis für Autofahrer. Sie kosten diese nicht nur wertvolle Zeit, sondern auch viel Geld.
Laut einer Studie von \cite[]{Cookson} kosten Staus jeden deutschen Autofahren pro Jahr durchschnittlich 1770 €.
In Summe ergeben sich hieraus beinahe 80 Milliarden Euro an Kosten.
Staus sind allerdings nicht nur finanziell für Privatpersonen oder Unternehmen ein Problem,
sondern sie erhöhen auch das Unfallrisiko und tragen maßgeblich zur schlechten Luftqualität in Innenstädten bei.
Aufgrund längerer Fahrzeiten und der häufigen Be- und Entschleunigung, steigt der Kraftstoffverbrauch der
Fahrzeuge und dadurch auch die Schadstoffbelastung in der Luft \cite[]{Hemmerle2016}.

Die wichtigste Voraussetzung, um Staus präventiv entgegenwirken zu können, ist, den Verkehr so gut wie
möglich zu verstehen. Nötig ist ein Verständnis des Straßenverkehrs als Ganzes, sowie der Auswirkungen,
welche einzelne Verkehrsteilnehmer und deren Verhalten, auf diesen haben. Hierzu ist das Erstellen von
Simulationen sowie die Auswertung realer Verkehrsaufkommen in Analysen unerlässlich.
Die auf diese Weise gesammelten Erkenntnisse bilden die Grundlage, um Straßenabschnitte und Infrastrukturanlagen
wie Ampeln, insbesondere auch in Innenstädten, intelligent zu gestalten. In Zukünft könnten auch
autonome Fahrzeuge von Erkenntnissen aus Verkehrsanalysen profitieren.

Dank der Tatsache, dass unbemannte Luftfahrzeuge (\acrshort*{uav}) wie Drohnen immer leichter und günstiger
verfügbar sind, und die von ihnen erstellten Aufnahmen teils eine sehr gute Qualität besitzen, werden
diese immer häufiger zur Analyse des Straßenverkehrs eingesetzt. Über Methoden aus dem Umfeld des
maschinellen Sehens und maschinellen Lernens können aus Luftaufnahmen eine Vielzahl an interessanter
Informationen extrahiert werden.

Diese Arbeit beschäftigt sich mit der Realisierung einer automatischen Fahrspurerkennung in Luftaufnahmen.
Hierzu werden die Trajektoriedaten von Fahrzeugen ausgewertet, welche aus Luftaufnahmen rekonstruiert wurden.
Die Analyse von Verkehrssituationen wird, aufgrund der oben genannten Probleme und der zunehmenden Relevanz
des autonomen Fahrens, immer wichtiger.
Eine automatisierte Spurerkennung ist ein wichtiger Teil des Analyseprozesses, da mit Hilfe der
erkannten Spuren unter anderem Spurwechsel- und Überholvorgänge sowie das Verhalten der Fahrzeuge
auf einer Spur untereinander untersucht werden können.

\section{Rahmen der Arbeit}
\label{sec:rahmen_arbeit}

Diese Masterthesis wurde im Rahmen des Forschungsprojektes MEC-View umgesetzt. Das Projekt und dessen
Teilprojekt Luftbeobachtung wird nachfolgend vorgestellt.

\subsection{Das Projekt MEC-View}
\label{sec:mec_view}

Das Forschungsprojekt MEC-View, welches vom Bundesministerium für Wirtschaft und Energie (BMWi) gefordert wird,
hat das Ziel, autonomes Fahren im urbanen Raum zu ermöglichen und für alle Verkehrsteilnehmer sicher zu gestalten.
Gerade in Innenstädten sind die Möglichkeiten des autonomen Fahrens, aufgrund von unübersichtlichen Kreuzungen,
Fußgängern und Fahrradfahrern, verdeckten Sichten und anderen Faktoren, begrenzt.
Abbildung \ref{fig:intro_mec_view_arch} gibt einen Überblick über das Forschungsprojekt und veranschaulicht dessen Ziel.

\begin{figure}[H]
\centering
    \includegraphics[width=0.7\linewidth]{resources/img/mec_view_arch}
\caption[Überblick MEC-View Projekt]{Überblick MEC-View Projekt \cite[]{mecViewWeb}}
\label{fig:intro_mec_view_arch}
\end{figure}

In urbanen Gebieten
sollen neben Daten von fahrzeuginternen Sensoren auch Informationen externer Infrastruktur-Sensoren verwendet werden,
damit autonome Fahrzeuge eine fundierte Verhaltensentscheidung auf Basis eines detaillierten Umfeldmodells treffen können.
Im Rahmen des Forschungsprojektes MEC-View wird eine Pilot-Anlage zur Umfelderfassung an einer vorfahrtberechtigten Straßenkreuzung
in Ulm aufgebaut und getestet. In dieser Anlage werden die Verkehrsteilnehmer über Kameras und LIDAR-Sensoren erfasst
und die ermittelten Daten über ein schnelles LTE/5G-Mobilfunknetz an einen \textit{Mobile Edge Computing} (\acrshort*{mec}) Server übertragen.
Hier werden die Daten in Echtzeit zu einem Umfeldmodell fusioniert, welches anschließend den autonomen Fahrzeugen
zur besseren Navigation zur Verfügung gestellt wird. Beteiligt an diesem Forschungsprojekt sind neben
der IT-Designers GmbH unter anderem auch die Daimler AG, die Robert Bosch GmbH, Osram, Nokia und die Universität Ulm.
Jeder Projektpartner ist verantwortlich für unterschiedliche Teilaspekte des Projektes. Die IT-Designers GmbH,
bei welcher diese Arbeit angefertigt wird, entwickelt den MEC-Server und ist verantwortlich für das
Teilprojekt \textit{Luftbeobachtung}. \cite[]{mecViewWeb}

\subsection{Das MEC-View Teilprojekt Luftbeobachtung}
\label{sec:mecview_sim}

Im MEC-View Teilprojekt \textit{Luftbeobachtung} werden Verkehrsanalysen und Simulationen erstellt, welche dabei helfen
das Verhalten des Verkehrs besser zu verstehen und es somit ermöglichen, diesen zu optimieren.
Mithilfe der Analysen kann beispielsweise untersucht werden, wie durch die Anpassung von Verkehrssteuerungsanlagen
oder durch die Änderung des Fahrverhaltens einzelner Fahrzeuge, eine Verbesserung der Verkehrssituation erreicht werden kann.
Die Erkenntnisse können insbesondere auch in die Verhaltenssteuerung von autonomen Fahrzeugen mit einfließen.
Aus diesem Grund sind entsprechende Untersuchungen auch für das MEC-View Hauptprojekt relevant.

Die Untersuchungen werden im MEC-View Projekt anhand von Luftbeobachtungen durchgeführt, welche von Drohnen getätigt werden.
In den Videoaufnahmen werden mithilfe eines neuronalen Netzes die Positionen und Fahrzeugklassen der Verkehrsteilnehmer ermittelt.
Mittels dieser kann anschließend beispielsweise die Geschwindigkeit und Beschleunigung der einzelnen Fahrzeuge bestimmt werden.
Zur Erstellung der Analysen ist es zudem wichtig, eine Kenntnis der Topologie der untersuchten Straßen, das heißt des
Verlaufs der Fahrbahnen und Spuren, zu besitzen. In Kombination mit den Fahrzeugpositionen können so interessante
Kenngrößen wie der Verkehrsfluss oder die Verkehrsdichte ermittelt werden.

\section{Motivation und Ziele}
\label{sec:motivation_goals}

Im Rahmen dieser Arbeit wird ein Verfahren zur automatischen Erkennung von Fahrspuren in Luftaufnahmen
auf Basis von Trajektoriedaten entwickelt. Die Spurerkennung wird in die Anwendung \textit{Vehicle-Tracker}
integriert, welche im Rahmen des MEC-View Luftbeobachtungs Projekt erstellt wird. Sie dient der Auswertung
von Luftbeobachtungen des Straßenverkehrs.
In der \textit{Vehicle-Tracker} Applikation mussten bislang die Fahrspurverläufe in jeder Aufnahme
händisch definiert werden. Dieser Prozess ist insbesondere dann aufwendig, wenn die zu untersuchenden
Straßenabschnitte beispielsweise mehrspurige Kreuzungen oder Kreisverkehre beinhalten. Das in dieser Arbeit
entwickelte Spurerkennungs-Modul soll die manuelle Spur-Definition weitestgehend ersetzen und es so ermöglichen
in Zukunft mehr Luftaufnahmen mit weniger Aufwand auszuwerten.

Der Verlauf und die Geometrie der Fahrspuren wird in dieser Thesis anhand der Bewegungsbahnen von Fahrzeugen, den sogenannten Trajektorien,
ermittelt. Im Gegensatz zu einer visuellen Detektierung hat das Verfahren den Vorteil, dass Fahrspuren auch in Aufnahmen
mit schlechten Lichtverhältnissen oder Verdeckungen der Fahrbahnen und Spurmarkierungen erkannt werden können.

Zum Thema Spurerkennung existieren zwar bereits Veröffentlichungen (siehe Abschnitt \ref{sec:rw_lane_detection}),
allerdings können die
vorgestellten Methoden meist nur in sehr speziellen Szenarien eingesetzt werden oder die erkannten Spuren
entsprechen den realen Fahrspurverläufen nur schlecht. Ziel dieser Arbeit ist es, ein Verfahren zu entwickeln,
welches Fahrspuren in möglichst vielen unterschiedlichen Szenarien erkennen kann.
Die Spuren sollen außerdem den realen Fahrbahnverläufen so gut wie möglich entsprechen.

\section{Aufbau dieser Arbeit}
\label{sec:aufbau}

Die vorliegende Arbeit ist wie folgt strukturiert:

\begin{itemize}
    \item Die zum Verständnis der Arbeit und des entwickelten Spurerkennungsalgorithmus benötigten
            Grundlagen sind in \textbf{Kapitel \ref{sec:position_extraction} und \ref{sec:tra_clustering}} beschrieben.
            Kapitel \ref{sec:position_extraction} erläutert, wie aus Luftaufnahmen die Trajektorien von Fahrzeugen
            rekonstruiert und in ein Weltkoodinatensystem überführt werden können.
            Kapitel \ref{sec:tra_clustering} stellt die grundlegenden Konzepte der Clusteranalyse vor, welche
            bei der Umsetzung der Spurerkennung zum Einsatz kommt.
    \item In \textbf{Kapitel \ref{cha:related_work}} werden verwandte Arbeiten, welche sich bereits mit
            der Thematik der Spurerkennung und der Clusteranalyse von Trajektorien befassen, vorgestellt und untersucht.
            Zudem werden Defizite der vorhandenen Lösungen und benötigte Neuerungen aufgezeigt.
    \item In \textbf{Kapitel \ref{cha:konzeption}} wird das Konzept für die Umsetzung der Spurerkennung vorgestellt.
            Es werden Anforderungen definiert und das Spurerkennungs-Modul wird in den Gesamtkontext
            der Applikation \textit{Vehicle-Tracker} eingeordnet.
    \item Nach der Konzeption wird in \textbf{Kapitel \ref{cha:realisation}} erläutert, wie die Spurerkennung in dieser Arbeit
            umgesetzt wurde. Es werden die verschiedenen Schritte des entwickelten Algorithmus vorgestellt.
    \item In \textbf{Kapitel \ref{cha:results}} wird der Spurerkennungsalgorithmus evaluiert.
            Es wird auf die Stärken und Schwächen der wichtigsten Verarbeitungsschritte des Algorithmus eingegangen.
            Außerdem werden konkrete Ergebnisse in Form von Screenshots der erkannten Fahrspuren vorgestellt.
    \item \textbf{Kapitel \ref{cha:end}} bildet den Schluss dieser Masterarbeit. Hier werden die Ergebnisse der
            Arbeit nochmals zusammengefasst und es wird ein Ausblick gegeben, in welchen Anwendungsgebieten die Spurerkennung
            in Zukunft eingesetzt werden kann und welche Verbesserungen an dem entwickelten Verfahren noch vorgenommen werden können.
    \item Im \textbf{\hyperref[cha:anhang_a]{Anhang}} dieser Arbeit sind Aufnahmen der Straßenabschnitte dargestellt,
            mit deren Hilfe der Spurerkennungsalgorithmus entwickelt und evaluiert wurde.
\end{itemize}


%!TEX root = ../Thesis.tex

\chapter{Grundlagen}
\label{cha:grundlagen}

In diesem Kapitel werden die für das Verständnis und die Durchführung der Arbeit benötigten
Grundlagenthemen vorgestellt. Nach einer kurzen Erläuterung der Möglichkeiten der Verkehrsanalysen
mittels Luftaufnahmen, wird daher darauf eingegangen, auf welche Weise die in dieser Arbeit verwendeten
Fahrzeugtrajektorien ermittelt werden.
Anschließend werden Methoden vorgestellt, welche zur Bereinigung der gewonnenen Daten verwendet werden können.
Als wichtiges Mittel zur Ableitung von Fahrspuren aus Trajektorien werden zudem
verschiedene Cluster-Algorithmen und Distanzmaße vorgestellt.

\section{Rekonstruktion von Fahrzeugtrajektorien aus Luftaufnahmen}
\label{sec:position_extraction}

% Beschreibung des kompletten Vorgangs bis Bewegungsbahnen der Autos vorliegen
% Tracking --> World-Matching --> initiale Glättung
% Resultat beschreiben: Koordinaten der Fahrzeuge in World-Koord.-System --> Distanzen in Metern (ermöglicht "Plausibilitätskontrollen")
% Verwendung von Bounding-Boxes (Mittelpunkt bzw. Front-Punkt identifiziert Fahrzeugposition)
% --> evlt. Problem bei Aufnahmen mit niedrigem Kamera-Winkel

% Die in dieser Arbeit verwendeten Fahrzeugtrajektorien stammen aus der Anwendung ``Tracker-Application''
% des MEC-View Teilprojektes \textit{Luftbeobachtung}. Nachfolgend wird beschrieben, wie diese aus den Videoaufnahmen
% rekonstruiert werden.

% allgemein: Es existieren unterschiedliche Ansätze zur Identifikation und Tracking von Fahrzeugen in Aufnahmen
% Einerseits: Supervised Tracking (alter Ansatz)
% Oder: Hintergrund Subtraktionsverfahren (???) (Eher nicht)
% Oder: Unsupervised Tracking. Object Detection API Tensorflow. So in TrackerApplication
%     Erkennung von Fahrzeugen in jedem einzelnen Frame des Video
%     Erstellen von Bounding Boxes --> Daraus Positionen
% Positionen in Bildkoordinaten
% Umwandlung in Weltkoordinaten (Weltkoordinaten System, Kameramodell etc.. siehe Arbeit Stefan)
% Endergebnis: Fahrzeugpositionen in Weltkoordinatensystem mit Einheit Meter

\section{Datenaufbereitung und Bereinigung}
\label{sec:tra_preprocessing}

% ALLGEMEINE Beschreibung von möglichen Datenbereinigungsschritten
% Resampling (Distanz oder Geschwindigkeit)
% Padding etc. (Interpolation)
% Glättung (RANSAC, Wavelet)

\section{Clusteranalyse}
\label{sec:tra_clustering}

Die Clusteranalyse (kurz Clustering) ist ein wichtiges Werkzeug zur Auswertung von Daten unterschiedlichster
Art. Sie stellt dabei kein konkretes Vorgehen oder einen Algorithmus dar, sondern beschreibt ein
allgemeines Problem, welches auf unterschiedlichste Weise gelöst werden kann.
Grundsätzlich ist das Ziel der Clusteranalyse, Datenobjekte aufgrund ihrer Eigenschaften und Beziehungen
untereinander so zu gruppieren, dass sich die Objekte einer Gruppe möglichst stark ähneln und sich
von den Objekten anderer Gruppen möglichst stark unterscheiden. Je höher die \textit{Homogenität} in einem Cluster
und die \textit{Differenz} zwischen den Clustern, desto besser ist die gewählte Clustering Methode.
Der Einsatz von Clustering ist in vielen Anwendungsgebieten und in den unterschiedlichsten wissenschaftlichen
Disziplinen sehr beliebt, um ein Verständnis für Daten zu erhalten beziehungsweise diese anschließend weiter
verarbeiten zu können.
So kommt die Clusteranalyse unter anderem in den Feldern des maschinellen Lernens, der Mustererkennung, Bildanalyse,
der Biologie (Taxonomie) oder im Bereich Data Mining zum Einsatz. \cite[]{tan2007introduction}

Die Clusteranalyse hat viel mit dem Problem der Klassifizierung von Daten gemein, insofern sie Datenobjekten
Label zuordnet. Im Gegensatz zu \textit{überwachten} Klassifizierungsansätzen, wie dem heute populären überwachten
Lernen, leiten Cluster-Algorithmen die Label allerdings alleine aus den vorhandenen Daten ab.
Es kommen keine Vergleichsobjekte mit bekannten, händisch vergebenen Labeln zum Einsatz.
Aus diesem Grund wird die Clusteranalyse auch häufig als \textit{unüberwachte Klassifizierung} bezeichnet. \cite[]{tan2007introduction}

Das Konzept eines \textit{Clusters} ist nicht genau definiert, was in einer Vielzahl an unterschiedlichen Ansichten
und Algorithmen resultiert, welche sich jeweils für andere Anwendungsfälle eignen und verschiedene Eigenschaften
besitzen. Hieraus ergibt sich auch die Tatsache, dass Clustering keine selbsttätiger Prozess ist, welcher sich auf
einheitliche Weise auf unterschiedliche Probleme anwenden lässt. Jedes Problem erfordert die individuelle und sorgfältige
Auswahl eines passenden Algorithmus, eines Distanzmaßes und der richtigen Parameter. Die Bestimmung dieser geschieht
iterativ und nicht selten nach dem Prinzip des \textit{Trial and Error}. In Abbildung \ref{fig:grund_clustering_example}
ist beispielhaft ein Datensatz (links) mit -- für den Menschen intuitiv ersichtlich -- 7 unterschiedlichen Clustern (rechts)
dargestellt. Nach \cite[]{Jain2010} kann allerdings kein verfügbarer Clustering Algorithmus diese alle erkennen.
\cite[]{Jain1999, tan2007introduction}

\begin{figure}[H]
    \centering
    \includegraphics[width=0.8\linewidth]{../resources/img/grundlagen/clustering_example}
    \caption[Rohdaten (links) und erwünschtes Clustering-Ergebnis (rechts)]{Rohdaten (links) und erwünschtes Clustering-Ergebnis (rechts) \cite[]{Jain2010}}
    \label{fig:grund_clustering_example}
\end{figure}

Aufgrund der Limitationen, welche alle Cluster-Algorithmen besitzen, muss der Analyst sich vor deren Anwendung intensiv
mit den zu verarbeitenden Daten beschäftigen. Er muss ein Verständnis dafür besitzen, welche Struktur die Daten
besitzen, beziehungsweise annehmen können, und nach welchen Mustern zu suchen ist.
Besonders wichtiger ist zudem auch die Auswahl der richtigen, das heißt relevanten, Datenmerkmale (\textit{``Feature Selektion''})
und die Wahl deren Repräsentation (\textit{``Feature Transformation''}).
Die Selektion und gegebenenfalls Transformation der Daten muss in einem
Vorverarbeitungsschritt geschehen, dessen Qualität einen maßgeblichen Einfluss auf das finale Clustering Ergebnis hat.
Basierend auf vorangegangener Beschreibung und \cite[]{Jain1999}, lässt sich der Ablauf einer Clusteranalyse wie folgt darstellen: \\

\begin{figure}[H]
    \centering
    \includegraphics[width=\linewidth]{../resources/img/grundlagen/clustering_flow}
    \caption{Ablauf einer Clusteranalyse}
    \label{fig:grund_clustering_workflow}
\end{figure}

\cite[]{Jain2010} nennt einige weitere Herausforderungen, welchen man sich bei der Clusteranalyse bewusst sein muss:

\begin{itemize}
    \item Daten können Ausreißer enthalten. Wie sollen diese behandelt werden?
    \item Die Anzahl der Zielcluster ist üblicherweise nicht bekannt.
    \item Validierung der gefundenen Cluster
\end{itemize}

\subsection{Eigenschaften von Cluster-Sets und Clustern}

Das aus einer Analyse resultierende Cluster-Set und die einzelnen Cluster selbst,
können in verschiedene Kategorien unterteilt werden beziehungsweise unterschiedliche Eigenschaften besitzen.
Nachfolgend sind die wichtigsten basierend auf \cite[]{tan2007introduction} und \cite[]{Jain1999,Jain2010} aufgeführt.

\subsubsection{Cluster-Sets}

Bei Cluster-Sets kann grundsätzlich zwischen nachfolgenden Eigenschaften unterschieden werden.

\paragraph{Hierarchisch vs. Partitioniert}
Von \textit{hierarchischen} Cluster-Sets wird gesprochen, wenn die einzelnen Cluster verschachtelt sind und dabei eine
Baum-Struktur bilden. Cluster sind hingegen \textit{partitioniert}, wenn keine Überlagerungen zwischen ihnen existiert.

\paragraph{Exklusiv vs. Überlappend vs. Fuzzy}
\textit{Exklusive} Cluster-Sets liegen vor, wenn jedem Datenwert ein oder kein Zielcluster zugeordnet wird.
Im Gegensatz hierzu können bei \textit{überlappenden} Cluster-Sets Objekte einer oder mehrerer Gruppen angehören.
Bei dem sogenannten \textit{Fuzzy} oder \textit{Soft} Cluster-Sets, gehört ein Datenobjekt einem Cluster
mit einer bestimmten Wahrscheinlichkeit oder Gewicht an. Algorithmen, welche Daten eine
Wahrscheinlichkeit für die Zugehörigkeit zu einem Cluster zuweisen, werden \textit{probabilistische}
Cluster-Algorithmen genannt.

\paragraph{Komplett vs. Partielle}
Von \textit{kompletten} Cluster-Sets wird gesprochen, wenn jedes Element der Eingangsdaten einem Cluster zugeordnet wird.
Bei \textit{partiellen} Sets ist dies nicht der Fall. Hier kann ein bestimmter Anteil an Datenwerten als Ausreißer markiert
werden, welche keine Gruppe besitzen.

% TODO: evtl. Bild einfügen

\subsubsection{Cluster}

Da, wie oben erwähnt, nicht klar definiert ist, was ein Cluster ausmacht, können auch diese unterschiedliche Eigenschaften
besitzen. Die wichtigsten Cluster-Arten sind nachfolgend erläutert.

\paragraph{Klar separierte Cluster}
Unter \textit{klar separiereten} Clustern versteht man solche, in welchen jedes Datenelement einen geringeren
Abstand zu allen anderen Elementen des Clusters hat, als zu Elementen außerhalb des Clusters. Diese
idealistische Definition eines Clusters ist nur dann erfüllt, wenn die in den Daten enthaltenen Cluster einen
großen Abstand voneinander haben. Dies ist in der Realität allerdings selten der Fall.

\paragraph{Prototyp basierte Cluster}
Von einem \textit{Prototyp basierten} Cluster wird gesprochen, wenn alle Elemente einer Gruppe einen
geringeren Abstand zu einem Prototyp oder Referenzwert des Clusters besitzen, als zu denen anderer Gruppierungen.
Ein solcher Prototyp ist üblicherweise der Mittelwert der Datenelemente eines Clusters (\textit{Centroid}).

\paragraph{Graphen basierte Cluster}
Die Definition eines \textit{Graphen basierten} Clusters kann immer dann verwendet werden, wenn Daten
als vernetzter Graph dargestellt werden. In einem solchen sind die Elemente Knoten und die Kanten
repräsentieren Beziehungen zwischen ihnen. Ein Cluster in einem solchen Graphen ist definiert als Menge von
Knoten, welche untereinander verbunden sind, jedoch keine Verbindungen zu Elementen außerhalb des Clusters haben.

\paragraph{Dichte basierte Cluster}
\textit{Dichte basierte} Cluster sind definiert als Regionen mit einer hohen Dichte an Objekten, welche von
Regionen umgeben sind, welche eine geringe Objektdichte besitzen. Elemente, welche in einer solchen Region
mit geringen Dichte liegen, welche aus Ausreißer interpretiert. Dichte Bereiche werden üblicherweise
gefunden, indem die Nachbarschaften von Elementen untersucht werden.

\paragraph{Konzeptionelle Cluster}
% TODO: Verweis Bild hinzufügen
Eine sehr allgemeine Definition eines Clusters ist die der \textit{konzeptionellen} Gruppen. Hiermit ist
gemeint, dass die Elemente eines Clusters einige gemeinsame Eigenschaften besitzen. Dies schließt die oben genannten
Cluster-Arten mit ein, lässt sich allerdings beliebig erweitern. So sind beispielsweise in Abbildung \ref{fig:basic_cluster_style} e)
konzeptionelle Cluster dargestellt, die die Form zweier Kreise und eines Dreiecks haben. Um solche Muster
erkennen zu können, würde ein Algorithmus eine besondere Definition eines Clusters benötigen.

\begin{figure}[H]
    \centering
    \subfloat[Klar separierte Cluster]{{
        \includegraphics[align=c, width=0.3\linewidth]{../resources/img/grundlagen/cluster/Cluster01}
    }}
    \qquad
    \qquad
    \subfloat[Centroid-basierte Cluster]{{
        \includegraphics[align=c, width=0.3\linewidth]{../resources/img/grundlagen/cluster/Cluster02}
    }}
    \hfill
    \subfloat[Graphen-basierte Cluster]{{
        \includegraphics[align=c, width=0.3\linewidth]{../resources/img/grundlagen/cluster/Cluster03}
    }}
    \qquad
    \qquad
    \subfloat[Dichte-basierte Cluster]{{
        \includegraphics[align=c, width=0.3\linewidth]{../resources/img/grundlagen/cluster/Cluster04}
    }}
    \hfill
    \subfloat[Konzeptionelle Cluster]{{
        \includegraphics[align=c, width=0.6\linewidth]{../resources/img/grundlagen/cluster/Cluster05}
    }}
    \caption[Visualisierung verschiedener Clusterarten]{Visualisierung verschiedener Clusterarten (basierend auf \cite[]{tan2007introduction})}
    \label{fig:basic_cluster_style}
\end{figure}
% TODO: Bild hinzufügen und referenzieren

\subsection{Cluster-Algorithmen}
\label{sec:cluster_algos}

Um mit den oben beschriebenen unterschiedlichen Cluster-Set und Cluster Definitionen umgehen zu können,
existieren verschiedene Clustering-Modelle.
Einige wichtige Clustering-Ansätze sind die Vernetzungs-Modelle, Centroid-basierte-Modelle, Verteilungs-Modelle
oder Dichte-Modelle. Für jedes dieser Modelle existieren unterschiedliche Algorithmen. Im Folgenden werden
diese Modelle und jeweils exemplarisch ein Algorithmus der diese vertritt vorgestellt.

\subsubsection{Vernetzungs-Modelle}

Vernetzungs-Modelle werden auch häufig \textit{hierarchische Cluster-Modelle} genannt. Sie beruhen auf
der Annahme, dass Elemente, welche nahe beieinander liegen, eine höhere Gemeinsamkeit besitzen als solche,
welche weiter voneinander entfernt sind. Zur Bestimmung der Nähe zwischen Elementen benötigen Vernetzungs-Modelle,
wie auch andere Cluster-Modelle, eine
Definition von Distanz. Diese legt ein sogenanntes \textit{Distanzmaß} fest. Zusätzlich ist ein \textit{Link-Kriterium} notwendig,
welches bestimmt, wie genau die Entfernung zwischen zwei Clustern ermittelt wird. Übliche Link-Kriterien
sind \textit{Minimum-Linkage}, welches die minimale Distanz zwischen den Objekten der Cluster als Distanz verwendet,
oder \textit{Maximum-Linkage} beziehungsweise \textit{Average-Linkage}. \cite[]{Jain1999, GeorgeSeif2018}

Grundsätzlich teilen sich hierarchische Cluster-Algorithmen in zwei Gruppen auf:
\textit{Agglomerative} (Bottom-Up) und \textit{Divisive} (Top-Down) Algorithmen.
Agglomerative Ansätze weisen zu Beginn des Cluster-Vorgangs jedem Datenelement eine eigene Gruppe zu und vereinigen
diese anschließend.
Bei divisiven Ansätzen werden hingegen zu Beginn alle Elemente in einem Cluster zusammengefasst und
diese in den nachfolgenden Schritten geteilt.

Als Beispiel wird anschließend der \textit{agglomerative-hierarchische Cluster-Algorithmus} genauer vorgestellt.
Sein Vorgehen lässt sich sehr gut anhand sogenannter Dendrogramme oder geschachtelter Cluster-Diagramme darstellen
(siehe Abbildung \ref{fig:grund_agglo_clustering})

% TODO: Bild selbst erstellen
\begin{figure}[H]
    \centering
    \includegraphics[width=0.7\linewidth]{../resources/img/grundlagen/agglo_clustering}
    \caption{Agglomeratives Clustering dargestellt als Dendrogramm und geschachteltes Cluster-Diagram}
    \label{fig:grund_agglo_clustering}
\end{figure}

Im ersten Schritt des Algorithmus werden alle Datenpunkte als separate Cluster markiert. Diesen Schritt
repräsentieren die Blätter des Dendrogramms.
Anschließend muss ein Distanzmaß und ein Link-Kriterium gewählt werden.
Das am häufigsten verwendete Distanzmaß ist sicherlich der euklidsche Abstand, welcher die Distanz zwischen zwei Punkten
oder Vektoren im $n$-dimensionalen Raum bestimmt. Er ist definiert durch die Formel \ref{eq_dist}.

\begin{ceqn}
\begin{align}
\label{eq_dist}
    dist(p,q) = ||q-p||_2 = \sqrt{\sum_{i=1}^n (q_i-p_i)^2}
\end{align}
\end{ceqn}

Wird als Link-Kriterium beispielsweise \textit{Minimum-Linkage} gewählt, ist dieses definiert als:

\begin{ceqn}
\begin{align}
\label{eq_linkage}
    link(P, Q) = min\{ dist(p,q) : p \in P, q \in Q\}
\end{align}
\end{ceqn}

Hierbei entsprechen $P$ und $Q$ zwei Clustern, welche die Elemente $p \in P$ und $q \in Q$ enthalten.
Auf Basis des gewählten Link-Kriteriums kann nun eine Distanz-Matrix für die einzelnen Cluster
erstellt werden.
Die zwei Cluster mit minimalem Abstand voneinander werden anschließend zusammengeführt und die
vorherigen Schritte werden wiederholt, bis nur noch ein Cluster (Wurzel des Dendrogramms) beziehungsweise
die gewünschte Clusteranzahl übrig ist. \cite[]{GeorgeSeif2018, tan2007introduction}

Bei den meisten Varianten des agglomerativen Clusterings muss der Nutzer die Anzahl der Zielcluster im
vorraus festlegen, was problematisch ist, da diese meist nicht bekannt ist. Umgangen werden kann dies nur,
indem ein Link-Kriterium gewählt wird, das ab einer bestimmten Distanz zwischen den Clustern diese nichtmehr
fusioniert \cite[]{GeorgeSeif2018}.

Die Zeitkomplexität des agglomerativen Clusterings beträgt bestenfalls $O(m^2log\ m)$, weshalb die Menge der Daten,
welche mit ihm verarbeitet werden können erheblich begrenzt ist \cite[]{tan2007introduction}.

\subsubsection{Centroid-Modelle}

Centroid basierte Cluster-Modelle betrachten im Gegensatz zu hierarchischen Modellen nicht die Distanz
zwischen Clustern, sondern die Entfernung von Objekten zu Referenzpunkten, sogenannten \textit{Centroids}.

Ein Beispiel für einen Centroid-Cluster-Algorithmus ist \textit{k-Means}. Dieser ist aufgrund seines Alters,
seiner Einfachheit und der vielen Weiterentwicklungen wohl der bekannteste Cluster-Algorithmus überhaupt.

Das Ziel von k-Mean ist es, für eine n-dimensionale Punktmenge $X = \{ x_1 ... x_n \}$ ein Cluster-Set $C = \{ c_1 ... c_k \}$
zu finden, welches die Summe der quadratischen Abweichung (Gleichung \ref{eq_kmeans1}) zwischen allein Punkten in einem Cluster und deren
Mittelwert $\mu_k$ (Centroids) minimiert.

\begin{ceqn}
\begin{align}
    \label{eq_kmeans1}
    J(c_k) = \sum_{k=1}^K \sum_{x_i \in c_k} || x_i - \mu_k ||^2
\end{align}
\end{ceqn}

Eine Lösung für dieses Problem zu finden, ist NP-Schwer. Aus diesem Grund
ist k-Means ein approximativer Ansatz, welcher nicht garantieren kann, ein globales Minimum zu finden.
Die Funktionsweise des Algorithmus ist in Abbildung \ref{fig:grund_kmeans_clustering} dargestellt.
Die Kreuze entsprechen hierbei den Centroids, welche sich über die Iterationen hinweg verschieben.

\begin{figure}[H]
    \centering
    \includegraphics[width=0.9\linewidth]{../resources/img/grundlagen/k-means}
    \caption[Funktionsweise von k-Means]{Funktionsweise von k-Means \cite[]{tan2007introduction}}
    \label{fig:grund_kmeans_clustering}
\end{figure}

Ausgehend von der Punktmenge $X$ und der gesuchten Cluster-Anzahl $k$,
werden im ersten Schritt $k$ zufällig positionierte Centroids $\mu_k$ definiert.
Anschließend wird für alle Punkte $x_i$ der nächstgelegene Centroid $\mu_j$ gesucht.

\begin{ceqn}
\begin{align}
    \label{eq_kmeans2}
    j = arg\ min(dist(x_i, \mu_j))
\end{align}
\end{ceqn}

$x_i$ wird daraufhin Mitglied in Cluster $C_j$. Als Distanzmaß ($dist$) kann hier wieder der euklidsche Abstand
(Gleichung \ref{eq_dist}) verwendet werden oder aber auch beliebige andere sinnvolle Metriken.
Nachdem alle Punkte $x_i$ einem Cluster zugewiesen wurden, werden die Centroid Positionen neu bestimmt.
Hierzu wird der Durchschnitt aller Punkte eines Clusters berechnet:

\begin{ceqn}
\begin{align}
    \label{eq_kmeans3}
    c_j = \frac{1}{n} \sum_{x_j \in C_j} x_j
\end{align}
\end{ceqn}

Diese zwei Schritte werden mehrfach wiederholt, bis das Ergebnis konvergiert, das heißt die Zuweisungen sich
nurnoch geringfügig ändern. \cite[]{Jain2010}

Der primäre Nachteil des k-Means Algorithmus ist, das auch bei ihm die Anzahl der Zielcluster spezifiziert
werden muss. Desweiteren ist sein Ergebnis aufgrund der zufälligen Initialisierung der Centroids
nicht deterministisch. Vorteil von k-Means ist hingegen, dass seine Zeitkomplexität bei $O(n)$ liegt.

Um die genannten Nachteile, zumindest in Teilen, umgehen zu können, existieren diverse Weiterentwicklungen des k-Mean
Algorithmus. So stammen beispielsweise von \cite[]{Hamerly} und \cite[]{Pelleg} die Algorithmen \textit{g-Means}
beziehungsweise \textit{x-Means}, welche die Clusteranzahl $k$ auf Basis mehrerer k-Means Durchläufe und
statistischer Kennzahlen bestimmen.

\subsubsection{Distributions-Modelle}

Distributions-Cluster-Modelle basieren auf der Verwendung von statistischen Wahrscheinlichkeitsverteilungen wie
beispielsweise der Gauß-Verteilung. Cluster werden darüber definiert, wie wahrscheinlich es ist, dass Objekte
der selben Verteilung angehören. Problematisch ist die Verwendung dieser Cluster-Methodik, da sie anfällig für
das Problem des \textit{``Overfitting''} ist, wenn die Komplexität der verwendeten Modelle nicht beschränkt wird.
Zudem ist die Annahme, dass vielen realen Datensätzen ein statistisches Verteilungsmodell zugrundeliegt, gefährlich.
Ist diese These jedoch berechtigt, haben die Modelle den Vorteil, dass sie neben der Zuweisung von Objekten zu Clustern
auch Korrelationen zwischen einzelnen Attributen aufzeigen können. \cite[]{AndersDrachen2014}

Nachfolgend wird der bekannteste Vertreter der Distributions-Cluster-Algorithmen vorgestellt:
das \textit{Expectation–maximization} (EM) Verfahren unter Verwendung sogenannter \textit{Gaussian-Mixture-Models} (GMM).
Die Funktionsweise des EM-Algorithmus hat grundsätzlich viel gemein mit der des k-Mean Ansatzes.
Es wird ebenfalls mit einer festen Anzahl zufällig initialisierter Modelle gestartet, welche anschließend über mehrere Iterationen
an die Daten angepasst werden. Im Gegensatz zu k-Means, sind die gewählten Modelle hingegen Gauß-Verteilungen,
welche zwei Parameter besitzen: ihren Mittelwert und die Standardabweichung.
Das Vorgehen des EM-Algorithmus ist nachfolgend, basierend auf \cite[]{GeorgeSeif2018}, beschrieben und in
Abbildung \ref{fig:grund_em_clustering} grafisch dargestellt.

\begin{description}
    \item[1)] Wahl der Clusteranzahl $k$ und Initialisierung der Gauß-Modelle für die entsprechenden Cluster.
    \item[2)] Berechnung der Wahrscheinlichkeit, dass ein Datenpunkt zu einem Cluster gehört. Je näher
              ein Datenpunkt dem Zentrum einer Gauß-Verteilung ist, desto höher die Wahrscheinlichkeit für dessen Zugehörigkeit.
    \item[3)] Basierend auf den Wahrscheinlichkeiten werden die Parameter der Verteilungen neu berechnet.
              Hierzu wird die gewichtete Summe der Datenpunkt-Positionen errechnet. Die Gewichte entsprechen dabei
              den Wahrscheinlichkeiten, dass ein Element zu einem Cluster gehört. Hierdurch werden die Gauß-Modelle automatisch
              den in den Daten enthaltenen Clustern angepasst.
    \item[4)] Wiederholdung der Schritte 2) und 3), bis das Clustering-Ergebnis konvergiert.
\end{description}

\begin{figure}[H]
    \centering
    \subfloat[Iteration 1]{{
        \includegraphics[align=c, width=0.22\linewidth]{../resources/img/grundlagen/clustering_EM/EM1}
    }}
    \subfloat[Iteration 2]{{
        \includegraphics[align=c, width=0.22\linewidth]{../resources/img/grundlagen/clustering_EM/EM2}
    }}
    \subfloat[Iteration 3]{{
        \includegraphics[align=c, width=0.22\linewidth]{../resources/img/grundlagen/clustering_EM/EM3}
    }}
    \subfloat[Iteration 4]{{
        \includegraphics[align=c, width=0.22\linewidth]{../resources/img/grundlagen/clustering_EM/EM4}
    }}
    \caption[Darstellung des EM-Cluster-Algorithmus über mehrere Iterationen]{Darstellung des EM-Cluster-Algorithmus über mehrere Iterationen \cite[]{GeorgeSeif2018}}
    \label{fig:grund_em_clustering}
\end{figure}

Ziel des EM-Algorithmus ist es, die Parameter der Gauß-Modelle so zu optimieren, dass diese die Verteilung der Daten bestmöglich beschreiben.
Am Ende des Clusterings besitzt jeder Datenwert die Zugehörigkeit-Wahrscheinlichkeiten für die einzelnen Cluster.
Ein Element wird jenem Cluster zugeordnet, für welches es die höchste Wahrscheinlichkeit besitzt.

\subsubsection{Dichte-Modelle}

Dichte basierte Cluster sind, wie oben beschrieben, definiert als Regionen hoher Objekt-Dichte, welche
von Bereichen geringer Dichte umgeben sind. Dichte-Clustering-Modelle suchen nach eben solchen Regionen.
Großer Vorteil der Algorithmen dieser Klasse ist, dass sie Cluster beliebiger Formen finden können,
nicht auf die Vorgabe einer Clusteranzahl angewiesen sind und mit Ausreißern umgehen können.

Als Vertreter der Dichte-basierten Ansätze wird nachfolgend der \textit{DBSCAN} Algorithmus
(\textit{Density-Based Spatial Clustering of Applications with Noise}), wie in \cite[]{Gao2012} beschrieben, vorgestellt.
Er verwendet als Maß für die Dichte einer Region die sogenannte \textit{$\epsilon$ -Nachbarschaft} (\textit{Eps}).
Diese selektiert für ein Objekt $p$ alle Objekte, welche innerhalb des Radius $\epsilon$ um dieses liegen:

\begin{ceqn}
\begin{align}
    \label{eq_dbscan_1}
    N_{\epsilon}(p) = \{ q | dist(p,q) \leq \epsilon \}
\end{align}
\end{ceqn}

Eine $\epsilon$ -Nachbarschaft besitzt eine hohe Dichte, wenn in ihr mindestens $MinPts$ Objekte liegen.

Basierend auf der Definition von \textit{Eps}, werden die in einem Datensatz vorhandenen Elemente in
drei Klassen unterteilt. Sie sind entweder \textit{Kern-}, \textit{Rand-} oder \textit{Ausreißer-} Objekte.
Ein Kernobjekt hat mindestens $MinPts$ andere Punkte in \textit{Eps}.
Randobjekte besitzen weniger als $MinPts$ in \textit{Eps}, liegen aber in der Nachbarschaft eines Kernobjektes.
Ausreißerobjekte sind weder Kern- noch Randobjekte.

\begin{figure}[H]
    \centering
    \subfloat[]{{
        \includegraphics[align=c, width=0.3\linewidth]{../resources/img/grundlagen/clustering_dbscan/dbscan1}
    }}
    \subfloat[]{{
        \includegraphics[align=c, width=0.3\linewidth]{../resources/img/grundlagen/clustering_dbscan/dbscan2}
    }}
    \subfloat[]{{
        \includegraphics[align=c, width=0.3\linewidth]{../resources/img/grundlagen/clustering_dbscan/dbscan3}
    }}
    \caption[Schritte des DBSCAN Algorithmus]{Schritte des DBSCAN Algorithmus, a) Rohdaten, b) Klassifizierung in Kern- (grün), Rand- (blau) und Ausreißer- (rot) Punkte, c) Cluster Ergebnis \cite[]{Gao2012}}
    \label{fig:grund_dbscan_clustering}
\end{figure}

Auf Basis der drei Objektklassen, lässt sich das Prinzip der dichte-basierten \textit{Erreichbarkeit} definieren.
Ein Objekt $q$ ist von $p$ \textit{direkt} erreichbar, wenn $p$ ein Kernobjekt ist und $q$ in dessen \textit{Eps} liegt.
In Abbildung \ref{fig:grund_dbscan_reachability} gilt dies beispielsweise für $p$ und $p_2$.
Zwei Elemente sind \textit{indirekt} erreichbar, wenn sie über eine Reihe von Zwischenschritten (direkte Relationen)
verbunden sind (transitiv). Dies ist in Abbildung \ref{fig:grund_dbscan_reachability} für $q$ und $p$ der Fall.

\begin{figure}[H]
    \centering
    \includegraphics[width=0.32\linewidth]{../resources/img/grundlagen/clustering_dbscan/reachability}
    \caption[Erreichbarkeit in DBSCAN]{Erreichbarkeit in DBSCAN \cite[]{Gao2012}}
    \label{fig:grund_dbscan_reachability}
\end{figure}

Der DBSCAN Algorithmus lässt sich, basierend auf den obigen Definitionen, informell wie folgt beschreiben:

\begin{description}
    \item[1)] Unterteilung der Objekte in die drei Objektklassen. (Abb. \ref{fig:grund_dbscan_clustering} b))
    \item[2)] Aussortierung der Ausreißer-Objekte.
    \item[3)] Wahl eines nicht zugewiesenen Kernobjektes.
    \item[4)] Erstellung eines neuen Clusters für das Kernobjekt und alle von ihm ausgehend direkt oder indirekt erreichbaren Objekte
    \item[5)] Wiederholdung der Schritte 3) und 4), bis alle Kern- und Randobjekte einem Cluster zugewiesen sind. (Abb. \ref{fig:grund_dbscan_clustering} c))
\end{description}

DBSCAN besitzt die oben beschriebenen Vorteile Dichte-basierter Cluster-Algorithmen. Dank einer Zeitkomplexität
von $O(n\ log\ n)$ kann er außerdem auch auf große Datensätze angewendet werden.
Nachteil des Ansatzes ist hingegen, dass er schlecht mit Clustern umgehen kann, welche unterschiedliche Dichten besitzen.

\section{Ähnlichkeitsmaße zum Vergleich von Fahrzeugtrajektorien}
\label{sec:distance_measures}

Bei der Clusteranalyse ist neben der Wahl des passenden Cluster-Algorithmus insbesondere
die Entscheidung, welches Ähnlichkeits- beziehungsweise Distanzmaß verwendet wird, ausschlaggebend.
Im obigen Abschnitt wurden bereits die euklidsche Distanz (Gleichung \ref{eq_dist}) als ein mögliches Distanzmaß
definiert. Dieses kann jedoch nur zur Bestimmung der Distanz zwischen $n$-dimensionalen Punkten im euklidschen Raum verwendet
werden. Dies gilt ebenso für andere einfache Maße wie die Manhatten-Distanz oder die Pearson-Distanz.

Um Fahrzeugtrajektorien korrekt gruppieren zu können, ist ein Ähnlichkeitsmaß notwendig, welches je nach Anforderungen
die unterschiedlichen Aspekte der Trajektorien vergleicht. Häufig werden die Eigenschaften Lage, Form und Länge
hierzu herangezogen. In der Literatur werden diverse Maße zum Vergleich von Trajektorien vorgestellt. Diese besitzen
alle unterschiedliche Eigenschaften, Vor- und Nachteile.

Nachfolgend werden exemplarisch drei Ähnlichkeitsmaße vorgestellt, anhand welcher ersichtlich ist, welche Abwägungen
bei der Wahl des Maßes gemacht werden müssen.
In allen drei Fällen werden die Trajektorien als Reihen 2-dimensionaler Punkte mit Länge $n$ interpretiert: $t_i = \{(x_1, y_1), (x_2, y_2), ..., (x_n, y_n)\}$.
Der $n$-te Punkt einer Trajektorie ist gegeben über $t_i(n)$ und deren Punkt-Länge über $len(t_i)$.
Die Menge der zu vergleichenden Trajektorien ist $T = \{t_1, t_2, ..., t_m\}$.
Abbildung \ref{fig:grund_trajectories} zeigt eine Auswahl möglicher Trajektorien.

\begin{figure}[H]
\centering
\includegraphics[width=0.6\linewidth]{../resources/img/grundlagen/trajectories}
\caption{Trajektorien im 2-dimensionalen Raum}
\label{fig:grund_trajectories}
\end{figure}

\subsection{HU Distanz}
\label{sec:hu_distance}

Die HU Distanz wurde erstmals in der Arbeit \textit{``Similarity based vehicle trajectory clustering and anomaly detection''}
von \cite[]{Hu2005} verwendet. Es ist ein sehr einfaches Distanzmaß, welches auf der mittleren euklidschen Distanz
zwischen zwei Trajektorien basiert. Berechnet wird die HU Distanz für zwei Trajektorien $t_1$ und $t_2$ wie folgt:

\begin{ceqn}
\begin{align}
\label{eq_hu_distance1}
    D_{HU}(t_1, t_2) &= \frac{1}{N} \sum_{n = 1}^N dist(t_1(n), t_2(n)) \\
\label{eq_hu_distance2}
    wobei\ N &= min(len(t_1), len(t_2))
\end{align}
\end{ceqn}

Aus dieser Formel lassen sich die Vor- und Nachteile der HU Distanz ableiten. Der klare Vorteile der
HU Distanz ist deren Einfachheit und die Effizienz von $O(n)$.
Nachteil ist hingegen, dass das Distanzmaß nur gut funktioniert, wenn die Trajektorien bestimmte
Kriterien erfüllen. So sollten Trajektorien, welche einem Cluster angehören, auch immer möglichst auf selber Höhe beginnen,
damit deren mittlerer Abstand nicht, aufgrund einer Verschiebung, erhöht wird.
Außerdem ist es notwendig, die Abstände zwischen den Punkten der Trajektorien auf die selbe Länge zu bringen,
damit beim paarweisen Vergleich immer Elemente verglichen werden, welche gleichweit vom Start der Spuren entfernt sind.
Diese Eigenschaften der Trajektorien müssen über einen Vorverarbeitungsschritt geschaffen werden.
Problematisch bei der Verwendung der HU Distanz ist außerdem, dass beim Vergleich zweier Trajektorien immer nur
die ersten $N$ Punkte (s. Gleichung \ref{eq_hu_distance2}) betrachtet werden. Die kann dazu führen, dass zwei
Trajektorien, welche zu Beginn fast identisch sind und später auseinanderlaufen, trotzdem einen hohen Ähnlichkeitswert besitzen
(siehe $t_5$ und $t_6$ in Abbildung \ref{fig:grund_trajectories}).

Die HU Distanz kann aufgrund der genannten Einschränken nur in speziellen Fällen oder unter Verwendung eines
Vorverarbeitungsschrittes angewandt werden. Sie liefert ansonsten suboptimale Clustering Ergebnisse.

\subsection{Hausdorff Distanz}
\label{sec:hausdorff_distance}

Die Hausdorff Distanz ist ein komplexeres Maß zur Bestimmung der Ähnlichkeit zwischen zwei Trajektorien.
Sie misst grundsätzlich den Abstand zwischen zwei nicht-leeren, ungeordneten Teilmengen $A$ und $B$ und ist für
Trajektorien definiert über die Gleichungen \cite[]{Atev2010}:

\begin{ceqn}
\begin{align}
\label{eq_hausdorff1}
    D_{HD}(t_1, t_2) &= max(h(t_1, t_2), h(t_2, t_1)) \\
\label{eq_hausdorff2}
    h(t_1, t_2) &= \underset{i\ \in\ t_1}{max}\ \underset{j\ \in\ t_2}{min}\ dist(i, j)
\end{align}
\end{ceqn}

$h(t_1, t_2)$ wird als gerichtete Hausdorff Distanz \textit{von} $t_1$ \textit{nach} $t_2$ bezeichnet.
Sie findet die maximale Distanz einer Trajektorie zum nächsten Punkt der anderen Trajektorie \cite[]{Huttenlocher}.
Da $h$ gerichtet ist, gilt $h(t_1, t_2) \neq h(t_2, t_1)$. Aus diesem Grund wird die Hausdorff Distanz
\textit{zwischen} zwei Trajektorien mittels $D_{HD}$ bestimmt. $dist$ kann ein beliebiges Maß für die Distanz zweier
Punkte sein, wie beispielsweise die euklidsche Distanz.
Grundsätzlich lässt sich über die Hausdorff Distanz die Form zweier Trajektorien vergleichen. Diese sind ähnlich,
wenn jeder Punkt einer Trajektorie einen nahegelegenen Punkt in der Vergleichsbahn besitzt.

Vorteil der Hausdorff Distanz im Vergleich zur HU Distanz ist, dass diese immer vollständige Trajektorien vergleicht
und nicht nur Teile. Außerdem ist bei ihrer Verwendung keine Vorverarbeitung in Form von Resampling et cetera notwendig.
Problematisch ist das Distanzmaß hingegen, da es mit ungeordneten Sets arbeitet und somit im Fall von Trajektorien deren
Orientierung nicht beachtet. Zwei parallel aber in entgegengesetzte Richtungen laufende Trajektorien würden
nach Hausdorff daher eine hohe Ähnlichkeit besitzen.
Zudem kann das Distanzmaß schlecht mit Ausreißern umgehen, da bereits ein einzelner dieser Punkte, bei ansonsten identischen
Trajektorien, zu einer beliebig kleinen Ähnlichkeit führen kann.
Von Nachteil ist auch, dass die Zeitkomplexität der Hausdorff-Distanz bei $O(n\ m)$ liegt.

\subsection{Longest-Common-Subsequence}
\label{sec:lcss_distance}

Das \textit{Longest-Common-Subsequence} (LCSS) Ähnlichkeitsmaß basiert auf dem allgemeinen Problem der Findung
einer längsten gemeinsamen Subsequenz zwischen zwei Sequenzen. Da Trajektorien, nach obiger Definition, lediglich Punktfolgen sind,
lässt sich das Verfahren sehr gut auf diese anwenden. Aufgrund einiger kleiner Erweiterungen des Basis-Algorithmus, besitzt
das LCSS Ähnlichkeitsmaß einige besondere Eigenschaften. Der LCSS Algorithmus für Trajektorien ist grundsätzlich
wie folgt definiert \cite[]{Vlachos2002}:

\begin{ceqn}
\begin{align}
\label{eq_lcss}
    LCSS_{\epsilon, \delta}(t_1, t_2) =
    \begin{cases}
        0 & \text{if } t_1 \text{ or } t_2 \in \emptyset \\
        1 + LCSS_{\epsilon, \delta}(t_1', t_2') & \text{if } dist(t_1(n), t_2(m)) < \epsilon \\
        & \land\ |n - m| \leq \delta \\
        max(LCSS_{\epsilon, \delta}(t_1', t_2), LCSS_{\epsilon, \delta}(t_1, t_2')) & \text{otherwise}
    \end{cases}
\end{align}
\end{ceqn}

Hierbei gilt $t_1' = \{ t_1(0),\ ...,\ t_1(n-1)\}$. Die Parameter $\epsilon$ und $\delta$ bestimmen das
Vergleichs-Verhalten des Algorithmus. Über $\epsilon$ wird definiert, wieweit zwei Punkte maximal voneinander entfernt liegen
können, um immer noch als ``übereinstimmend'' zu gelten. $\delta$ bestimmt hingegen, wieweit man in der Zeit gehen darf, um einen
übereinstimmenden Punkt zu finden. Abbildung XXX veranschaulicht die Bedeutung von $\epsilon$ und $\delta$.
Da die obige LCSS Funktion nur ein diskretes Zählmaß definiert, ist das eigentliche LCSS Ähnlichkeitsmaß üblicherweise
gegeben über \cite[]{Vlachos2002}:

\begin{ceqn}
\begin{align}
    D_{LCSS}(\delta, \epsilon, t_1, t_2) = 1 - \frac{LCSS_{\delta, \epsilon}(t_1, t_2)}{min(len(t_1), len(t_2))}
\end{align}
\end{ceqn}

% TODO: Bild (LCSS Parametererläuterung) hinzufügen und referenzieren (Vlachos et al.)

Vorteile der LCSS Ähnlichkeitdefinition sind, dass sie mit kompletten Trajektorien arbeitet und robust
gegenüber Ausreißern ist, da nicht für alle Punkte Übereinstimmungen in den Trajektorien gefunden werden müssen.
Über $\epsilon$ und $\delta$ kann die ``Strenge'' des Algorithmus geregelt werden.
Zudem berücksichtigt das LCSS Maß die Orientierung der Trajektorien, solange $\delta$ nicht zu hoch gewählt wird.
Die rekursive Definition des LCSS Algorithmus aus Gleichung \ref{eq_lcss} lässt sich mittels dynamischer Programmierung
mit Zeitkomplexität $O(n\ m)$ berechnen.

\subsubsection{Übersicht Ähnlichkeitsmaße}

Anhand der drei ausgewählten und oben exemplarisch beschriebenen Ähnlichkeitsmaße, ist bereits ersichtlich,
dass die Wahl eines passenden Maßes nicht trivial ist. Es muss die Qualität und Form der Daten berücksichtigt werden
und abgewogen werden, in wieweit es möglich beziehungsweise gewünscht ist, die Daten vorzuverarbeiten.
Das primäre Auswahlkriterium ist allerdings natürlich die situationsabhängige Definition von ``Ähnlichkeit'':
Sind sich Trajektorien ähnlich, wenn sie lediglich die selbe Form haben und ansonsten irgendwo im Raum liegen? Sind sie
sich ähnlich, wenn sie die selbe Form haben und im Raum nahe beieinander liegen? Ist ihre Orientierung relevant?
Dies sind wichtige Fragen, welche vor der Wahl eines Ähnlichkeitsmaßes geklärt werden müssen.
Da die Maße als Distanzfunktionen bei der Clusteranalyse verwendet werden, ist ihr Verhalten ausschlaggebend
für den Erfolg der Gruppierung.

Wichtige Eigenschaften einiger in der Literatur häufig verwendeten Vergleichsmaße, inklusive der drei oben beschriebenen,
sind nachfolgend nochmals in tabellarischer Form festgehalten.

\begin{table}[H]
    \caption{Parallelen Bienenkolonie und Cloud Load-Balancing}
    \label{tab:parallelen}
    \centering
    \begin{tabular}{l|ccc}
        \toprule
        \textbf{Ähnlichkeitsmaß} & \textbf{gerichtet} & \textbf{PreProc. nötig} & \textbf{Ausreißer-resistent} \\
        \midrule \addlinespace
        HU \cite[]{Hu2005} & \cmark & \cmark & \cmark \\
        \addlinespace
        PCA \cite[]{Bashir2003} & \cmark & \cmark & \cmark \\
        \addlinespace
        DTW \cite[]{Keogh2000} & \cmark & \xmark & \xmark \\
        \addlinespace
        HD \cite[]{Chen2011} & \xmark & \xmark & \xmark \\
        \addlinespace
        mod. HD \cite[]{Atev2006} & \cmark & \xmark & \cmark \\
        \addlinespace
        PF \cite[]{Piciarelli2006} & \cmark & \xmark & \xmark \\
        \addlinespace
        LCSS \cite[]{Vlachos2002} & \cmark & \xmark & \cmark \\
        \addlinespace
        \bottomrule
    \end{tabular}
\end{table}

%!TEX root = ../Thesis.tex

\chapter{Verwandte Arbeiten}
\label{cha:related_work}

Das folgende Kapitel gibt eine Überblick über wissenschaftliche
Arbeiten, welche sich bereits mit der Analyse von Trajektoriedaten und der Erkennung von Fahrspuren beschäftigen.
Zu Beginn werden Arbeiten vorgestellt, welche sich mit der Clusteranalyse von Trajektorien befassen.
Anschließend wird betrachtet, wie in der Literatur die Erkennung von Fahrspuren auf Basis
von Trajektorien umgesetzt wird.
Am Ende des Kapitels werden Defizite der existierenden Lösungen aufgezeigt und analysiert, welche
spezifischen Neuerungen für die Umsetzung dieser Arbeit nötig sind.

\section{Clusteranalyse von Trajektorien}
\label{sec:rw_clustering}

Aufgrund der vielen Erkenntnisse welche aus Trajektoriedaten gewonnen werden können, ist ihre
Auswertung schon seit geraumer Zeit Gegenstand wissenschaftlicher Untersuchungen.
Nachfolgende Arbeiten beschäftigen sich mit der Clusteranalyse von Trajektorien.
Die Auswahl zeigt prototypisch, wie unterschiedliche die Anwendungsszenarien und Ziele bei solchen Analyse sind.

% Fu et al., 2005
\subsubsection*{Similarity based vehicle trajectory clustering and anomaly detection}
Eine Arbeit, welche ein sehr typisches Anwendungszenario behandelt, stammt von \cite[]{Hu2005}. Die Autoren
beschreiben in dieser Veröffentlichung ein Verfahren zur Clusteranalyse von Fahrzeugtrajektorien. Ziel dieser
ist es, auf Basis der entdeckten Spur-Cluster, anormale Verkehrsmanöver in Live-Aufnahmen von Straßenabschnitten
detektieren zu können. Solche Manöver sind beispielsweise ``Fahren abseits der üblichen Bahnen'' oder
``zu schnelles/langsames Fahren''.
Die Fahrzeugtrajektorien sind in dieser Arbeit als Sequenzen zwei-dimensionaler Punkte definiert.
Um sie zu gruppieren, setzen Hu et al. auf klassische Clusterverfahren und die Verwendung eines
einfachen, metrischen Distanzmaßes. Dieses Maß, bekannt als HU-Distanz (siehe Abschnitt \ref{sec:hu_distance}),
vergleicht Trajektorien über den mittleren Abstand zwischen Punktpaaren.
Da dies nur zuverlässig möglich ist, wenn die Trajektorien einige Bedingungen erfüllen, müssen die
Autoren diese vorverarbeiten. Sie vereinheitlichen daher die Abstände der Punkte einer Trajektorie und erweitern
sie in Richtung der Szenen-Grenzen.

\begin{figure}[H]
    \centering
    \includegraphics[width=0.5\linewidth]{resources/img/RelatedWork/Fu_HierarchicalClustering}
    \caption[Zweistufiger Clustering-Vorgang von Hu et al.]
            {Zweistufiger Clustering-Vorgang von \cite[]{Hu2005}; Identifikation von dominanten Pfaden und Fahrspuren}
    \label{fig:relw_hu_two_step_cluster}
\end{figure}

Unter Verwendung des definierten Distanzmaßes werden die Trajektorien in einem zweistufigen Verfahren verarbeitet.
In den zwei Phasen werden, wie in Abbildung \ref{fig:relw_hu_two_step_cluster} dargestellt, zuerst dominante
Fahrpfade extrahiert, welche anschließend weiter in einzelne Fahrspuren untergliedert werden.
Als Clusteralgorithmen vergleichen die Autoren den \textit{Spectral-Clustering} Ansatz \cite[]{Ng2002}
mit einem \textit{Fuzzy-k-Means} Verfahren \cite[]{xie1991validity}.
Die Untersuchungen zeigen, dass der Spectral Clustering Ansatz nicht nur bessere Ergebnisse liefert, sonderen dieser
über mehrere Durchläufe hinweg auch stabil sind, wohingegen die Resultate des Fuzzy-Ansatzes variieren.


% Junejo et al., 2004
\subsubsection*{Multi Feature Path Modeling for Video Surveillance}
Eine weitere Arbeit welche das Ziel hat, anormale Bewegungsmuster auf Basis von Trajektorien zu entdecken,
stammt von \cite[]{Junejo2004}. In diesem Fall geht es den Autoren allerdings nicht um das Finden von Fahrzeug-Fahrspuren,
sondern um die Extraktion von Laufpfaden von Fußgängern.
Die Bewegungsbahnen der Passanten werden aus Aufnahmen stationärer Überwachungskameras gewonnen und
als zwei-dimensionale Punktreihen repräsentiert.
Zum Vergleich der Trajektorien, verwenden die Autoren die Hausdorff-Distanz als Distanzmaß.
Die üblicherweise negativen Eigenschaften dieses
Vergleichkriteriums (siehe Abschnitt \ref{sec:hausdorff_distance}), konkret die Missachtung der
Trajektorie-Orientierung, sind bei diesem Anwendungsfall kein Nachteil, sondern gewünscht.
Da Fußgänger auf einem Weg in entgegengesetzte Richtungen gehen können, muss die Orientierung
ihrer Trajektorien ignoriert werden.
Auf Basis der Hausdorff-Distanz erstellen Junejo et al. einen vollständigen Graphen, in welchem die Knoten Trajektorien
und die gewichteten Kanten den Distanzen zwischen Trajektorien entsprechen.
Sie zerlegen diesen Graphen mit Hilfe eines rekursiven \textit{min-cut}-Graphen-Algorithmus, welcher sich
an der Arbeit von \cite[]{boykov2004experimental} orientiert, und erhalten so die Cluster für die
extrahierten Fußgänger-Trajektorien.

% Atev et al., 2010
\subsubsection*{Clustering of Vehicle Trajectories}
\label{sec:atev_et_al}
In \cite[]{Atev2010} ist das Ziel der Autoren, ein Verfahren zu finden, mit welchem Fahrzeugtrajektorien
bestmöglich gruppiert werden können, ohne diese im Voraus anpassen zu müssen.
Sie vergleichen hierzu die Performance von drei unterschiedlichen Distanzmaßen unter Verwendung von zwei Clusteralgorithmen.
Primäres Augenmerk legen die Autoren auf ein von ihnen bereits in \cite[]{Atev2006} entwickeltes Distanzmaß,
welches auf der Hausdorff-Distanz basiert und sowohl die Orientierung von Trajektorien berücksichtigt
als auch robust gegenüber Ausreißern ist. Dieses neue Maß ist für zwei Trajektorien $P$ und $Q$
wie folgt definiert:

\begin{ceqn}
\begin{align}
\label{eq_modHausdorff}
    h_{\alpha, N, C}(P, Q) = \overset{\alpha}{\underset{p \in P}{ord}}\ \Big\{ \underset{q \in N_Q(C_{P,Q}(p))}{min} d(p, q) \Big\}
\end{align}
\end{ceqn}

Hierbei entspricht $C_{P,Q}$ einem Mapping $P \rightarrow Q$, welches einem Punkt $p \in P$ einen entsprechenden
Punkt $q \in Q$ zuweist, welcher die selbe relative Position in $Q$ besitzt wie $p$ in $P$.
$N_Q$ definiert ein Subset von $Q$ als Nachbarschaft des Punktes $q$. Zusammen definieren $N_Q$ und $C_{P,Q}$ eine
Struktur, in welcher der Vergleich der Trajektorien stattfindet. Dieses Vorgehen wird in Abbildung
\ref{fig:relw_atev_modh} visualisiert. Der Operator $ord_{p \in P}^{\alpha} f(p)$ selektiert jenen Wert aus $f(p)$, welcher
größer ist als $\alpha$-Prozent der Werte.

\begin{figure}[H]
    \centering
    \includegraphics[width=0.6\linewidth]{resources/img/RelatedWork/Atev_modHausdorff}
    \caption[Funktionsweise der modifizierten Hausdorff-Distanz]{Funktionsweise der modifizierten Hausdorff-Distanz \cite[]{Atev2010}}
    \label{fig:relw_atev_modh}
\end{figure}

Dank dieser Modifizierungen eignet sich das Distanzmaß für den Vergleich von Trajektorien:
$C_{P,Q}$ sorgt für den Einbezug der Orientierung und über die Nachbarschaft $N_Q$ und $ord_{p \in P}^{\alpha} f(p)$
kann der Einfluss von Ausreißern minimiert werden.

Das Distanzmaß vergleichen Atev et al. unter Verwendung eines Spectral und eines Agglomerativen
Clusteralgorithmus mit der \textit{Longest-Common-Subsequence} (LCSS) und der \textit{Dynamic-Time-Warping}-Distanz (\acrshort*{dtw}).
Die Ergebnisse der Untersuchungen für vier verschiedene Datensätze zeigen, dass die beste Cluster-Performance
mit Hilfe der modifizierten Hausdorff-Distanz und des Spectral-Clustering erreicht wird.

Dass das von Atev et al. vorgeschlagene Distanzmaß gute Clusterergebnisse produziert, wurde auch von \cite[]{Morris2009}
bestätigt. In ihrer Untersuchung waren alledings die Ergebnisse, welche mithilfe des LCSS Maßes erreicht wurden,
ebenso gut oder teilweise besser.

% Chen et al., 2011
\subsubsection*{Clustering of trajectories based on Hausdorff Distance}
Eine weitere interessante Arbeit zur Clusteranalyse von Trajektorien stammt von \cite[]{Chen2011}.
Die Autoren haben das Ziel, Muster in den Bewegungsbahnen von Hurrikans, welche im Zeitraum von 1850 bis 2010
über den Atlantik zogen, zu erkennen.
Sie verwenden hierzu einen angepassten DBSCAN Clusteralgorithmus und das Hausdorff-Distanzmaß.
Um die Missachtung der Orientierung kompensieren zu können, und zudem auch Ähnlichkeiten
in Sub-Trajektorien zu erkennen, wählen die Autoren eine etwas andere Darstellung der Trajektorien.
Sie definieren eine Bewegungsbahn als eine Folge sogenannter \textit{``Flow-Vektoren''}, welche neben
Positions- auch Richtungsinformationen enthalten. Ein solcher Vektor ist definiert über:

\begin{ceqn}
\begin{align}
    f_i = (x_i, y_i, dx_i, dy_i)
\end{align}
\end{ceqn}

wobei gilt:

\begin{ceqn}
\begin{align}
    dx_i = (x_{i+1} - x_i)/\sqrt{(x_{i+1} - x_i)^2 + (y_{i+1} - y_i)^2} \\
    dy_i = (y_{i+1} - y_i)/\sqrt{(x_{i+1} - x_i)^2 + (y_{i+1} - y_i)^2}
\end{align}
\end{ceqn}

Die Distanz zwischen zwei \textit{Flow-Vektoren} ist ihr euklidscher Abstand. Auf diese Weise wird bei
der Berechnung der Hausdorff-Distanz (siehe Abschnitt \ref{sec:hausdorff_distance}) auch die Richtung
der Trajektorien berücksichtigt.
Um ähnliche Sub-Trajektorien entdecken zu können, teilen Chen et al. die Trajektorien an den Positionen
``charakteristischer'' Vektoren. Diese beschreiben Richtungsänderungen in einer
Bewegungsbahn und werden identifiziert über die Abweichungen in den Richtungskomponenten zweier
aufeinanderfolgender Flow-Vektoren.
Veranschaulicht ist dies in Abbildung \ref{fig:relw_chen_flow_vector}.

\begin{figure}[H]
    \centering
    \includegraphics[width=0.45\linewidth]{resources/img/RelatedWork/Chen_trajectory_splitting}
    \caption[Zerlegung einer Trajektorie in Sub-Trajektorien (Chen et al.)]
            {Zerlegung einer Trajektorie in Sub-Trajektorien anhand von ``charakteristischen Flow-Vektoren'' \cite[]{Chen2011}}
    \label{fig:relw_chen_flow_vector}
\end{figure}

Die auf diese Weise ermittelten Sub-Trajektorien, werden von den Autoren mittels eines DBSCAN Algorithmus gruppiert.
Sie können so die üblichen Bewegungsbahnen von Hurrikans über dem Atlantik bestimmen.


% Vlachos et al., 2002
\subsubsection*{Discovering Similar Multidimensional Trajectories}
\label{sec:relw_vlachos}
Die Arbeit \cite[]{Vlachos2002} thematisiert nicht direkt die Clusteranalyse von Trajektorien sondern
beschäftigt sich mit dem Vergleich von Bewegungsbahnen im drei-dimensionalen Raum. Konkret ist ihr Ziel,
Trajektorien vergleichen zu können, welche etwa die Handbewegungen beim Ausführen von Zeichensprache beschreiben.
Hierzu definieren die Autoren erstmals die Grundversion des LCSS Distanzmaßes, welches in vielen Arbeiten,
unter anderem in \cite[]{Atev2006}, \cite[]{Buzan2004} und \cite[]{Chen2005} zum Einsatz kommt.
Auf dessen Basis erstellen sie ein Distanzmaß, welche es ermöglicht formgleiche aber im Raum verschobene Trajektorien zu finden.
Die Grundversion der LCSS-Distanz und ein darauf basierendes, einfaches Distanzmaß ist, nach Vlachos et al.,
bereits in Abschnitt \ref{sec:lcss_distance} vorgestellt worden.
Dieses Maß erweitern die Autoren zudem wie folgt:

\begin{ceqn}
\begin{align}
    D2_{LCSS}(\delta, \epsilon, A, B) = 1 - \underset{f_{c,d} \in F}{max}\ D_{LCSS}(\delta, \epsilon, A, f_{c,d}(B))
\end{align}
\end{ceqn}

Hierbei ist $F$ eine Menge von Translations-Funktionen, welche die Trajektorien entlang der Achsen verschieben.
Sie besitzen die Form

\begin{ceqn}
\begin{align}
    f_{c, d}(A) = ((a_{x, 1} + c, a_{y, 1} + d), ..., (a_{x, n} + c, a_{y, n} + d))
\end{align}
\end{ceqn}

Abbildung \ref{fig:relw_vlachos_translation} veranschaulicht die Funktionsweise des Distanzmaßes. Es eignet sich immer dann, wenn Trajektorien
mit ähnlicher Form gefunden werden sollen, welche zudem eine gewisse räumliche Verschiebung aufweisen können.
Diese kann über die Größe von $F$ gesteuert werden.

\begin{figure}[H]
    \centering
    \includegraphics[width=0.55\linewidth]{resources/img/RelatedWork/vlachos_translation}
    \caption[Verschiebung einer Trajektorie im Raum]{Verschiebung einer Trajektorie im Raum \cite[]{Vlachos2002}}
    \label{fig:relw_vlachos_translation}
\end{figure}

% Ren et al., 2014
\subsubsection*{Lane Detection in Video-Based Intelligent Transportation Monitoring via Fast Extracting and Clustering of Vehicle Motion Trajectories}

\cite[]{Ren2014} stellen in ihrer Arbeit ein Vorgehen zur Clusteranalyse von Trajektorien vor,
welches auf der \textit{Rough-Set}-Theorie beruht. Sie extrahieren Fahrzeugpositionen aus Aufnahmen stationärer
Überwachungskameras und stellen diese, wie die meisten Autoren, als Sequenzen zwei-dimensionaler
Punkte dar. Ihr Ziel ist anschließend, anhand einer Gruppierung der Fahrzeugtrajektorien,
die Spurmittelpunkte der Fahrbahnen zu bestimmen. Da ein nicht unerheblicher Anteil der Trajektorien Spurwechselvorgänge
enthält, welche eine Extraktion der Mittellinien erschweren, verwenden Ren et al. einen iterativen \textit{Rough-k-Means}
Algorithmus zur Gruppierung der Trajektorien. Hierbei wird jedes Cluster über eine obere und untere Approximation beschrieben.
Die untere Approximation enthält dabei die Trajektorien, welche eindeutig der Spur zugeordnet werden können.
Die obere Näherung hingegen auch jene, welche Spurwechsel et cetera beschreiben. Bei der Berechnung der Spurmitten, werden
die Trajektorien der unteren Approximation höher gewichtet, als die der oberen. Ein sehr ähnlicher Cluster-Ansatz wurde
bereits in \cite[]{Lingras2004} vorgestellt.
Die initialen Mittellinien bestimmen die Autoren anhand einer \textit{Aktivitäts}- oder \textit{Heat}-Map,
welche sie während der Extraktion der Fahrzeugpositionen erstellen. Die Clusteranzahl $k$ muss händisch definiert werden.

Als Maß für die Distanz zwischen einer Trajektorie $A_x$ und einer Spurmitte $c_i$ verwenden Ren et al. die
Hausdorff-Distanz $h(A_x, c_i)$. Für eine Trajektorie wird somit die nächste Mittellinie wie folgt gefunden:

\begin{ceqn}
\begin{align}
    h(A_x, c_m) = \underset{i = 1 ... k}{min}\ h(A_x, c_i)
\end{align}
\end{ceqn}

Hieraus ergibt sich die nachfolgende Definition für die Zuordnung der Bewegungsbahnen zu den Cluster-Näherungen:

\begin{ceqn}
\begin{align}
    \begin{cases}
        A_x \in \overline{C_m} \land A_x \in \overline{C_j} & \text{if } j \neq m \land \frac{h(A_x, c_j)}{h(A_x, c_m)} \leq \lambda \\
        A_x \in \underline{C_m} & \text{otherwise}
    \end{cases}
\end{align}
\end{ceqn}

Für den Grenzwert $\lambda$ gilt $1 \leq \lambda \leq 1.5$. $\overline{C_m}$ und $\underline{C_m}$ entsprechen der oberen und unteren
Näherung des $m$-ten Clusters und es gilt $\underline{C_m} \subseteq \overline{C_m}$.
Nachdem in jeder Iteration des Cluster-Vorgangs die Näherungen auf diese Weise bestimmt wurden, werden die neue Mittellinien
anhand Gleichung \ref{eq_ren_rough} errechnet.

\begin{ceqn}
\begin{align}
    \label{eq_ren_rough}
    c_i =
    \begin{cases}
        \frac{w_l \sum_{A_x \in \underline{C_i}} A_x}{|\underline{C_i}|} + \frac{(1 - w_l) \sum_{A_x \in (\overline{C_i} - \underline{C_i})} A_x}{|\overline{C_i} - \underline{C_i}|} & \text{if } \overline{C_i} \neq \underline{C_i} \\
        \frac{\sum_{A_x \in \underline{C_i}} A_x}{|\underline{C_i|}} & \text{otherwise}
    \end{cases}
\end{align}
\end{ceqn}

Als Gewichtungen $w_l$ verwenden Ren et al. Werte im Bereich $[0.5, 1]$. $| \cdot |$ entspricht hier der Kardinalität einer Menge.

Unter Verwendung dieser Clustering-Methode ist es den Autoren von \cite[]{Ren2014} möglich, auch bei einer hohen Anzahl von
Ausreißern und Spurwechselvorgängen, stabile Spurmittellinien zu bestimmen. Ergebnisse, welche dies zeigen, sind in Abbildung
\ref{fig:relw_ren_example_detection} dargestellt.

\begin{figure}[H]
    \centering
    \includegraphics[width=0.95\linewidth]{resources/img/RelatedWork/ren_examples_detection}
    \caption[Ergebnisse Spurmittellinien-Erkennung (Ren et al.)]
            {Ergebnisse Spurmittellinien-Erkennung auf unterschiedlichen Straßenabschnitten \cite[]{Ren2014}}
    \label{fig:relw_ren_example_detection}
\end{figure}


\section{Erkennung und Definition von Fahrspuren}
\label{sec:rw_lane_detection}

Ziel dieser Arbeit ist es, Fahrspuren zuverlässig in Videoaufnahmen erkennen zu können. Dieser Abschnitt
geht daher auf Veröffentlichungen mit ähnlichen Zielen ein.

Die meisten Veröffentlichungen in diesem Bereich lösen das Problem, indem sie nach visuellen Merkmalen,
primär Spurtrennlinien, in Videoaufnahmen suchen. Die Aufnahmen werden dabei entweder von stationären Kameras,
von bemannten oder unbemannten Luftfahrzeugen oder einem Fahrzeug selbst erstellt. Zur Extraktion der Merkmale werden
üblicherweise Methoden aus den Gebieten des maschinellen Sehen (\acrshort*{cv}) oder maschinellen Lernens (\acrshort*{ml}) verwendet.
Arbeiten, welche CV-basierte Ansätze verfolgen, stammen beispielsweise von \cite[]{Lai2000}, \cite[]{McCall2006} oder \cite[]{Aly2008}.
ML-gestützte Arbeiten wurden dahingegend unter anderem von \cite[]{Kim2008}, \cite[]{Gopalan2012} oder
\cite[]{Neven2018} veröffentlicht.
Ein häufiges Einsatzgebiet für die visuelle Spurerkennung über eine im Fahrzeug verbaute Kamera, sind Spurhalte- oder
Spurwechsel-Assistenten.
% Ein mögliches Anwendungsgebiet für solche Ansätze ist Beispiel die Entwicklung eines Spurhalte-Assistenten.

Da visuelle Ansätze, wie bereits zu Beginn der Arbeit erläutert, aufgrund von Verdeckungen oder Änderungen
in der Belichtung problematisch sind, werden nachfolgend ausschließlich Arbeiten vorgestellt, welche Fahrspuren
aus Trajektoriedaten extrahieren. Hierbei setzten die meisten als ersten Schritt auf eine Clusteranalyse von Trajektorien.
In der Repräsentation und Extraktion der Fahrspuren variieren die Ansätze hingegen.

% Fu et al. & Junejo et al.
%   Clustering und dann Bestimmung Envelopes basierend auf Varianz der Trajektorien in Cluster
\subsubsection*{A System for Learning Statistical Motion Patterns}
Die Arbeit \cite[]{WeimingHu2006} basiert auf dem bereits früher veröffentlichten Artikel \cite[]{Hu2005} der selben Autoren,
welcher oben beschrieben wurde. In dieser Publikation gehen Hu et al. genauer darauf ein, wie sie auf Basis
der Ergebnisse der Clusteranalyse, statistische Informationen über die Fahrbahnen berechnen.
Hierzu wird für jedes Cluster zuerst eine Referenz-Trajektorie $T_r$ bestimmt. Dies ist jene Bewegungsbahn,
bei welcher die Summe der Distanzen zu allen anderen Trajektorien des Clusters minimal ist. Das verwendete
Distanzmaß ist hierbei des selbe, welches auch bei der Clusteranalyse zum Einsatz kam.
Die Autoren berechnen anschließend für jede Trajektorie-Gruppe eine Kette Gausscher-Wahrscheinlichkeits-Verteilungen
$\{ \varphi_1, \varphi_2, ..., \varphi_l \}$, wobei $l$ die Anzahl der Punkte der Referenz-Trajektorie ist.
Für jedes $\varphi_i$ wird der Mittelwert und die Kovarianz, basierend auf den Trajektorie-Punkten, welche dem
$i$-ten Punkt von $T_r$ am nächsten liegen, berechnet.
Anhand der Kovarianz-Werte erstellen die Autoren Hüllen für die Fahrbahnen. Ergebnisse dieses Vorgehens sind in
Abbildung \ref{fig:relw_hu_example_envelope} dargestellt.

\begin{figure}[H]
    \centering
    \subfloat{{
        \includegraphics[align=c, width=0.49\linewidth]{resources/img/RelatedWork/hu_example_envelope}
    }}
    \subfloat{{
        \includegraphics[align=c, width=0.49\linewidth]{resources/img/RelatedWork/hu_example_envelope2}
    }}
    \caption[Spurhüllen in Hu et al.]{Trajektorien und ermittelte Spurhüllen in \cite[]{WeimingHu2006}}
    \label{fig:relw_hu_example_envelope}
\end{figure}

Anzumerken ist, dass die extrahierten Spurhüllen nicht die tatsächliche Form, insbesondere die Breite, einer Fahrspur
wiederspiegeln. Sie sind deutlich schmäler.


% Makris et al., 2005
\subsubsection*{Learning Semantic Scene Models From Observing Activity in Visual Surveillance}
In \cite[]{Makris2005} extrahieren die Autoren übliche Bewegungsbahnen von Fußgängern aus stationären Videoaufnahmen.
Sie verwenden hierzu keine klassische Clusteranalyse, sondern ein iteratives und adaptives Online Verfahren,
welches neue Bewegungstrajektorien automatisch bestehenden Routen zuordnet oder neue initialisiert,
falls zu hohe Abweichungen zwischen der Trajektorie und den existierenden Bahnen bestehen.
Routen definieren Makris et al. hierbei als eine Sequenz von Knoten, welche folgende Merkmale besitzen:

\begin{itemize}
    \item 2D-Mittelpunkt
    \item Gewichtung (Anzahl der Trajektorien im Bereich des Knoten)
    \item Zwei Hüllpunkte (Maximale- beziehungsweise Standardabweichung der Trajektorien der Route im Bereich des Knoten)
\end{itemize}

Abbildung \ref{fig:relw_hu_example_envelope} a) veranschaulicht die Definition einer Route.
Um aus einfachen Bewegungsbahnen Routen zu erstellen, verwenden die Autoren das nachfolgend beschriebene Verfahren,
welches dem agglomerativen Cluster-Ansatz ähnelt.

\begin{enumerate}
    \item Die erste Trajektorie initialisiert die erste Route.
    \item Neue Trajektorien werden mit bestehenden Bahnen abgeglichen und
    \begin{enumerate}
        \item bei Übereinstimmung wird Route aktualisiert, oder
        \item bei keiner Übereinstimmung wird neue Route erzeugt.
    \end{enumerate}
    \item Aktualisierte Routen werden auf definierten Knotenabstand $r$ resampled.
    \item Alle Routen werden miteinander verglichen und
    \begin{enumerate}
        \item bei einer Überlagerung der Routen werden diese fusioniert.
    \end{enumerate}
\end{enumerate}

Als Vergleichsmaß für die Trajektorien und Routen, verwenden Makris et al. die maximale Distanz zwischen einer Trajektorie und
einer Routen-Hülle. Diese Distanz muss sich unterhalb eines bestimmten Grenzwertes befinden, damit eine Trajektorie
einer Route zugeordnet wird. Geschieht dies, dann werden alle Hüll- und Mittelpunkte neue berechnet.
Ein Ergebnis des Verfahrens ist in Abbildung \ref{fig:relw_results_makris} b) dargestellt.

\begin{figure}[H]
    \centering
    \subfloat[]{{
        \includegraphics[align=c, width=0.4\linewidth]{resources/img/RelatedWork/makris_route}
    }}
    \qquad
    \subfloat[]{{
        \includegraphics[align=c, width=0.4\linewidth]{resources/img/RelatedWork/makris_result}
    }}
    \caption[Routen-Definition und Ergebnisse Routen-Erkennung (Makris et al.)]{a) Routen Definition, b) Ergebnisse Routen-Erkennung \cite[]{Makris2005}}
    \label{fig:relw_results_makris}
\end{figure}


% Morris et al., 2011
\subsubsection*{Trajectory Learning for Activity Understanding: Unsupervised, Multilevel, and Long-Term Adaptive Approach}

In \cite[]{Morris2011} stellen die Autoren ein dreistufiges Framework zur Auswertung von Fahrzeugtrajektorien vor.
Ziel des Frameworks ist es, die Bewegungsmuster von Objekten mittels eines erlernten Vokabulars beschreiben zu können
sowie Aktivitäten vorhersagen und Anomalien erkennen zu können.
In einem ersten Schritt werden daher sogenannte \textit{Points of Interests} (POI) identifiziert. Die von Morris
et al. untersuchten POI's sind die Eintritts-, Austritts- und Stop-Zonen innerhalb einer Szene.
Nach deren Bestimmung identifiziert das vorgestellte Framework übliche Bewegungsmuster in den Trajektorien
und definiert auf deren Basis anschließend Pfade.

Sich häufig wiederholende Bewegungsmuster werden von den Autoren über eine Clusteranalyse ermittelt.
Zum Vergleich der Trajektorien kommt das in Abschnitt \ref{sec:lcss_distance} vorgestellte LCSS Distanzmaß zum Einsatz
und als Clusteralgorithmus das Spectral-Clustering. Morris et al. verwenden statt des im Spectral-Clustering
üblicherweise eingesetzten k-Mean-Algorithmus, einen Fuzzy-C-Mean Ansatz, welcher für jede Trajektorie
einen Zugehörigkeitswert $u_{ik} \in [0, 1]$ zum Cluster $k$ bestimmt.
Anhand der Zugehörigkeiten der Bewegungsbahnen zu den Clustern werden Referenz-Trajektorien bestimmt.
Diese ergeben sich für jedes Cluster als gewichteter Durchschnitt aller Trajektorien. Als Gewichte
werden die Werte $u_{ik}$ verwendet. Die so erstellten Referenz-Linien sind in Abbildung
\ref{fig:relw_morris_results} a) zu sehen.

\begin{figure}[H]
    \centering
    \subfloat[]{{
        \includegraphics[align=c, width=0.35\linewidth]{resources/img/RelatedWork/morris_routes_paths_1}
    }}
    \qquad
    \subfloat[]{{
        \includegraphics[align=c, width=0.35\linewidth]{resources/img/RelatedWork/morris_routes_paths_2}
    }}
    \caption[Referenz-Trajektorien und Pfade (Morris et al.)]{a) Refernz-Trajektorien, b) Pfade basierend auf HMMs \cite[]{Morris2011}}
    \label{fig:relw_morris_results}
\end{figure}

Basierend auf den Clustern definieren Morris et al. außerdem Pfade, welche die räumliche und zeitliche Dynamik
der Fahrzeuge an einer bestimmten Stelle abbilden. Sie verwenden hierzu Hidden-Markov-Modelle (\acrshort*{hmm}),
welche sie mittels der Baum-Welch-Methode trainieren.
Die Ergebnisse dieses Ansatzes sind in Abbildung \ref{fig:relw_morris_results} b) dargestellt. Die
Ellipsen repräsentieren die HMM's, welche angeben, wie die erwartete Bewegung eines Fahrzeugs in einem
bestimmten Bereich aussieht.
Die in \cite[]{Morris2011} bestimmten Pfade ermöglichen es diverse Aussagen über das Verhalten der Fahrzeuge
zu treffen. Sie repräsentieren allerdings nicht die realen Spur-Geometrien eines Straßenabschnittes.

% Hsieh et al., 2006
\subsubsection*{Automatic Traffic Surveillance System for Vehicle Tracking and Classification}
% Heat Map --> Spurmittellinien --> Begrenzer aus Mittellinien bestimmen

\cite[]{Hsieh2006} stellen in ihrer Arbeit ein System zur Extraktion von Fahrzeugpositionen aus Aufnahmen
stationärer Verkehrskameras vor. Auf diesen aufbauend definieren sie einen Algorithmus, mit dessen Hilfe es
möglich ist, Spurmittel- und Begrenzungs-Linien zu entdecken.
Hierzu erzeugen Hsieh et al. ein Histogramm $H_{vehicle}(x,y)$, welches die Häufigkeit abbildet, das sich Fahrzeuge über eine
bestimmte Pixel-Position $(x, y)$ bewegen. Ein solches ist in Abbildung \ref{fig:relw_hsieh_results} a) dargestellt.
Die Autoren gehen davon aus, dass die Fahrzeugpositionen sich mehrheitlich in der Mitte einer Spur befinden und
ein Histogramm, welches für eine Fahrbahn mit $L_n$ Spuren erstellt wurde, $L_n$ Maxima in jeder Zeile besitzt,
welche die Spurmitten darstellen.
Sie isolieren daher die Maxima in $H_{vehicle}$ als Mittellinien. Es sei anschließend $C_{L_k}^{j}$ die $k$-te Spurmittellinie
in Zeile $j$ des Histogramms. Daraus berechnen Hsieh et al. die Spurbegrenzungslinien. Alle innen liegenden
Begrenzungen ergeben sich aus Gleichung \ref{eq_hsieh1}. $DL_{k}^{j}$ entspricht hierbei einem Punkt
in der $j$-ten Reihe der $k$-ten Begrenzung.

\begin{ceqn}
\begin{align}
\label{eq_hsieh1}
    DL_{k}^{j} = \frac{1}{2} (C_{L_{k-1}}^{j} + C_{L_{k}}^{j})
\end{align}
\end{ceqn}

Die Breite $w_{L_k}^{j}$ der $k$-ten Fahrspur ergibt sich zudem wie folgt:

\begin{ceqn}
\begin{align}
\label{eq_hsieh2}
    w_{L_k}^{j} = | C_{L_{k}}^{j} - C_{L_{k-1}}^{j} |
\end{align}
\end{ceqn}

Die Positionen der äußeren Spurbegrenzungen einer Fahrbahn ergeben sich für die $j$-te Zeile des Histogramms
aus den Gleichungen \ref{eq_hsieh3} und \ref{eq_hsieh4}.

\begin{ceqn}
\begin{align}
\label{eq_hsieh3}
    (x_{DL_0^j}, j) &= (x_{DL_1^j - w_{L_0}^j}, j) \\
\label{eq_hsieh4}
    (x_{DL_{N_L}^j}, j) &= (x_{DL_{N_{L-1}}^j + w_{L_0}^j}, j)
\end{align}
\end{ceqn}

Ein Beispiel für die Ergebnisse der Spurerkennung von Hsieh et al. ist in Abbildung \ref{fig:relw_hsieh_results} b) dargestellt.
Das Verfahren funktioniert gut, wenn Fahrspuren nebeneinader und parallel zueinander liegen. Die Dimensionen
sich kreuzender Fahrspuren können mit dem Verfahren beispielsweise allerdings nicht bestimmt werden.

\begin{figure}[H]
    \centering
    \subfloat[]{{
        \includegraphics[align=c, width=0.33\linewidth]{resources/img/RelatedWork/hsieh_histogram}
    }}
    \subfloat[]{{
        \includegraphics[align=c, width=0.68\linewidth]{resources/img/RelatedWork/hsieh_centers_and_borders}
    }}
    \caption[Ergebnisse Histogramm Erstellung und Spurextraktion (Hsieh et al.)]{a) Fahrzeugpositions Histogramm, b) Spurmittellinien und Spurbegrenzungslinien \cite[]{Hsieh2006}}
    \label{fig:relw_hsieh_results}
\end{figure}


\section{Defizite vorhandener Lösungen und benötigte Neuerungen}
\label{sec:rw_deficites}

Es existiert eine große Anzahl von Arbeiten, welche sich mit der Clusteranalyse von Trajektoriedaten
befasst. Die in Abschnitt \ref{sec:rw_clustering} vorgestellten Veröffentlichungen stellen nur eine
kleine Auswahl dar.
Die Arbeiten entwickeln Lösungen in verschiedensten Anwendungsgebieten wie der Verkehrs-, Wetter und Verhaltensanalyse
und suchen daher in den Trajektoriedaten nach unterschiedlichsten Mustern. Die geforderte Genauigkeit der
Clustering-Ergebnisse unterscheidet sich je nach Anwendungsfall ebenfalls stark.
Im Rahmen dieser Masterarbeit muss ein Verfahren identifiziert und entwickelt werden, welches Fahrspur-Cluster
in Trajektoriedaten zuverlässig identifizieren kann. Die Fahrspuren können hierbei unterschiedlichste
Geometrien aufweisen.

Ein Defizit vieler vorhandener Arbeiten ist, dass diese meist mit wenigen unterschiedlichen
Trajektoriedatensätzen arbeiten und die verwendeten Daten selten Defekte wie Unterbrechungen,
Ausreißer oder inkorrekte Positionsinformationen aufweisen, welche allerdings durch Fehler in der Fahrzeugverfolgung entstehen können.
In einigen Arbeit wird teilweise nur mit generierten Trajektoriedaten gearbeitet.
Da in der vorliegenden Thesis davon ausgegangen werden muss, dass all diese Probleme auftreten können,
muss ein zuverlässiges Verfahren zur Bereinigung von Trajektoriedaten entwickelt werden. Ohne einen solchen Vorverarbeitungschritt,
wäre ein akkurates Clustering nicht möglich.

Deutlich weniger Veröffentlichungen gibt es zum Thema der Identifikation von Fahrspuren. Zwar existieren
einige Arbeiten, welche Fahrspuren auf Basis von Trajektorien ermitteln, diese entsprechen aber
in den allermeisten Fällen nicht den realen Verläufen der Fahrbahnen. Häufig werden Fahrspuren
anhand statistischer Verfahren, wie in \cite[]{WeimingHu2006} oder in \cite[]{Teng2015} beschrieben, ermittelt.
\cite[]{Hsieh2006} ermitteln in ihrer Arbeit Spur-Begrenzungslinien, allerdings nur für kurze, parallele
Fahrspuren eines Autobahnabschnittes. Auch die von \cite[]{Liu2010}, \cite[]{Sochor2014} und \cite[]{Chen2013}
veröffentlichten Verfahren eignen sich nur zur Identifikation von parallelen Fahrspur-Geometrien,
welche von statischen Kameras aufgenommen werden.

Ziel dieser Arbeit ist es, Fahrspuren in unterschiedlichen Straßentopologien identifizieren zu können.
Die erkannten Spuren sollen außerdem den realen Abmaßen der Spuren auf der Straße bestmöglich entsprechen.
Da dies von keiner existierenden Arbeit geboten wird, muss das Verfahren hierzu selbst entwickelt werden.
Es muss zudem eine Lösung zur Partitionierung von Fahrspuren entwickelt werden, da sich Spuren
in dieser Arbeit nicht über längere Bereiche hinweg überlagern dürfen.

% Es muss zudem für jede
% automatisch erkannte Spur überprüft werden, ob es sich bei ihr tatsächlich um eine real Fahrspur handeln kann,
% oder ob ihr Verlauf im Verhältnis zu dem der anderen Bahnen nicht plausibel ist.

% In dieser Arbeit sollen reale Spur-Geometrien für unterschiedlichste Straßenabschnitte bestimmt werden,
% weshalb das hierzu eingesetzte Verfahren selbst entwickelt werden muss.
%!TEX root = ../Thesis.tex

\chapter{Untersuchung möglicher Straßentopologien}
\label{cha:street_topologies}

% Beschreibung möglicher (Auswahl) Straßentopologien und ihrer Herausforderungen für die Erkennung von Fahrbahnen
% Idealerweise immer Bild einer entsprechenden Topologie und der entsprechenden Roh-Trajektorien

% Erläuterung: Was ist eine "Fahrspur" in dieser Arbeit. (d.h. nicht notwendigerweise eine Spur, wie sie auf Straße markiert ist. Beschreibt übliche Fahrbahn eines Fahrzeugs?!)

% Landstraßen (ein / zweispurig),
\section{Landstraßen}

% einfachste Straßentopologie, daher einfach darin Fahrspuren zu identifizieren
% entweder klar separierte Fahrspuren, welche parallel zueinander laufen oder eine (breite) Spur, welche in beide Richtungen genutzt wird
% üblicherweise wenig Abzweigungen, Einfahren etc.

% Bild einer leicht gekrümmten / gewundenen Landstraße


% Autobahnen (inkl. Auffahrten, Abfahrten)
\section{Autobahnen}

% ähneln Landstraße, Spuren hier immer separiert. Fahrzeuge fahren in auf einer Spur in eine Richtung
% Bahnen üblicherweise breiter.
% Auffahrten und Abfahren existieren

% Bild Entennest


% Kreuzungen (inkl. Abbiegespuren)
\section{Kreuzungen}

% Fahrbahnen kreuzen sich, geregelt über Ampelanlagen oder Rechts-vor-Links
% Abbiegespuren
% Fahrspuren überlagern sich. Sinnvolle Aufteilung

% Kreisverkehre
\section{Kreisverkehre}

% 12 Bewegungsbahnen durch Kreisverkehr (ausgenommen 360Grad Wendungen etc.)
% schwierig zu definieren, wie fahrspuren durch Kreisverkehr verlaufen
% Welche partitionieren
% Werden alle Bahnen richtig erkannt? Viele Überdeckungen. Richtig geclusterd?
%!TEX root = ../Thesis.tex

\chapter{Konzeption des Spurerkennung-Moduls}
\label{cha:konzeption}

In diesem Kapitel der Arbeit wird das zu entwickelnde Teilmodul \textit{``Spurerkennung''}
der \acrshort*{mec}-View \textit{TrackerApplication} konzipiert. Hierzu wird zuerst dessen Rolle und Position im Gesamtkontext
der Anwendung betrachtet. Anschließend werden Anforderungen und ein Entwurf des Moduls aufgestellt.

\section{Überblick über das Gesamtsystem}

Das Modul \textit{Spurerkennung} dient der Erreichung der in Abschnitt
\ref{sec:motivation_goals} definierten Ziele. Erstellt wird es im Rahmen des MEC-View Teilprojektes
\textit{Luftbeobachtung} als Teilmodul der Anwendung \textit{TrackerApplication}.
Abbildung \ref{fig:concept_laneDetection_context} gibt einen Überblick über das System.
Es werden hierbei jene Module beziehungsweise Schritte vorgestellt, welche mit der Spurerkennung in
Zusammenhang stehen.

\begin{figure}[H]
    \centering
    \includegraphics[width=\linewidth]{../resources/img/konzeption/Context_LaneDetection}
    \caption{Kontext des Spurerkennung Moduls}
    \label{fig:concept_laneDetection_context}
\end{figure}

Die mithilfe von Drohnen erstellten Videoaufnahmen können in der MEC-View \textit{TrackerApplication}
verarbeitet und analysiert werden. In einem ersten Schritt \textit{``MapMatching''}, wird hierzu üblicherweise
ein Welt-Koordinatensystem in Metern definiert. Anschließend können die Positionen und Typen
der Fahrzeuge bestimmt werden. Diese ersten zwei Schritte, welche der Extraktion von Fahrzeuginformationen
dienen, sind in Abschnitt \ref{sec:position_extraction} genauer beschrieben.
Die Fahrzeuginformationen, insbesondere die Positionsinformationen, dienen anschließend dem \textit{Spurerkennung}-Modul
als Eingabe. Aus ihnen extrahierte Spurdaten können anschließend in der Anwendung visualisiert werden oder in Kombination
mit den Fahrzeuginformationen zur Analyse des Verkehrsflusses eingesetzt werden.


\section{Anforderungen an das Modul}
\label{sec:requirements}

In diesem Abschnitt werden die wichtigsten funktionalen und nicht funktionalen Anforderungen
des Moduls festgehalten.

\subsection{Funktionale Anforderungen}

\paragraph{Anforderung 1000 (Top-Level)}
Das \textit{Spurerkennungs}-Modul soll es ermöglichen, mithilfe der \textit{TrackerApplication}
automatisch Fahrspuren aus den Positionsinformationen von Fahrzeugen in Luftaufnahmen abzuleiten.

\paragraph{Anforderung 2000}
Das Modul soll die Erkennung von Fahrspuren in den in Kapitel \ref{cha:street_topologies} vorgestellten
Straßentopologien unterstützen.

\paragraph{Anforderung 2100}
Das Modul soll unabhängig vom Aufnahmewinkel der Kamera Fahrspuren zuverlässig aus Videoaufnahmen ableiten können. 

\paragraph{Anforderung 2200}
Das Modul soll Fahrspuren bei Überlagerungen sinnvoll partitionieren können.

\paragraph{Anforderung 2300}
Das Modul soll die Enden benachbarter und paralleler Fahrspuren angleichen. Dies dient ihrer Verwendung in der
Verkehrsfluss-Analyse.

\paragraph{Anforderung 2400}
Das Modul soll es ermöglichen, die aus den Trajektorien abgeleiteten Fahrspuren in der \textit{TrackingApplication}
zu visualisieren.

\subsection{Nicht funktionale Anforderungen}

\paragraph{Anforderung 3000}
Das \textit{Spurerkennungs}-Modul muss robust mit Ausreißern und Tracking-Fehlern in den Trajektorien umgehen können.

\paragraph{Anforderung 3100}
Die Performance des Spurerkennung-Vorgangs ist nicht von höchster Priorität. Eine Erkennung sollte allerdings
dennoch maximal wenige Minuten dauern.


\section{Entwurf des Moduls Spurerkennung}
\label{sec:design}

In diesem Abschnitt wird, basierend auf den Erkenntnissen der Literaturrecherche und den Anforderungen,
ein grober Entwurf des \textit{Spurerkennung}-Moduls vorgestellt.

Das Modul definiert primär einen Algorithmus, welcher aus Fahrzeuginformationen wie der Position
oder Geschwindigkeit von Fahrzeugen, Fahrspuren ableitet. Die Grundfunktionsweise dieses Algorithmus
ist in Abbildung \ref{fig:concept_laneDetection_activity} in Form eines Aktivitätsdiagrams dargestellt.

\begin{figure}[H]
    \centering
    \includegraphics[width=0.8\linewidth]{../resources/img/konzeption/activity_laneDetection}
    \caption{Grundstruktur des Spurerkennungs-Algorithmus}
    \label{fig:concept_laneDetection_activity}
\end{figure}

Die einzelnen Schritte des Algorithmus werden im \textit{Spurerkennungs}-Modul der \textit{TrackerApplication}
als einzelne Komponenten implementiert.

Die nachfolgenden Umsetzungskapitel beschreiben, wie die einzelnen Schritte des Algorithmus aus
Abbildung \ref{fig:concept_laneDetection_activity} konkret realisiert wurden und welche Probleme es hierbei
zu überwinden galt.
%!TEX root = ../Thesis.tex

\chapter{Clusteranalyse von Fahrzeugtrajektorien}
\label{cha:realisation_clustering}

In diesem Kapitel wird die Umsetzung der Clusteranalyse der Trajektorien vorgestellt.
Es wird zuerst darauf eingegangen, welche Trajektorie-Repräsentation gewählt wurde. Anschließend
werden die verschiedenen Schritte zur Bereinigung und Vorverarbeitung der Daten beschrieben, welche
in dieser Arbeit zum Einsatz kommen. Schlussendlich folgt die Erläuterung der eigentlichen Clusteranalyse.
Hier werden die verschiedenen untersuchten Ansätze vorgestellt und ihre Ergebnisse diskutiert.

Die \textit{TrackerApplication} ist in Java und Scala implementiert. Ihre Benutzeroberfläche basiert
auf JavaFX. Das in dieser Arbeit erstellte Modul \textit{Spurerkennung} wird komplett mit Scala umgesetzt. 

\section{Erstellen von Trajektorien}

Die Ergebnisse der Fahrzeugverfolgung (siehe Abschnitt \ref{sec:position_extraction}) werden
in der \textit{TrackingApplication} in Form sogenannter \textit{TrackedObject}`s gespeichert.
Ein solches Objekt repräsentiert eine zusammenhängende, nicht-unterbrochene Verfolgung eines Fahrzeugs.
Die wichtigsten Informationen, die ein \textit{TrackedObject} beinhaltet, sind eine eindeutige ID,
die Frame-Positionen des Starts und Endes der Verfolgung und die Objekt-Klasse des Fahrzeugs. Es wird
zwischen den vier Klassen \textit{``Auto''}, \textit{``Lastwagen''}, \textit{``Transporter''}
und \textit{``Zweirad''} unterschieden.
Für jedes verfolgte Objekt können die zugehörigen Positions-, Geschwindigkeits-, Beschleunigungs-
und Dimensions-Informationen abgerufen werden. Diese werden für jedes Frame, welches zwischen dem Start-
und End-Frame des Objektes liegt, bestimmt.

Da für die Ableitung von Fahrspuren aus Trajektorien lediglich die positionsbezogenen Eigenschaften
der Fahrzeuge relevant sind, werden Bewegungbahnen in dieser Arbeit über jene definiert.
Geschwindigkeit, Beschleunigung und Dimension der Fahrzeuge wird in der Clusteranalyse nicht berücksichtigt.
Abbildung \ref{fig:real_trajectory_classDia} zeigt den Aufbau einer Trajektorie im Modul \textit{Spurerkennung}.

\begin{figure}[H]
\centering
    \includegraphics[width=0.38\linewidth]{../resources/img/umsetzung/U1/Trajectory_ClassDia}
\caption{Aufbau Trajektorie-Klasse}
\label{fig:real_trajectory_classDia}
\end{figure}

Die Felder \textit{id} und \textit{objectClass} werden aus dem der Trajektorie zugrundeliegenden \textit{TrackedObject}
übernommen. 
Die Positionen eines Fahrzeugs werden in Form von 2D-Welt-Koordinaten (siehe Abschnitt \ref{sec:position_extraction})
in \textit{positions} gespeichert.
Die Sequenz \textit{distToStart} enthält für jeden Punkt der Bewegungsbahn die Distanz zum Start der Trajektorie in Metern.
Die Werte ergeben sich aus Formel \ref{eq_real_distToStart}, wobei $p_n$ dem $n$-ten Punkt in der Trajektorie entspricht
und $dist$ der euklidschen Distanz zwischen zwei Punkten.

\begin{ceqn}
\begin{align}
\label{eq_real_distToStart}
    distToStart(p_n) =
    \begin{cases}
        0 & \text{if } n = 0 \\
        dist(p_n,\ p_{n-1}) + distToStart(p_{n-1}) & \text{otherwise} 
    \end{cases}
\end{align}
\end{ceqn}

Aus \textit{distToStart} ergibt sich zudem die Gesamtlänge einer Trajektorie, welche extra gespeichert wird.
Das Feld \textit{clusterLabel} ordnet jede Trajektorie nach der Clusteranalyse einem bestimmten Cluster zu.
Zuvor enthält es keinen Wert.

% TODO: Evtl Bilder und Beschreibung austauschen
Zur Untersuchung der Fahrzeugtrajektorien ist es hilfreich diese zu visualisieren. Abbildung \ref{fig:real_trajs_raw_neckartor}
zeigt so beispielsweise 1240 Trajektorien, welche aus einer Aufnahme des Stuttgarter Neckartors extrahiert wurden.
In Abbildung \ref{fig:real_neckartor} ist ein Ausschnitt der entsprechenden Aufnahme zu sehen.

\begin{figure}[H]
\centering
    \includegraphics[width=0.55\linewidth]{../resources/img/umsetzung/U1/Plot_RawTrajectories_Neckartor}
\caption{Unverarbeitete Trajektorien vom Stuttgarter Neckartor}
\label{fig:real_trajs_raw_neckartor}
\end{figure}

\begin{figure}[H]
\centering
    \includegraphics[width=0.8\linewidth]{../resources/img/umsetzung/U1/Neckartor_Aufnahme}
\caption{Das Stuttgarter Neckartor}
\label{fig:real_neckartor}
\end{figure}

In Abbildung \ref{fig:real_trajs_raw_neckartor} sind die verschiedenen Bewegungsbahnen der Fahrzeuge für
den menschlichen Betrachter bereits klar erkennbar.
Direkt fallen aber auch die Trajektorien der stehenden oder sich auf Parkplätzen
bewegenden Autos im oberen Bereich der Aufnahme ins Auge. Diese dürfen nicht in die Clusteranalyse mit einbezogen werden.
Bei genauerer Untersuchung der Trajektorien zeigen sich weitere Probleme, welche das Clustering negativ
beeinflussen würden. Zwei sind in nachfolgender Abbildung dargestellt.
\ref{fig:real_defects_trajectories} a) zeigt, wie Fahrzeuge Punktwolken beim Stilstand vor Lichtsignalanlagen bilden.
In \ref{fig:real_defects_trajectories} b) wird deutlich, dass in manchen Bereichen sehr viele Trajektorie-Unterbrechungen
auftreten. Hier wird die Straße üblicherweise von Bäumen, Brücken et cetera überlagert.

\begin{figure}[H]
    \centering
    \subfloat[]{{
        \includegraphics[align=c, width=0.4\linewidth]{../resources/img/umsetzung/U1/trajectories_defect1}
    }}
    \subfloat[]{{
        \includegraphics[align=c, width=0.5\linewidth]{../resources/img/umsetzung/U1/trajectories_defect2}
    }}
    \caption{a) Punktwolken vor Lichtsignalanlagen, b) Unterbrechungen aufgrund von Überdeckung}
    \label{fig:real_defects_trajectories}
\end{figure}

Um von diesen und weiteren Effekten bei der Clusteranalyse nicht beeinflusst zu werden, durchlaufen die
``Roh-Trajektorien'' einen Vorverarbeitungsschritt. Dieser wird im nächsten Abschnitt vorgestellt.

\section{Vorverarbeitung der Trajektorien}
\label{sec:realisation_preprocessing}

% Aussortierung zu kurzer Trajektorien
%   Stehend, Teiltrajectorien etc.
% Resampling (wichtig Datenmenge)
% Aussortierung zu kurzer Trajektorien
% Prüfen isCompleteTraj.
%   Unterbrochene Traj. ausfiltern
% Trimmen Truck-Trajektories

% Wird eine Fahrzeugverfolgung, beispielsweise aufgrund von Überdeckungen, unterbrochen, so wird ein
% Kraftfahrzeug von mehreren \textit{TrackedObjects} repräsentiert, welche sich allerdings nicht einander
% zuordnen lassen.

\section{Clustering der Trajektorien}
\label{sec:realisation_clustering}

% genaue Beschreibung des Vorgehens, bis finale Clustering Lösung erreicht wurde
% Ansätze: Gründe, Stärken, tatsächliche Problem
% Ansatz A:
%   Mod. Hausdorff Distanz und Spectral Clustering (bas. auf Avet et al.) (Erklärung Grundfunktionsweise Spectral-Clustering)
%   Weil: SC performant, deterministisch, oft verwendet
%   Probleme: Clusteranzahl Bestimmung, Umgehen mit Ausreißern, Tatsächliche Ergebnisse nicht gut
%   --> Performance / Qualität des Ansatzes konnte für vorhandene Daten nicht bestätigt werden
%   --> Problematisch auch Umgang mit vielen Parametern
% Ansatz B:
%   LCSS Distanz (in anderen Papern gute Ergebnisse) (impl. mittels bottom up dyna. programmierung, Verwendung Eucl. Dist.)
%   Wieso D2 aus Vlachos et al. verwendet? (Verschiebung unerwünscht)
%   DBSCAN Clustering
%   --> DM kann besser mit Ausreißern umgehen und DBSCAN berücksichtigt diese auch
%   bessere Ergebnisse

% Probleme: Erkennung von Abbiegespuren, mit wenig Fahrzeugen und Abbiegevorgängen auf mehrere Spuren
%   Dafür: Weitere Verarbeitung der Clustering Outlier
%   Beschreibung Verfahren (dichte-basiert, suchen von initial Dichten Regionen, Verfolgung bis Ausdünnung)

\chapter{Fahrspur-Bestimmung aus Trajektorie-Clustern}
\label{cha:lane_definition}

% Cluster-Bereinigung: Entfernen von Outliern (Spurwechselvorgänge)
%   Beschreibung Probleme: Performance, Zuverlässigkeit (initial Distanzbasiert, erweitert Dichtebasiert --> Ähnliche Ergebnisse und Performance)
%   Verfahren mit evtl. besseren Resultaten noch aufwendiger und basierend meist auf selben Ideen (Distanzmaße, Dichten etc.)
% Bestimmung Referenz-Trajektorie
% Bestimmung von Spur-Envelopes
% Partitionierung der initialen Spur-Schätzungen
% Alignment der Spuren
%!TEX root = ../Thesis.tex

\chapter{Realisierung LaneDetection in MEC-View TrackerApplication Software}
\label{cha:realisation_tracker}
%!TEX root = ../Thesis.tex

\chapter{Ergebnisse und Auswertung}
\label{cha:results}

% Screenshots weiterer Straßenabschnitte
% Aufzeigen von Problemen und deren Ursachen
% Mögliche Weiterentwicklungen

Die Stärken, Schwächen und Ergebnisse des entwickelten Algorithmus werden im nachfolgenden Kapitel
zusammengefasst, diskutiert und ausgewertet. Es wird zuerst abschnittsweise auf die drei primären Schritte
des Verfahrens eingegangen. Anschließend werden Beispiele erkannter Fahrspuren aufgeführt.

\section{Evaluierung der Datenvorverarbeitung}

% Sehr wichtig für zuverlässige Funktionsweise der nachfolgenden Schritte.
% Kurz: Ziele (Bereinigung von Ausreißern, Reduktion der Komplexität)
% Funktioniert gut: Entfernt stehende oder unterbrochene Trajektorien, außerdem einzelne Ausreißer aufgrund von Tracking Fehlern 
% Beispiel Heilbronner Straße (Raw | filtered)

% Problematisch: Trajektorien einer Spur enden alle auf unterschiedlichen Höhen (--> Horizont)
% Algorithmus aus XXX würde viele (im Schlimmsten Fall alle) Trajektorien entfernen
% Beispiel Steinheim (RAW | Trajektorien an Horizont | Trimmed Trajs)
% Verweiß 

Die Datenvorverarbeitung ist ein wichtiger Teilschritt bei der Erkennung von Fahrspuren. Nur wenn aus
den Roh-Trajektorien die meisten Defekte entfernt wurden, können die nachfolgenden
Schritte zuverlässig funktionieren. Die meisten Defekte in den Roh-Trajektoriedaten werden durch stehende
Fahrzeuge und fehlerhafte oder unterbrochene Fahrzeugverfolgungen verursacht. Die angewandten Schritte zur
Entfernung der Anomalien sind in Abschnitt \ref{sec:realisation_preprocessing} beschrieben.

Dass die Entfernung von Ausreißern funktioniert, wurde bereits zum Teil in Abbildung
\ref{fig:real_result_2nd_Prepro} anhand des Neckartor-Trajaktoriedatensatzes gezeigt.
In Abbildung \ref{fig:results_prePro_heilbronner} sind nun die Trajektorien eines weiteren Datensatzes dargestellt,
welcher von der Heilbronner-Straße in Stuttgart stammt. 

\begin{figure}[H]
    \centering
    \subfloat[]{{
        \includegraphics[align=c, width=0.33\linewidth]{resources/img/results/Heilbronner/rawTrajectories}
    }}
    \qquad \qquad
    \subfloat[]{{
        \includegraphics[align=c, width=0.33\linewidth]{resources/img/results/Heilbronner/preProTrajs}
    }}
    \caption{Ergebnis Vorverarbeitung Heilbronner-Straße}
    \label{fig:results_prePro_heilbronner}
\end{figure}

Die zwei obigen Plots zeigen gut, dass die vielen in a) vorkommenden Defekte entfernt wurden. Von den
circa 1050 Roh-Trajektorien im ursprünglichen Datensatz bleiben nach der Vorverarbeitung etwa 450 intakte
Bewegungsbahnen übrig. Die Mehrzahl der Defekte in diesem Fall stammt von fehlerhaften Objekterkennungen
und unterbrochener Fahrzeug-Detektionen, welche aufgrund der Verdeckung von Fahrspuren durch Bäume entstehen.

Ein problematisches Verhalten des Vorverarbeitungsschrittes wurde beim Testen der Spurerkennung anhand eines
Datensatzes aus Steinheim deutlich. In Abbildung \ref{fig:results_horizon_problem} a) ist der
untersuchte Straßenabschnitt dargestellt.

\begin{figure}[H]
    \centering
    \subfloat[]{{
        \includegraphics[align=c, width=0.35\linewidth]{resources/img/results/Steinheim/steinheim}
    }}
    \qquad \qquad \qquad
    \subfloat[]{{
        \includegraphics[align=c, width=0.25\linewidth]{resources/img/results/Steinheim/preProTrajs_cut}
    }}
    \caption{Straßenausschnitt Steinheim a), gekürzte Trajektorien b)}
    \label{fig:results_horizon_problem}
\end{figure}

Kritisch an dieser Aufnahme ist, dass die Fahrzeuge, welche sich auf der oben linkes startenden Fahrbahn bewegen,
am Anfang beziehungsweise Ende ihrer Fahrt sehr klein sind. Da die Fahrzeuge in einer Aufnahme ab einer gewissen Größe
nurnoch sehr unzuverlässig detektiert werden, brechen die Trajektorien in diesem Fall auf sehr
unterschiedlichen Höhen ab.
Problematisch ist nun, dass der in Abschnitt \ref{sec:real1_remove_broken_trajectories} beschriebene Algorithmus
zur Entfernung unterbrochener Trajektorien, aufgrund der stark variierenden Start- und End-Positionen,
auch die meisten Trajektorien auf der nach hinten verlaufenden Fahrbahn entfernt und somit die Spuren nicht erkannt werden können.

Da die Fahrzeuge im Bereich des Horizonts nur unzuverlässig erkannt werden, ist auch eine Fahrverhaltensanalyse
mithilfe von Fahrspuren hier wenig sinnvoll. Es wurde daher entschieden dem Anwender die Möglichkeit zu geben, eine
Horizont-Linie zu definieren. In einem ersten Vorverarbeitungsschritt werden alle Trajektorie-Punkte oberhalb dieser
Linie entfernt. Das Ergebnis dieses Verfahrens ist in Abbildung \ref{fig:results_horizon_problem} b)
dargestellt. Dank der Beschneidung der Trajektorien bleiben diese in den nachfolgenden Verarbeitungsschritten
erhalten und es können Spuren im gewünschten Abschnitt erkannt werden.

\section{Evaluierung der Clusteranalyse}

\begin{figure}[H]
\centering
    \includegraphics[width=0.38\linewidth]{resources/img/results/Heilbronner/filteredClusters}
\caption{Trajektorie-Cluster Heilbronner Straße}
\label{fig:results_clusters_heilbronner}
\end{figure}

\begin{figure}[H]
    \centering
    \subfloat[]{{
        \includegraphics[align=c, width=0.35\linewidth]{resources/img/results/Neckartor/filteredClusters1}
    }}
    \qquad \qquad
    \subfloat[]{{
        \includegraphics[align=c, width=0.35\linewidth]{resources/img/results/Neckartor/filteredClusters2}
    }}
    \caption{Trajektorie-Cluster Neckator-Kreuzung}
    \label{fig:results_clusters_neckartor_filtered}
\end{figure}

\section{Evaluierung der Spur-Geometrie Bestimmung}

\section{Beispiele erkannter Fahrspuren}

%!TEX root = ../Thesis.tex

\chapter{Fazit und Ausblick}
\label{cha:end}


% % %%%%%% Literaturverzeichnis (darf im deutschen nicht in den Anhang!)
% Einfaches Literaturverzeichnis
% \input{chapters/bibEinfach}
% Literaturverzeichnis mit Bibtex
\bibliography{bib/sources}

\printglossary

% % %%%%%% Anhang
\appendix
% %!TEX root = ../Thesis.tex

\chapter{Anhang - Verwendete Datensätze}
\label{cha:anhang_a}

Nachfolgend sind Aufnahmen der in dieser Arbeit verwendeten Luftaufnahmen zu sehen.

\subsection*{Datensatz Entennest}

\begin{figure}[H]
\centering
    \includegraphics[width=0.6\linewidth]{resources/img/Anhang/Entennest}
\caption[]{Staßenabschnitt Aufnahme Entennest}
\label{fig:anhang_ds_entennest}
\end{figure}

\subsection*{Datensatz Neckartor}

\begin{figure}[H]
\centering
    \includegraphics[width=0.6\linewidth]{resources/img/Anhang/Neckartor}
\caption[]{Staßenabschnitt Aufnahme Neckartor}
\label{fig:anhang_ds_neckartor}
\end{figure}

\subsection*{Datensatz Heilbronner-Straße}

\begin{figure}[H]
\centering
    \includegraphics[width=0.6\linewidth]{resources/img/Anhang/Heilbronner}
\caption[]{Staßenabschnitt Aufnahme Heilbronner-Straße}
\label{fig:anhang_ds_heilbronner}
\end{figure}

\subsection*{Datensatz Düsseldorf}

\begin{figure}[H]
\centering
    \includegraphics[width=0.6\linewidth]{resources/img/Anhang/Duesseldorf}
\caption[]{Staßenabschnitt Aufnahme Düsseldorf}
\label{fig:anhang_ds_duesseldorf}
\end{figure}

\subsection*{Datensatz Steinheim}

\begin{figure}[H]
\centering
    \includegraphics[width=0.6\linewidth]{resources/img/Anhang/Steinheim}
\caption[]{Staßenabschnitt Aufnahme Steinheim}
\label{fig:anhang_ds_steinheim}
\end{figure}

% %  Inhalt ENDE %%%%%%%%%%%%%%%%%%%%%%%%%%%%%%%%%%%%%%%%%%%%%%%%%%%%%%%%%%
\end{document}
