
% \newacronym{ide}{IDE}{Integrierte Entwicklungsumgebung}
% \newacronym[longplural={Java Virtuellen Maschine}]{jvm}{JVM}{Java Virtuelle Maschine}
% \newacronym{}{}{}


\newacronym{mec}{MEC}{Mobile Edge Computing}
\newacronym{lcss}{LCSS}{Longest Common Subsequence}
\newacronym{dtw}{DTW}{Dynamic Time Warping}
\newacronym{raa}{RAA}{Richtlinien für die Anlage von Autobahnen}
\newacronym{ral}{RAL}{Richtlinien für die Anlage von Landstraßen}
\newacronym{em}{EM}{Expectation-Maximization}
\newacronym{gmm}{GMM}{Gaussian-Mixture-Models}
\newacronym{pca}{PCA}{Principal Component Analysis}
\newacronym{hd}{HD}{Hausdorff-Distanz}
\newacronym{dbscan}{DBSCAN}{Density-based spatial clustering of applications with noise}
\newacronym{hmm}{HMM}{Hidden Markov Model}
\newacronym{uav}{UAV}{Unmanned Aerial Vehicles}
\newacronym{roi}{ROI}{Region of Interest}
\newacronym{cnn}{CNN}{Convolutional Neural Networks}


% \newglossaryentry{enterprise}
% {
%   name={Enterprise Anwendung},
%   description={Bei Enterprise Anwendungen, zu deutsch ``Unternehmens-Anwendungen'', handelt es sich um Softwareprodukte, welche für gewöhnlich das Ziel haben, einem Unternehmen einen gewissen Mehrwert oder unternehmerischen Vorteil zu liefern. Für sie sind zumeist die Punkte Skalierbarkeit, Wartbarkeit, Trennung der Verantwortungen, Dokumentation und Sicherheit von besonderer Wichtigkeit.},
%   plural={Enterprise Anwendungen}
% }